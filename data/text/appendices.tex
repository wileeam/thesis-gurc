% -*- mode: TeX -*-
% -*- coding: utf-8 -*-

\begin{appendices}
    %\addappheadtotoc
    \appendixpage
    \renewcommand\thechapter{\greek{chapter}}
    \renewcommand\thesection{\thechapter.\greek{section}}
    \crefalias{section}{appsec}
    \crefalias{chapter}{appchp}

    %\chapter{Excerpts of \aclp*{tos} and \aclp*{pp}}
    \chapter{Excerpts of Terms of Services and Privacy Policies}
        \label{chapter:thesis:excerpts-of-tos-and-pp}
    
    % TODO: replace the word sections with a cleverref command
    \lettrine{\textcolor[gray]{.25}{I}}{n} the following \namecrefs{section:thesis:related-work} we quote some 
    selected excerpts of the \ac{tos} and \ac{pp} of a few \ac{osn} service providers 
    that we thought --- as computer scientists --- interesting to be aware of in 
    our realm of research.
    
    We include the last update date stated by the service provider of the \ac{tos} 
    and \ac{pp} as well as the corresponding \acp{url} where we originally found 
    the terms and policies --- however, the wording and even the \ac{url} are likely 
    to change at any time.

    \section[\Facebook]{\Facebook (\FacebookInc)}
        \label{section:thesis:excerpts-facebook}
    % \cite{Facebook15}
    \Facebook's \acs{tos} are called ``Statement of Rights and Responsibilities
    '' and available at \url{https://www.facebook.com/terms}. The last revision as of 
    this writing is dated on January 30th, 2015. The following is an excerpt of some 
    selected sections of the \ac{tos}.

    \begin{quote_tos}
        \[...\]
        \textbf{Privacy}

        Your privacy is very important to us. We designed our Data Policy to make important 
        disclosures about how you can use Facebook to share with others and how we collect 
        and can use your content and information. We encourage you to read the Data Policy, 
        and to use it to help you make informed decisions. 

        \vspace{\baselineskip}

        \textbf{Sharing Your Content and Information}

        You own all of the content and information you post on Facebook, and you can control 
        how it is shared through your privacy and application settings. In addition:

        \begin{enumerate}
            \item For content that is covered by intellectual property rights, like photos 
            and videos (IP content), you specifically give us the following permission, 
            subject to your privacy and application settings: you grant us a non-exclusive, 
            transferable, sub-licensable, royalty-free, worldwide license to use any IP 
            content that you post on or in connection with Facebook (IP License). This IP 
            License ends when you delete your IP content or your account unless your content 
            has been shared with others, and they have not deleted it.
    
            \item When you delete IP content, it is deleted in a manner similar to emptying 
            the recycle bin on a computer. However, you understand that removed content 
            may persist in backup copies for a reasonable period of time (but will not be 
            available to others).
    
            \item When you use an application, the application may ask for your permission 
            to access your content and information as well as content and information that 
            others have shared with you.  We require applications to respect your privacy, 
            and your agreement with that application will control how the application can 
            use, store, and transfer that content and information.  (To learn more about 
            Platform, including how you can control what information other people may share 
            with applications, read our Data Policy and Platform Page.)
    
            \item When you publish content or information using the Public setting, it means 
            that you are allowing everyone, including people off of Facebook, to access 
            and use that information, and to associate it with you (i.e., your name and 
            profile picture).
    
            \item We always appreciate your feedback or other suggestions about Facebook, 
            but you understand that we may use your feedback or suggestions without any 
            obligation to compensate you for them (just as you have no obligation to offer 
            them).
        \end{enumerate}
        \[...\]
    \end{quote_tos}

    \section[\LinkedIn]{\LinkedIn (\LinkedInCorp)}
        \label{section:thesis:excerpts-linkedin}
    % \cite{LinkedIn14}
    \LinkedIn's \acs{tos} are called ``User Agreement'' and available at \url{https://www.linkedin.com/legal/user-agreement}. 
    The last revision as of this writing is dated on October 23rd, 2014. The following 
    is an excerpt of some selected sections of the \ac{tos}.

    \begin{quote_tos}
        \[...\]
        \textbf{Your License to LinkedIn}
        
        As between you and LinkedIn, you own the content and information that you 
        submit or post to the Services and you are only granting LinkedIn the following 
        non-exclusive license: A worldwide, transferable and sublicensable right 
        to use, copy, modify, distribute, publish, and process, information and 
        content that you provide through our Services, without any further consent, 
        notice and/or compensation to you or others. These rights are limited in 
        the following ways:
        \begin{enumerate}[label=\alph*]
            \item You can end this license for specific content by deleting such 
            content from the Services, or generally by closing your account, except 
            (a) to the extent you shared it with others as part of the Service and 
            they copied or stored it and (b) for the reasonable time it takes to 
            remove from backup and other systems.
            \item We will not include your content in advertisements for the products 
            and services of others (including sponsored content) to others without 
            your separate consent. However, we have the right, without compensation 
            to you or others, to serve ads near your content and information, and 
            your comments on sponsored content may be visible as noted in the Privacy 
            Policy.
            \item We will get your consent if we want to give others the right to 
            publish your posts beyond the Service. However, other Members and/or 
            Visitors may access and share your content and information, consistent 
            with your settings and degree of connection with them.
            \item While we may edit and make formatting changes to your content 
            (such as translating it, modifying the size, layout or file type or 
            removing metadata), we will not modify the meaning of your expression.
            \item Because you own your content and information and we only have 
            non-exclusive rights to it, you may choose to make it available to others, 
            including under the terms of a Creative Commons license.
        \end{enumerate}
        You agree that we may access, store and use any information that you provide 
        in accordance with the terms of the Privacy Policy and your privacy settings.

        By submitting suggestions or other feedback regarding our Services to LinkedIn, 
        you agree that LinkedIn can use and share (but does not have to) such feedback 
        for any purpose without compensation to you.

        You agree to only provide content or information if that does not violate 
        the law nor anyone's rights (e.g., without violating any intellectual property 
        rights or breaching a contract). You also agree that your profile information 
        will be truthful. LinkedIn may be required by law to remove certain information 
        or content in certain countries.
        \[...\]
        \textbf{Limits}
        
        LinkedIn reserves the right to limit your use of the Services, including 
        the number of your connections and your ability to contact other Members. 
        LinkedIn reserves the right to restrict, suspend, or terminate your account 
        if LinkedIn believes that you may be in breach of this Agreement or law 
        or are misusing the Services (e.g. violating any Do and Don'ts).

        LinkedIn reserves all of its intellectual property rights in the Services. 
        For example, LinkedIn, SlideShare, LinkedIn (stylized), the SlideShare and 
        “in” logos and other LinkedIn trademarks, service marks, graphics, and logos 
        used in connection with LinkedIn are trademarks or registered trademarks 
        of LinkedIn. Other trademarks and logos used in connection with the Services 
        may be the trademarks of their respective owners
        \[...\]
    \end{quote_tos}

    \section[\Twitter]{\Twitter (\TwitterInc)}
        \label{section:thesis:excerpts-twitter}
    % \cite{Twitter16}
    \Twitter's \acs{tos} are called ``Twitter Terms of Service'' and available at 
    \url{https://twitter.com/tos}. The last revision as of this writing is dated 
    on September 30th, 2016. The following is an excerpt of some selected sections 
    of the \ac{tos}.

    \begin{quote_tos}
        \[...\]
        \textbf{Your Rights}
        
        You retain your rights to any Content you submit, post or display on or 
        through the Services. What’s yours is yours — you own your Content (and 
        your photos and videos are part of the Content).

        By submitting, posting or displaying Content on or through the Services, 
        you grant us a worldwide, non-exclusive, royalty-free license (with the 
        right to sublicense) to use, copy, reproduce, process, adapt, modify, publish, 
        transmit, display and distribute such Content in any and all media or distribution 
        methods (now known or later developed). This license authorizes us to make 
        your Content available to the rest of the world and to let others do the 
        same. You agree that this license includes the right for Twitter to provide, 
        promote, and improve the Services and to make Content submitted to or through 
        the Services available to other companies, organizations or individuals 
        for the syndication, broadcast, distribution, promotion or publication of 
        such Content on other media and services, subject to our terms and conditions 
        for such Content use. Such additional uses by Twitter, or other companies, 
        organizations or individuals, may be made with no compensation paid to you 
        with respect to the Content that you submit, post, transmit or otherwise 
        make available through the Services.

        Twitter has an evolving set of rules for how ecosystem partners can interact 
        with your Content on the Services. These rules exist to enable an open ecosystem 
        with your rights in mind. You understand that we may modify or adapt your 
        Content as it is distributed, syndicated, published, or broadcast by us and 
        our partners and/or make changes to your Content in order to adapt the Content 
        to different media. You represent and warrant that you have all the rights, 
        power and authority necessary to grant the rights granted herein to any 
        Content that you submit.
        \[...\]
    \end{quote_tos}
    
    \chapter{Additional remarks to \cref{article:thesis:passwords-peer-to-peer}: \usebibentry{KreitzBGRB12}{title}}
        \label{chapter:thesis:additional-remarks-p2p}
        % -*- mode: TeX -*-
% -*- coding: utf-8 -*-


        
    \chapter{Additional remarks to \cref{article:thesis:events-invitations-dosns}: \usebibentry{RodriguezCanoGB14}{title}}
        \label{chapter:thesis:additional-remarks-ei}
        % -*- mode: TeX -*-
% -*- coding: utf-8 -*-


\section{Clarifications}
    \label{section:thesis:appendix:ei:clarifications}
In the following list we try to further explain some words or sentences as published 
in the original text that otherwise could be misleading or even contradictory,
\begin{itemize}
    \item ``\acp{osn} have an infamous history of [...] issues...'' 
    and ``...popularity of these services...''\\
    \Acp{osn} have been on the frontpage of numerous media as responsible for intentional 
    and unintentional cases of data leakages, censorship and collaboration with 
    national governments to access personal data without consent --- see some examples 
    in the introduction. 
    
    However, such presumable bad publicity on management of their users' privacy 
    does not seem to affect their success. Every year, the service providers of 
    the major \acp{osn} report record numbers on their growth.
    
    Though publicity of such privacy breaches has made users of these services more 
    aware of their own privacy and implications of their data sharing activities 
    in the network.
    
    \item Number of public/private keypairs $rk_1/rk_1^S, \dots, rk_n/rk_n^S$\\
    The number of keypairs $n$ generated by the organiser \o{} to encrypt the 
    entries on the commit-list is sufficiently large to allow every invitee to be 
    able to commit to attend the event \eo{} --- in other words, $n >> |\I|$.
    
    Note that $n$ can not be the number of invited users, $|\I|$, because any invited 
    user could simply learn such number by counting the stored keys in the event 
    object \eo{}. 
    
\end{itemize}

\section{Assumptions}
    \label{section:thesis:appendix:ei:assumptions}
In the following list we make explicit some assumptions we implied in the original 
text as published or supplement some of those we did state in the text,
\begin{itemize}
    \item Communication among participants\\
    In our threat model described in \cref{subsection:event-invitations-dosns:threat-model}, 
    the exchanges of information --- messages --- among the participants in the 
    protocols should be assumed to be protected, for example, against traffic analysis. 
    
    If the traffic was not protected, a third party could easily infer who are the 
    invitees by simply observing the outgoing messages of the organiser when she 
    was issuing the invitations to the subset of invited users.
    
    Some traffic analysis countermeasure could be masking the communication channel 
    with dummy traffic, besides the communication channel itself being encrypted.

    \item Organiser trust\\
    %TODO Check the threat model section for type of malicious: honest-but-curious perhaps?
    Although our protocols are \ac{ttp}-free, we can not protect against an organiser 
    not following the protocol (or the participants for that matter). Therefore, there is some 
    certain trust implied in either party, and reason for us to provide our commit-disclose 
    protocol technique to, at least, identify misbehavior --- but unfortunately, 
    we cannot prevent them because there is no technology that can coerce someone 
    into doing something they don't want to do.

    \item Size of protocols' objects\\
    The largest object in our protocol is the event \eo{}, which accounts for various 
    data as depicted in \cref{figure:event-invitations-dosns:overview-objects-actions}. Namely, 
    publick keys: \e{}, \o{} and $rk_1/rk_1^S, \dots, rk_n/rk_n^S$, descriptions: 
    \dP{} and \dS{}, links: \ILL{}, \ILK{}, \CLL{} and \DLL{}, and privacy settings.
    
    The largest output of a secure and commonly used hash function, SHA-2, 
    is 512 bits. Moreover, a common key size\footnote{As of January 2016, the recommended 
    key size by the \ac{nsa} when using the RSA algorithm for digital signatures 
    is 3048 bits --- see \cite{iad16}} used in public/private cryptosystems such 
    as RSA is 2048 bits.
    
    Therefore, if we consider a key size for the public/private keypairs of 2048 bits, 
    we assume that the links are represented by a hash of 512 bits in our storage 
    system, and the descriptions and privacy settings are also represented as links, 
    we have that the size of the event object \eo{} is, at least, $(n + 3)*2048 + 3*512 = (n + 2)*2048 + 4*512 + 2*512 + 512$ 
    bits, excluding file format encoding and encryption overhead.
    
    For example, for $n = 100$, an \eo{} is less than 27 kilobytes. Even if we would 
    embed the descriptions and privacy settings into the object and consider overheads, 
    the size of \eo{} is rather low. Similarly, the size of the different lists 
    can be calculated in the same manner and expected to be low.
    
\end{itemize}

\section{Corrections}
    \label{section:thesis:appendix:ei:corrections}
\begin{itemize}
    \item Where in \cref{subsection:event-invitations-dosns:security-properties} 
    says ``Property IIP and ICP are closely related in the sense that if IIP holds 
    for a certain set of users, then...'' it should say ``Property IIP and ICP are 
    closely related in the sense that if IIP holds relative to a certain set of 
    users, then...''
\end{itemize}

        
    \chapter{Additional remarks to \cref{article:thesis:document-submission-system}: \usebibentry{GreschbachREB15}{title}}
        \label{chapter:thesis:additional-remarks-dss}
        % -*- mode: TeX -*-
% -*- coding: utf-8 -*-


\section{Clarifications}
    \label{section:thesis:appendix:dss:clarifications}
In the following list we try to further explain some words or sentences as published 
in the original text that otherwise could be misleading or even contradictory,
\begin{itemize}
    \item Stylometry\\
    In \cref{section:document-submission-system:anonymous-document-submission-system} 
    we assume that an adversary does not have the resources to achieve authorship 
    attribution by analysis of the linguistic style of writing --- hence, our protocol 
    does not protect against stylometry.
    
    In our scenario, we believe that such assumption is realistic enough because 
    our protocol is agnostic to the content of the submitted document. Reducing 
    the disclosure of information is out of the scope of our solution and a problem 
    on its own.
    
    We believe that the privacy-preserving and anonymisation protocol that we propose 
    is a necessary first step though we acknowledge that it is not sufficient.
    
    \item \Ac{dos} attacks\\
    There is a possibility for \ac{dos} attacks on the server side while the submission 
    of documents is in progress because the server does not know who submits what 
    due to blinding nor from where due to the anonymous communication channel.
    
    Although this is future work, there are some solutions to mitigate selective 
    \ac{dos} attacks in our scenario of anonymous communication channels. For example, 
    in \cite{DasB13}, or even in \cite{JansenTJS14} a 
    
    We have yet to explore mitigating solutions in our scenario of a centralised 
    storage where the stored files are unidentifiable during a period of time because 
    the server's storage is subject to flooding in such case.
    
    \item Popularity of blind signatures\\
    Blind signatures are widely used in the academic area of cryptography, for example, 
    in voting protocols or digital currencies. 
\end{itemize}

\section{Assumptions}
    \label{section:thesis:appendix:dss:assumptions}
In the following list we make explicit some assumptions we implied in the original 
text as published or supplement some of those we did state in the text,
\begin{itemize}
    \item Compromising capabilities of the adversary\\
    In some of our scenarios, we assume that the adversary has the capability to 
    compomise the server where the proposed \Ac{adss} is running. However, such 
    compomising capabilities are limited to the application level where our proposed 
    solution lies. For example, the adversary cannot affect the functionality or 
    security of the underlying \ac{os} primitives used by our system such as the 
    entropy of the \acp{rng}.
    
    \item Randomness of the random blinding factors\\
    It is in the student's best interest to randomly choose the random blinding 
    factors --- $b_{pass}$ and $b_{fail}$, otherwise a student who is willing to 
    disclose her identity may choose such factors in a less random manner, for example, 
    to obtain some more favorable grade.
    
\end{itemize}



        
    \chapter{Creative Commons Legal Code}
        \label{chapter:thesis:creative-commons-legal-code}
        % hacks to hide the section and subsection commands that are issued as part of the \doclicenseFullText macro of the doclicense package
        \renewcommand{\section}[1]{%
            \par\refstepcounter{section}% Increase section counter
            \sectionmark{#1}% Add section mark (header)
            % \addcontentsline{toc}{section}{\protect\numberline{\thesection}#1}% Add section to ToC
            % Add more content here, if needed.
        }
        \renewcommand{\subsection}[1]{%
            \par\refstepcounter{subsection}% Increase subsection counter
            \subsectionmark{#1}% Add subsection mark (header)
            % \addcontentsline{toc}{subsection}{\protect\numberline{\thesubsection}#1}% Add subsection to ToC
            % Add more content here, if needed.
        }
        The following is the legal code for the \doclicenseLongNameRef~license created 
        by \Ac{cc}, a global non-profit organization whose mission is to enable the 
        ``sharing and reuse of creativity and knowledge through the provision of 
        free legal tools''.
        {
        % redefine the quotation environment to override the default format
        \renewenvironment{quotation}{%
            \definecolor{silver}{rgb}{0.83, 0.83, 0.83}
            \setlength{\parindent}{0pt}
            \def\FrameCommand{%
                \hspace{1pt}%
                %{\color{DarkBlue}\vrule width 2pt}%
                %{\color{formalshade}\vrule width 4pt}%
                \colorbox{silver}%
            }%
            \MakeFramed{\advance\hsize-\width\FrameRestore}%
            \noindent\hspace{-4.55pt}% disable indenting first paragraph
            \begin{adjustwidth}{}{0pt}%
                \vspace{2pt}\vspace{2pt}%
        }
        {%
            \vspace{2pt}\end{adjustwidth}\endMakeFramed%
        }
        \doclicenseFullText
        }

\end{appendices}