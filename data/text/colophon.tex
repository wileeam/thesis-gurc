% -*- mode: TeX -*-
% -*- coding: utf-8 -*-

This thesis was typeset at 11:22:33 UTC on April 30th, 2017 from the git revision 
XXX of its main repository\footnote{A \emph{shallow} clone --- with \texttt{-{}-depth 
1} and the corresponding git revision annotated with the tag \texttt{vLicentiate} 
--- is available at \url{https://github.com/wileeam/thesis-gurc}.}
with \hologo{LaTeX2e} by means of the \hologo{TeX} engine \hologo{pdfTeX} included 
in the Mac\hologo{TeX} redistribution of \hologo{TeX} Live.

We used Hermann Zapf’s Palatino and Euler type faces --- in particular, Type 1 PostScript 
fonts URW Palladio L and FPL. Other fonts that we used were the Carlito and the 
Ubuntu font families.

The typographic style is based on the custom template\footnote{See \url{ftp://ftp.nada.kth.se/pub/tex/local/index.html} 
for further details.} made by Lars Engebretsen for the purpose of writing academic 
theses --- master, licentiate and doctoral ones --- adapted for the corporate design 
of KTH Royal Institute of Technology. The template is in turn a heavily configured 
version of the classical \emph{memoir} class packaged with most distributions of 
\hologo{TeX}. 

The headings of the chapters are adapted from an example found on the \Internet 
as so was the design of the part separators --- see the source code for further 
credits and details. Other typographic styles and modifications are inspired by 
the Chicago Manual of Style \cite{Chicago10}.

% LATEX2ε using Hermann Zapf’s Palatino and Euler type faces (Type 1
% PostScript fonts URW Palladio L and FPL were used). The listings are
% typeset in Bera Mono, originally developed by Bitstream, Inc. as “Bitstream
% Vera”. (Type 1 PostScript fonts were made available by Malte
% Rosenau and Ulrich Dirr.)
%
% The typographic style was inspired by Bringhurst’s genius as presented
% in The Elements of Typographic Style [157]. The LATEX package
% classicthesis has been made by André Miede. The colors of the
% package have been adapted for the corporate design of the University
% of Bern.
%
% note: The custom size of the text block was calculated using the
% directions given by Mr. Bringhurst (pages 26–29 and 175/176). 10 pt
% Palatino needs 133.21 pt for the string “abcdefghijklmnopqrstuvwxyz”.
% This yields a good line length between 24–26 pc (288–312 pt). Using
% a “double square text block” with a 1:2 ratio this results in a text block
% of 312:624 pt (which includes the headline in this design). A good
% alternative would be the “golden section text block” with a ratio of 1:1.62,
% here 312:505.44 pt. For comparison, DIV9 of the typearea package
% results in a line length of 389 pt (32.4 pc), which is by far too long.
% However, this information will only be of interest for hardcore pseudotypographers
% like me.

This is the final version as of 11:22:33 UTC on April 30th, 2017.