% -*- mode: TeX -*-
% -*- coding: utf-8 -*-


    % What is the problem?
    % Why is it a problem?
    % Why should we care?
    % What is our approach?
    % What are our findings?

    % Fix for bad grupin in lettrine package. See http://tex.stackexchange.com/questions/134371/lettrine-in-abstract-strange-behaviour
    % What is the problem?
    % Threats (to privacy) posed by centralised ISs to their users (and their data)
    % Why is it a problem?
    % (a) Massive collection and control of personal (and sensitive) data under one single player (or an oligpole)
    % {
    % \lettrine{\textcolor[gray]{.25}{P}}{rivacy enhancing technologies}\csname@restorepar\endcsname\everypar{}
    % have proven to be a beneficial area of research aiming at lessening the threats
    % to the privacy of users' personal information in centralised \aclp*{is} such
    % as \aclp*{osn}. In consequence, decentralised solutions have been proposed to
    % extend the control that users have over their data as opposed to the centralised
    % massive collection of personal and sensitive data.\par
    % }
    
    {
    \lettrine{\textcolor[gray]{.25}{L}}{as} tecnologías para mejorar la privacidad 
    --- en inglés, \aclp*{pet} --- han demostrado ser un área de investigación cuyo 
    objetivo es disminuir las amenazas a la privacidad de la información personal 
    de los usuarios en sistemas de información centralizados como los servicios 
    de redes sociales on line --- en inglés, \aclp*{osn} ---. Por ello, se han propuesto 
    soluciones descentralizadas para ampliar el control que los usuarios ejercen 
    sobre sus datos en contraposición a la recogida de datos personales y sensibles 
    en sistemas centralizados.\par
    }
    
    % Why should we care? (Why is it a problem?)
    % (b) Power of centralised service provider 
    % (c) Lack of control
    % (d) Threat to the right to be left alone
    Casos de mal uso, censura o incluso fuga de datos demuestran que el poder del 
    proveedor de servicios en sistemas de información centralizados disminuye la 
    privacidad del usuario. Las revelaciones en 2013 de un programa de vigilancia 
    a nivel global dirigido por agencias de inteligencia públicas en colaboración 
    con algunos de los proveedores de servicios de sistemas de información centralizados 
    ha acelerado el debate sobre las medidas a tomar para contrarrestar las amenazas 
    a la privacidad. En particular, la amenaza al ``derecho a la soledad'' --- en 
    inglés, ``right to be let alone''--- enunciado por Samuel Warren y Louis Brandeis 
    en 1890 en el influyente artículo legal, ``El derecho a la privacidad''.
    
    % What is our approach?
    % Decentralisation by means of privacy-preserving decentralised systems
    % Trade-offs: feature replication and security/privacy protection
    Los sistemas descentralizados que preservan la privacidad son soluciones viables  
    ante las amenazas a la privacidad, y una de las alternativas más comunes en 
    la actualidad. Sin embargo, la supresión de la autoridad central conlleva tratar 
    de resolver dos inconvenientes: replicar la funcionalidad de los sistemas 
    de información centralizados de forma que sean utilizables y asumir la vigilancia  
    de las amenazas a la seguridad y privacidad que anteriormente eran responsabilidad 
    de la autoridad central.
    
    % What are our findings? (What is our approach?)
    % Privacy-preserving decentralised systems by means of examples developing the functionality and showing the concepts of decentralisation and security and privacy properties.
    % Brief explanation of the three solutions
    En nuestra tesis, proponemos el uso de sistemas descentralizados que preservan 
    la privacidad y para ello desarrollamos tres soluciones a los sistemas de información 
    centralizados desde los puntos de vista de descentralización, funcionalidad 
    y, seguridad y privacidad. En los sistemas de información descentralizados, 
    diseñamos un mecanismo de autenticación de usuarios mediante el uso de credenciales 
    estándar usuario-contraseña cuya usabilidad es comparable a las aplicaciones 
    en sistemas centralizados. En el ámbito más práctico de los sistemas descentralizados 
    mostramos un ejemplo específico en el área de las redes sociales on line descentralizadas 
    --- en inglés, \aclp*{dosn} --- implementando un mecanismo de coordinación y 
    cooperación para la organización de eventos sin necesidad de existencia de un 
    tercero de confianza. Finalmente, en los sistemas de información centralizados 
    en los que la presencia de una autoridad central sigue siendo necesaria intentamos 
    mejorar uno de los aspectos de la privacidad del usuario: el anonimato, diseñando 
    e implementando un sistema para presentar y evaluar documentos de forma anónima 
    en el ámbito académico en un sistema de información genérico y centralizado.
    
    % What are our findings?
    % Privacy-preserving protocols in centralised and decentralised systems that mitigate the dangers to personal privacy
    Las soluciones que proponemos son algunos ejemplos concretos del concepto de 
    ``privacidad como control de datos'' --- en inglés, ``privacy as data control''--- 
    tal y como lo definió Anita Allen. Un paradigma que se puede conseguir en diversos 
    niveles tanto en sistemas de información centralizados como descentralizados. 
    No obstante, deseamos que los protocolos para preservar la privacidad que proponemos 
    junto con la evaluación de las propiedades de seguridad y privacidad sean de 
    utilidad en otros ámbitos para contribuir a mitigar las diversas amenazas a 
    la privacidad a las que no enfrentamos en la actualidad.
    