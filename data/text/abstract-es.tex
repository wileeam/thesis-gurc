% -*- mode: TeX -*-
% -*- coding: utf-8 -*-


    % What is the problem?
    % Why is it a problem?
    % Why should we care?
    % What is our approach?
    % What are our findings?

    % Fix for bad grupin in lettrine package. See http://tex.stackexchange.com/questions/134371/lettrine-in-abstract-strange-behaviour
    % What is the problem?
    % Threats (to privacy) posed by centralised ISs to their users (and their data)
    % Why is it a problem?
    % (a) Massive collection and control of personal (and sensitive) data under one single player (or an oligpole)
    % {
    % \lettrine{\textcolor[gray]{.25}{P}}{rivacy enhancing technologies}\csname@restorepar\endcsname\everypar{}
    % have proven to be a beneficial area of research aiming at lessening the threats
    % to the privacy of users' personal information in centralised \aclp*{is} such
    % as \aclp*{osn}. In consequence, decentralised solutions have been proposed to
    % extend the control that users have over their data as opposed to the centralised
    % massive collection of personal and sensitive data.\par
    % }
    
    {
    \lettrine{\textcolor[gray]{.25}{L}}{as} tecnologías para mejorar la privacidad 
    --- en inglés, \aclp*{pet} --- han demostrado ser un área de investigación cuyo 
    objetivo es disminuir las amenazas a la privacidad de la información personal 
    de los usuarios en sistemas de información centralizados como los servicios 
    de redes sociales on line --- en inglés, \aclp*{osn} ---. Por ello, se han propuesto 
    soluciones descentralizadas para ampliar el control que los usuarios ejercen 
    sobre sus datos en contraposición a la recogida de datos personales y sensibles 
    en sistemas centralizados.\par
    }
    
    % Why should we care? (Why is it a problem?)
    % (b) Power of centralised service provider 
    % (c) Lack of control
    % (d) Threat to the right to be left alone
    The power that the service provider has in centralised \aclp*{is} has been shown 
    to diminish the user's privacy with cases of misuse, censorship or data leakage. 
    Moreover, the disclosures in 2013 of a global surveillance program led by public 
    intelligence institutions in collaboration with some of the service providers 
    of such centralised \aclp*{is} has accelerated the debate on how to take action 
    to counteract the threats to privacy. In particular, the threat to the legal 
    ``right to be let alone'' as defined by Samuel Warren and Louis Brandeis in 
    1890 in their influential law review article ``The Right to Privacy''.
    
    Se ha demostrado que el poder del proveedor de servicios en sistemas de información 
    centralizados disminuye la privacidad del usuario con casos de mal uso, censura 
    o incluso fuga de datos. Además, las revelaciones en 2013 de un programa de 
    vigilancia a nivel global dirigido por agencias de inteligencia públicas en 
    colaboración con algunos de los proveedores de servicios de sistemas de información 
    centralizados ha acelerado el debate sobre las medidas a tomar para contrarrestar 
    las amenazas a la privacidad. En particular, la amenaza al ``derecho a la soledad'' 
    --- en inglés, ``right to be let alone''--- enunciado por Samuel Warren y Louis 
    Brandeis en 1890 en el influyente artículo legal, ``El derecho a la privacidad''.
    
    % What is our approach?
    % Decentralisation by means of privacy-preserving decentralised systems
    % Trade-offs: feature replication and security/privacy protection
    Privacy-preserving decentralised systems are plausible solutions to such threats 
    and one of the most common alternatives used nowadays. However, the removal 
    of the central authority comes with two main trade-offs to be tackled, mimicking 
    the features of the centralised \acl*{is} in a usable way and taking over the 
    supervision of the security and privacy threats that once were a responsibility 
    of the central authority.
    
    Los sistemas descentralizados que preservan la privacidad son soluciones viables  
    contra las amenazas a la privacidad y una de las alternativas de uso común en 
    la actualidad. Sin embargo, eliminar a la autoridad central implica tener que 
    afrontar dos inconvenientes, m las características del sistema de 
    información centralizado de una manera utilizable y asumiendo la supervisión 
    de las amenazas de seguridad y privacidad que una vez eran responsabilidad de 
    la central autoridad.
    
    % What are our findings? (What is our approach?)
    % Privacy-preserving decentralised systems by means of examples developing the functionality and showing the concepts of decentralisation and security and privacy properties.
    % Brief explanation of the three solutions
    In our thesis, we propose the use of privacy-preserving decentralised systems 
    and develop three solutions to centralised \aclp*{is} in terms of decentralisation, 
    functionality and, achievable security and privacy. For decentralised \aclp*{is} 
    in general we show a concrete mechanism for user authentication via standard 
    user-password credentials with comparable usability to standard centralised 
    applications. Within the realm of practical decentralised systems we show a 
    specific example in the domain of \aclp*{dosn} implementing a coordination and 
    cooperation mechanism to organise events without the need of a trusted third 
    party. Finally, we step back to those centralised systems where the presence 
    of the central authority is still required and, instead, improve one of the 
    aspects of the user's privacy: anonymity, by showing an implementation of a 
    system to submit and grade documents anonymously in the academic sphere in a 
    generic centralised privacy-preserving system.
    
    En nuestra tesis proponemos el uso de sistemas descentralizados que preservan 
    la privacidad y desarrollamos tres soluciones a sistemas centralizados de información 
    en términos de descentralización, funcionalidad y seguridad alcanzable y privacidad. 
    Para los sistemas de información descentralizados en general, mostramos un mecanismo 
    concreto para la autenticación de usuarios a través de credenciales de contraseña 
    de usuario estándar con una usabilidad comparable a las aplicaciones centralizadas 
    estándar. En el ámbito de los sistemas descentralizados prácticos se muestra 
    un ejemplo específico en el ámbito de las redes sociales descentralizadas --- en inglés, \aclp*{dosn} --- en 
    línea que implementan un mecanismo de coordinación y cooperación para organizar 
    eventos sin la necesidad de un tercero de confianza. Finalmente, retrocedemos 
    a aquellos sistemas centralizados donde la presencia de la autoridad central 
    sigue siendo necesaria y, en cambio, mejoran uno de los aspectos de la privacidad 
    del usuario: el anonimato, mostrando la implementación de un sistema para presentar 
    y clasificar los documentos anónimamente En el ámbito académico en un sistema 
    genérico centralizado de preservación de la privacidad.
    
    % What are our findings?
    % Privacy-preserving protocols in centralised and decentralised systems that mitigate the dangers to personal privacy
    Our solutions are some concrete examples of how privacy as data control, as 
    the paradigm envisioned by Anita Allen, can be achieved to varying degrees in 
    privacy-preserving centralised and decentralised \aclp*{is}. Nonetheless, we 
    hope that the privacy-preserving protocols we propose and the evaluation of 
    the security and privacy properties can be useful in other scenarios as such 
    to mitigate the diverse dangers to personal privacy that we are facing at the 
    present times.
    
    Nuestras soluciones son algunos ejemplos concretos de cómo la privacidad como 
    control de datos, como el paradigma imaginado por Anita Allen, puede lograrse 
    en diversos grados en los sistemas de información centralizados y descentralizados 
    que preservan la privacidad. No obstante, esperamos que los protocolos de preservación 
    de la privacidad que proponemos y la evaluación de las propiedades de seguridad 
    y privacidad puedan ser útiles en otros escenarios como tales para mitigar los 
    diversos peligros a la privacidad personal que estamos enfrentando en los tiempos 
    actuales.
    