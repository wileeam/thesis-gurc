% -*- mode: TeX -*-
% -*- coding: utf-8 -*-

\requote{If something is important enough, even if the odds are against you, you 
should still do it}{Elon Musk}
% \requote{Persistence is very important. You should not give up unless you are forced
% to give up}{Elon Musk}

At the beginning of October in 2011 a Spaniard from Valladolid landed once more 
at the busiest Swedish airport --- Stockholm Arlanda Airport --- for a new adventure. 
Such adventure would take him South to the capital, Stockholm, instead of the usual 
journey North to the always beloved Uppsala of previous years.

That guy --- the author of this work --- came to Sweden with the idea of doing this 
thing called Ph.D. --- \emph{Philosophiae Doctor}. Back then, I did not know what 
kind of future I had ahead of me --- and still this thing of the future is a work 
in progress. Over time, I understood that no one except yourself decides for your 
own future. You take credit for both your successes and your mistakes, many times 
on your own, and some other times in collaboration with others.

It is now the time to say\footnote{The author would like to warn the reader that 
this section of this work may intentionally be ironic, humorous or sarcastic at 
times unlike the abuse of \emph{em dash} which is obviously intentional :). These 
statements are not representative of the remaining parts of this work though.} thank 
you. A thank you that no matter what I write will feel short and sometimes vague 
because I do not think that it is possible to describe it in words. Most likely 
Miguel de Cervantes Saavedra --- that prolific Spanish writer and father of the 
first modern novel known as \emph{Don Quijote de la Mancha} --- would have been 
one of the very few writers in the world who would actually be able to choose the 
words to the emotions and feelings that these acknowledgements aim to convey.

% Sonja Buchegger
And here it comes... first, and foremost, I thank my advisor, Sonja, whom I have 
shared uncountable hours talking and discussing our daily work, and also quite some 
other amount arguing. Her pragmatism along with her humor contributed to a constructive 
environment during our individual and group meetings. Her simplicity and punctilious 
comments would always make sure that I would stay focused and would push myself 
into achieving the next milestone. 

I have to admit that my Southern \emph{Europeanness} --- whom I am and will always 
be very proud of --- had to cope with her Central \emph{Europeanness} and that I 
did harshly fight --- subconsciously --- for a very long while. Despite my complaints 
--- not to her though :) --- I understood what she was patiently teaching me: the 
small things that make you a genuine researcher --- and considering my stubbornness 
that patience of hers is something to actually be proud of.

% Johan
I also have to thank Johan, who has served as my co-advisor, for his time and willingness 
to help at any time. I am very grateful to Johan who has been available when it 
was most needed and without hesitation offered his time and advice. 

% Dilian
Although not an advisor nor co-advisor, more like a counselor, Dilian has been crucial 
when his advice was most needed. He has always been there listening to my pleas 
and giving constructive criticism and advice so that I would made an informed decision. 
I am truly thankful for his judgement and time.

% Co-authors
Thanks to my co-authors, and work colleagues, --- in alphabetical order for the 
sake of some meaningful order --- Benny, Daniel, Grzegorz, Gunnar, Oleksandr, Sonja 
and Tomas. Gunnar was our second in command for a while with a humor rather unexpected 
for a Swede --- and appreciated. Grzegorz and Tomas were short term with us in different 
projects and it was a pleasure getting things done with them. Oleksandr has been 
a great support for me and I owe to him quite some venting in those moments of stress. 
Benny has been the private whom I have shared most of the time in this battle toward 
the Ph.D. and a pleasure to have worked with him. Daniel, the latest incorporation 
to the group, has brought quite some more joy to our discussions and a reminder 
of how we should aim at that ideal of ``perfect'' privacy --- and a deep love for 
Makefiles.

% Mat Näslund, Javier David, Jesús and Tobjorn
I very much owe to Mats N\"{a}slund--- who has reviewed this thesis --- for his 
comments, which have contributed to improve the final version of this text. In particular, 
the \crefname{subsection:thesis:utopia-of-privacy} \cref{subsection:thesis:utopia-of-privacy} 
stemmed from his remarks on the ideal of perfect privacy prompting the aforementioned 
\crefname{subsection:thesis:utopia-of-privacy}. Javier, my former Spanish studies 
comrade and, Jes\'{u}s and Torbj\"{o}rn, colleagues at the department, have also 
contributed to improving this text with their comments of the former and the revision 
of the abstracts in Spanish and Swedish of the latter two --- I may have some excuses 
for Swedish... but I would never imagined it would be so difficult to write in my 
own mother tongue. 

% Office mates
These years I have shared different offices --- not one... nor two... but three! 
--- with quite some people. The first few years, before we were moved one floor 
up, I spent some quality time with Torbj\"{o}rn, Mussard and Benny, each one adding 
that little bit of spice to the lively environment we had. Then, after the move, 
I joined Siavash in a tiny office that regardless of the minimal space we manage 
to make it feel spacious enough. Siavash and me had great conversations about everything, 
though when he finished his degree... I was kind of forced to move as well to the 
room next door. At first, disliking the idea of getting more space but later on... 
enjoying the German environment with Benny and Oliver.

In such German environment I have learnt quite a few things about life, a big pile 
about sustainable eating and living, and another big bunch on different topics of 
life --- seasoned with jokes about the German way of doing things...

% Department colleagues
At the department there have been many other people. Some passing by shortly, and 
some others still enjoying --- or suffering, depends on the interpretation :). Among 
those, all the other Ph.D students whom I have spent time with during lunches, lectures 
and other activities --- for example, Hamed managed to increase my average of daily 
steps because paying him a visit required quite some walk across the building, and 
really worth it! He has also been a great and needed support during the last couple 
of years.

% Grandma
I also owe my strength to continue on this path to that woman who never gave up 
and fought till the last breath. Her strength and smile inspired me on this path.
Even if I could not see her as often as I would like because of the distance, she 
would never forget to ask for me. Unfortunately, she did not get the chance to see 
with her eyes these lines finished. Wherever you are, I miss you grandma.

% Parents
Parents. Parents are... more than just a mother and a father --- cliché?. Both, 
mom and dad, have always been my biggest inspiration and support on this journey, 
Not only because they are both \emph{doctors} but also, and much more importantly, 
because they have made sure that all means were at my disposal to continue doing 
what I wanted. They may not fully understand the topics of this research although 
they have put their trust on my judgement because in the end... they just wanted 
the best for me. They have always been supporting me in both my happy times and 
taken my anger and despair during the not so happy times.

% Brother
Moreover, my brother has also been as supportive as my parents. He is a traveler 
himself, and knows what it is thing of being away from your origins. His visits 
have always been appreciated and not precisely for the goods that came along. And 
even the latest one was with some `extra' luggage, for Gabriel, my nephew, a little 
runner that hopefully one day will read these lines.

% Spanish friends
A shoutout goes to my friends in Spain, the `cuadrilla', whom I have visited on 
an off these years --- shortly though, the agenda would not permit longer. They 
have also been incredibly supportive this time and they deserve these lines of gratitude 
as well.

% Joshua
Last, but not least, Joshua is that guy who has understood my desperation --- or happiness... 
again, interpretations :) --- and whom I have also shared the joyful moments with 
the past few years. Somehow it feels like apologizing because he has been in the 
shadow many times sacrificing his own joy for the sake of this research. I can not 
be grateful enough for his patience and humor all this time --- rather challenging 
tasks considering my sometimes stubborn pragmatism --- and ever present smile to 
cheer me up.

% Barney
Another acknowledgement goes to our little Barney --- a friendly, long and black 
smoke feline \emph{sausage}; I did not imagine cats could compete in length with 
the dachshund dog breed :) --- whom \emph{miaus} in between has been troubled with 
my sleepless writing nights. He has made sure I would take frequent and proper breaks 
as he always takes his time to explore the desktop --- like now...

Finally --- and a warning to the random reader, these acknowledgements are, by no 
means, complete. They are a work in progress and they will be updated in the next 
iteration.

See you soon!

\vspace{1.5em}

% Sign off
\hfill\begin{minipage}[t]{0.33\textwidth}
        \begin{center}
            Solna, 30th April, 2017
            
            \vspace{0.75em}
            
            /Guille
        \end{center}
\end{minipage}

\section*{Funding}
The research leading to this thesis has been supported by the following projects:
\begin{itemize}
    \item Protection of personal information in social networks (Skydd av personlig 
    information f{\"o}r sociala n{\"a}tverk)\\
    Funded by the \ac*{ssf} grant: SSF FFL09-0086.
    % \url{http://stratresearch.se/en/research/ongoing-research/framtidens-forskningsledare-4/project/4048/}
    \item Privacy-preserving social and community networks\\
    Funded by the \ac*{vr} grant: VR 2009-3793.
    % \url{http://vrproj.vr.se/detail.asp?arendeid=69587}
\end{itemize}

We are grateful to the Swedish taxpayers for their commitment to research and development.
