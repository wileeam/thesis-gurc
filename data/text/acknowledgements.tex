% -*- mode: TeX -*-
% -*- coding: utf-8 -*-

At the beginning of October in 2011 a Spaniard from Valladolid landed once more 
at the busiest Swedish airport --- Stockholm Arlanda Airport --- for a new adventure. 
Such adventure would take him South to the capital, Stockholm, instead of the usual 
journey North to the always beloved Uppsala of previous years.

That guy --- the author of this work --- came to Sweden with the idea of doing this 
thing called Ph.D. --- \emph{Philosophiae Doctor}. Back then, I did not know what 
kind of future I had ahead of me --- and still this thing of the future is a work 
in progress. Over time I understood that no one except yourself decides for your 
own future. You take credit for both your successes and your mistakes, many times 
on your own, and some other times in collaboration with others.

Now it is the time to say\footnote{The author would like to warn the reader that 
this section of this work may intentionally be ironic, humorous or sarcastic at 
times unlike the abuse of \emph{em dash} which is obviously intentional :). These 
statements are not representative of the remaining parts of this work though.} thank 
you. A thank you that no matter what I write will feel short and sometimes vague 
because I do not think that it is possible to describe it in words. Most likely 
Miguel de Cervantes Saavedra --- that prolific Spanish writer and father of the 
first modern novel known as \emph{Don Quijote de la Mancha} --- would have been 
one of the very few writers in the world who would actually be able to choose the 
words to the emotions and feelings that these acknowledgements aim to convey.

And here it comes... first, and foremost, I thank my advisor, Sonja, whom I have 
shared uncountable hours talking and discussing our daily work, and also quite some 
other amount arguing. Her pragmatism along with her humor contributed to a constructive 
environment during our individual and group meetings. Her simplicity and punctilious 
comments would always make sure that I would stay focused and would push myself 
into achieving the next step.

I have to admit that my Southern Europeanness --- whom I am and will always be very 
proud of --- had to cope with her Northern Europeanness which I did harshly fight --- 
subconsciously --- for a very long while. Despite my complaints --- not to her though 
:) --- I understood those little details that makes you a genuine researcher while 
she was patiently teaching me --- and considering my stubbornness that patience of hers 
is something to actually be proud of.

I also have to thank Johan, who has served as my co-advisor, for his time and willingness 
to help any time. Johan has been available when it was most needed and without hesitation 
offered his time and advice. 

Thanks to my co-authors, and work colleagues, --- in alphabetical order for the 
sake of some meaningful order --- Benny, Daniel, Grzegorz, Gunnar, Oleksandr, Sonja 
and Tomas. Gunnar was our second in command for a while with a humor rather unexpected 
for a Swede --- and appreciated. Grzegorz and Tomas were short term with us in different 
projects and it was a pleasure getting things done. Oleksandr has been a great support 
for me and I owe to him quite some venting in those moments of stress. 


These years I have shared different offices --- not one... nor two... but three! 
--- with quite some people. The first few years before we were moved one floor up 
I spent some quality time with Torbjörn, Mussard and Benny, each one putting that 
little bit of spice to the lively environment we had. Then, after the move, I joined 
Siavash in that tiny office that regardless of the minimal space we manage to make 
it feel spacious enough. Siavash and me had great conversations about everything.
Though when he finished his degree... I was kind of forced to move as well to the 
room next door, at first, disliking the idea of getting more space but later on... 
enjoying the German environment with Benny and Oliver.

Interestingly enough, each one managed to Benny and Oliver are quite opposite when it comes to personality 

With Oliver I have had 
so many conversations --- and also jokes about the German way of doing things... --- 
that I can't count.

At the department there have been many other people. Some passing by shortly, and 
some others still enjoying --- or suffering, depends on the interpretation :). Among 
those, all the other Ph.D students 

I also owe my strength to continue on this path to that woman who never gave up 
and fought till the last breath. Her strength and smile inspired me on this path.
Even if I could not see her as often as I would like she would never forget to ask 
for me. Unfortunately, she did not get the chance to see with her eyes these lines 
finished. Wherever you are, I miss you grandma.

Parents are... parents. Both, mom and dad, have always been my inspiration for this 
because both are doctors, although in a different way... 

I admit I am blank on what to say but very much of this 
moment I owe to them. They have always ensured I can pursue what I wanted while 
at the same time they have taken my anger and desperation in this adventure. Both 
of them 

Well, my parents they always say they do not understand 
anything of the things I do and they put their trust on my judgement. 

Last, but not least, Joshua is that guy who has taken my desperation --- or happiness... 
again, interpretations :) --- and the joyful moments as well the past few years. 
Somehow it feels like apologising because he has taken most of my anger quite happily 

Another acknowledgement goes to our little cat, Barney. A long black smoke emph{sausage} --- 
I did not imagine cats could compete in length with the dachshund dog breed :) --- 
whom miaus in between has suffered some of my sleepless writing nights and he always 
made sure I would take a break while he would take his time to explore the desktop --- 
that he explored one hour ago...

I also have to thank the 'Cuadrilla' whom in the distance have been on

Finally --- and a warning to the random reader, these acknowledgements are, by no 
means, complete. They are a work in progress and they will be updated --- or enhanced ---
in the next iteration. I believe the name is called actually doctoral thesis. See 
you soon.

"If something is important enough, even if the odds are against you, you should still do it."
"Persistence is very important. You should not give up unless you are forced to give up."

Some time in October 2011 I landed in Stockholm with the idea of getting on this 
voyage of the Ph.D, and today is one of those that you look back, in retrospective.

Thank Sonja and Johan

Thank former and current colleagues/co-authors also Tomas and Polsih guy

Thank other people (Dilian, internal reviewer, opponent, examiner)

Thank former and current office mates

Thank people at department (students and faculty)

Thank parents, brother and Gabriel

Thank Joshua

Thank friends in Spain

Thank Barney


\section*{Funding}
The research leading to this thesis has been supported by the following projects:
\begin{itemize}
    \item Protection of personal information in social networks (Skydd av personlig 
    information f{\"o}r sociala n{\"a}tverk)\\
    Funded by the \ac*{ssf} grant: SSF FFL09-0086.
    % \url{http://stratresearch.se/en/research/ongoing-research/framtidens-forskningsledare-4/project/4048/}
    \item Privacy-preserving social and community networks\\
    Funded by the \ac*{vr} grant: VR 2009-3793.
    % \url{http://vrproj.vr.se/detail.asp?arendeid=69587}
\end{itemize}

We are grateful to the Swedish taxpayers for their commitment to research and development.
