% -*- mode: TeX -*-
% -*- coding: utf-8 -*-

\requote{If something is important enough, even if the odds are against you, you 
should still do it}{Elon Musk}
% \requote{Persistence is very important. You should not give up unless you are forced
% to give up}{Elon Musk}

\lettrine{\textcolor[gray]{.25}{A}}{t} the beginning of October in 2011 one Spaniard 
from Valladolid landed once more at the busiest Swedish airport --- Stockholm Arlanda 
Airport --- for a new adventure. Such adventure would take him this time South to 
the capital, Stockholm, instead of the usual journey North to the always beloved 
Uppsala of previous years.

That guy --- the author of this work --- came to Sweden with the idea of doing this 
thing called Ph.D. --- \emph{Philosophiae Doctor}. Back then, I did not know what 
kind of future was ahead of me --- and still this thing of future is work in progress. 
Over time, I understood that no one except yourself decides for your own future. 
You take credit for both your successes and your mistakes, many times on your own, 
and some other times in collaboration with others.

It is now the time to say\footnote{The author would like to warn the reader that 
this section of this work may intentionally be ironic, humorous or sarcastic at 
times unlike the abuse of \emph{em dash} which is obviously intentional :). These 
statements are not representative of the remaining parts of this work though.} thank 
you. A thank you that no matter what I write will feel short and sometimes vague 
because I do not think that it is possible to describe it in words. Most likely, 
Miguel de Cervantes Saavedra --- that prolific Spanish writer and father of the 
first modern novel known as \emph{Don Quijote de la Mancha} --- would have been 
one of the very few writers in the world who would actually be able to choose the 
words to the emotions and feelings that these acknowledgements aim to convey.

% Sonja
And here it comes... first, and foremost, I thank my advisor, Sonja, whom I have 
shared uncountable hours talking and discussing our daily work, and also quite some 
other amount arguing. Her pragmatism along with her humor contributed to a constructive 
environment during our individual and group meetings. Her simplicity and punctilious 
comments would always make sure that I would stay focused and would push myself 
into achieving the next milestone. 

I have to admit that my Southern \emph{Europeanness} --- whom I am and will always 
be very proud of --- had to cope with her Central \emph{Europeanness}, that I did 
harshly fight --- subconsciously --- for a very long while. Despite my complaints 
--- sometimes not to her though :) --- I understood what she was patiently teaching 
me: the small things that make you a genuine researcher --- and considering my stubbornness 
that patience of hers is something to actually be proud of.

% Johan
I also have to thank Johan, who has served as my co-advisor, for his time and willingness 
to help at any time. I am very grateful to him because he has been available when 
it was most needed and he has never hesitated to offer his time and advice. 

% Dilian
Although not an advisor nor co-advisor, more like a counselor, Dilian has been crucial 
when his experienced advice was most needed. He has always been there listening 
to my pleas and giving constructive criticism and advice so that I would make an 
informed decision. I am truly thankful for his judgement and time.

% Co-authors
Thanks to my co-authors, and work colleagues, --- in alphabetical order for the 
sake of some meaningful order --- Benny, Daniel, Grzegorz, Gunnar, Oleksandr, Sonja 
and Tomas. Gunnar was our second in command for a while with a humor rather unexpected 
for a Swede --- and appreciated. Grzegorz and Tomas were short term with us in different 
projects and it was a pleasure getting things done with them. Oleksandr has been 
a great support for me and I owe to him quite some venting in those moments of stress. 
Benny has been the private whom I have shared most of the time with in this battle toward 
the Ph.D. and a pleasure to have worked with. Daniel, the latest incorporation 
to the group, has brought a new kind of joy to our discussions and a reminder of 
how we should aim at that ideal of ``perfect'' privacy --- and a deep love for 
\emph{Makefile}.

% Mat Näslund, Javier David, Jesús and Torbjörn
I very much owe to Mats N\"{a}slund--- who has thoroughly reviewed this thesis --- 
for his comments, which have contributed to improve the final version of this text. 
In particular, the \namecref{subsection:thesis:utopia-of-privacy} ``\nameref{subsection:thesis:utopia-of-privacy}'' 
stemmed after his remarks on the ideal of ``perfect'' privacy. Javier, a former 
Spanish studies' comrade, Jes\'{u}s, a colleague at the Theoretical Computer Science 
department and, Carl-Erik, a wise and admired friend, have also contributed 
to improving this text with the overall comments of the former, and the revision 
of the abstract in Spanish and Swedish of the latter two respectively --- I may 
have some excuses for Swedish... but I would have never imagined it would be so 
difficult to write in my own mother tongue. 

% Office mates
These years I have shared different offices --- not one... nor two... but three! 
--- with quite some people. The first few years, before we had to move one floor 
up, I spent some quality time with Torbj\"{o}rn, Mussard and Benny, each one adding 
that little bit of spice to the lively environment we had. Then, after the move, 
I joined Siavash in a tiny office that regardless of the minimal space we manage 
to make it feel spacious enough. Siavash and me had great conversations about everything 
and I am thankful for the different points of view we discussed. Though when he 
finished his degree, I was kind of forced to move again, this time to the room next 
door. At first, disliking the idea of getting about three times more space but later 
on... enjoying the immersion in the German habitat with Benny and Oliver.

In such overseas German ``state'' I have learnt quite a few things about life, a 
big pile about sustainable eating and living, and another big bunch on random topics 
in the broad area of computer and information science --- sometimes seasoned with 
jokes about the German way of doing things.

% Department colleagues
At the department there have been many other people. Some passing by shortly, and 
some others still enjoying --- or suffering, depends on the interpretation :). Among 
those, all the other Ph.D students whom I have spent time with during lunches, lectures 
and other activities --- for example, Hamed managed to increase my average of daily 
steps because paying him a visit required quite some walk across the building, and 
really worth it! He has also been a great and needed support during the last couple 
of years.

% Grandma
I also owe my strength to continue on this path to that woman who never gave up 
and fought till the last breath. Her strength and smile inspired me on this path.
Even if I could not see her as often as I would like to because of the distance, she 
would never forget to ask for me. Unfortunately, she did not get the chance to see 
with her eyes these lines finished. Wherever you are, I miss you grandma.

% Parents
Parents. Parents are... more than just a mother and a father --- cliché?. Both, 
mom and dad, have always been my biggest inspiration and support in this journey. 
Not only because they are both \emph{doctors} but also, and much more importantly, 
because they have made sure that all means were at my disposal to continue doing 
what I always wanted. They may not fully understand the topics of this research instead, 
they have put their trust on my judgement because in the end, and as they say, ``you 
know better and we just want the best for you'''. They have always supported me 
in both my happy times and patiently taken my anger and despair during the not so 
happy times.

% Brother
Moreover, my brother has also been as supportive as my parents. He is a traveler 
himself and knows what it is ``thing'' of being away from your origins. His visits 
have always been appreciated and not precisely for the goods that came along. And 
even the latest visit came with some `extra' luggage, for Gabriel, my nephew, a 
little runner that hopefully one day will read these lines.

% Spanish friends and João
A shoutout goes to my friends in Spain, the `Cuadrilla', whom I have met on an off 
these years during my trips to the motherland --- shortly though, the agenda would 
not permit that much longer. They have also been incredibly supportive this time 
and they deserve these lines of gratitude as well. Also, my long-time Portuguese 
friend, Jo\~{a}o, deserves such recognition because he has been the punching bag 
of quite some of the stressful moments of this journey.

% Joshua
Last, but not least, Joshua is that guy who has thoroughly understood my desperation --- or happiness... 
again, interpretations :) --- and whom I have also shared with the joyful moments 
during the past few years. Somehow it feels like apologizing because he has been 
in the shadow many times sacrificing his own joy for the sake of science. I can 
not be grateful enough for his patience and humor during all this time --- rather 
challenging tasks considering my stubbornness. His ever present smile has been his 
best tool to cheer me up at those times of struggle to make me stand up again and 
continue the journey.

% Barney
Another acknowledgement goes to our little Barney, a friendly, long and black 
smoke feline \emph{sausage} --- I did not imagine that cats could compete in length 
with the dachshund dog breed :) --- whom \emph{miaus} in between has been troubled 
with my sleepless writing nights of this thesis. He has made sure I would take frequent 
and proper breaks as he always takes his time to explore the desktop --- like now...

Finally --- and a warning to the random reader, these acknowledgements are, by no 
means, complete. They are work in progress and they shall be updated in the next 
iteration.

See you soon!

\vspace{1.5em}

% Sign off
\hfill\begin{minipage}[t]{0.33\textwidth}
        \begin{center}
            Solna, April 30th, 2017
            
            \vspace{0.75em}
            
            /Guille
        \end{center}
\end{minipage}

\section*{Funding}
The research leading to this thesis has been supported by the following projects:
\begin{itemize}
    \item Protection of personal information in social networks (Skydd av personlig 
    information f{\"o}r sociala n{\"a}tverk).
        \begin{itemize}
            \item Funder: \acf*{ssf}
            \item Grant: SSF FFL09-0086
            \item Details: \url{http://web.archive.org/web/20170429103824/http://stratresearch.se/en/research/ongoing-research/framtidens-forskningsledare-4/project/4048/}
        \end{itemize}
        
    \item Privacy-preserving social and community networks (Skydd av personlig information 
    f{\"o}r sociala n{\"a}tverkstj{\"a}nster p{\aa} Internet).
        \begin{itemize}
            \item Funder: \acf*{vr}
            \item Grant: VR 2009-3793
            \item Details: \url{http://web.archive.org/web/20170429104127/http://vrproj.vr.se/detail.asp?arendeid=69587}
        \end{itemize}
\end{itemize}

We are grateful to the Swedish taxpayers for their commitment to research and development.
