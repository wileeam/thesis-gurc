% -*- mode: TeX -*-
% -*- coding: utf-8 -*-

\chapter{Conclusions}
    \label{chapter:thesis:conclusions}
\requote{Space: the final frontier. These are the voyages of the starship Enterprise. Its 
continuing mission: to explore strange new worlds, to seek out new life and new 
civilizations, to boldly go where no one has gone before}{Star Trek: The Next Generation}

\lettrine{\textcolor[gray]{.25}{W}}{hen} Sir Tim Berners-Lee invented the \Ac{www} 
in 1989, he imagined a large-scale and decentralised \ac{is} where everybody would 
be able to have their own website and plug-in a server from which they could offer 
their content. Such idea never worked out in the free-form in which it was envisioned 
and, instead, personal data ended up under the control of major technological companies 
in huge data centers --- ``silos'' --- distributed all over the world.

% https://archive.org/details/DWebSummit2016_Keynote_Tim_Berners_Lee
% https://web.archive.org/web/20161106081045/https://techcrunch.com/2016/10/09/a-decentralized-web-would-give-power-back-to-the-people-online/
The idea of bringing back the initial design of a decentralised \ac{www} has been 
in the minds of researchers and practitioners for a while now --- ``bring back the 
power to the people'' as Tim Berners-Lee said during his keynote ``Re-decentralizing 
the web -- some strategic questions'' at the ``Decentralized Web Summit'' in 2016. 

Decentralisation of \acp{is} on the \Internet --- or re-decentralisation --- has 
gained popularity in recent years due to increasing concerns on the risks of personal 
data misuse by major technological companies as they oversee a large portion of 
the data circulating the \Internet.

Until the turn of the 21st century the focus was on the privacy concerns of individuals 
as consumers and the policy making in that direction --- see some examples at \cite{MilbergBSK95}, 
\cite{Rindfleisch97}, \cite{Clarke99} and \cite{NamSPI06}. However, in 2013, the 
focus shifted to governmental mass surveillance after the disclosures of Edward 
Snowden --- a former contractor of the \aca{usa} government --- of the existence 
of a global surveillance apparatus led by the \aca{usa} government \Ac{nsa} in cooperation 
with other intelligence organisations and in partnership with major technological 
players such as \Google, \Facebook or \Apple. 

Such idea of decentralisation is what motivates this thesis as means to achieve 
privacy as data control in similar terms as those stated by Allen \cite{Allen99}. 
Though overruling the power of a central authority like the service provider of 
an \ac{osn} entails sharing the duties and responsibilities among a subset of federated 
servers or among all the users of the network in a \ac{p2p} fashion. In other words, 
there are many non-trivial trade-offs to consider which are worth the effort of 
being looked at.

With this thesis we aim at contributing to the field of decentralised \acp{is} by 
presenting solutions showing that decentralisation is possible and improves the 
control that a user has over her personal data --- her privacy is enhanced. We specify, 
analyse and design some protocols showing how some common functionality in generic 
\acp{is} can be mimicked in a decentralised scenario, such as a \ac{p2p} network, 
or in a more specific one, such as a \ac{dosn}. We also demonstrate how in some 
conventional processes where privacy is not seen as that important, it is possible 
to enhance the privacy of a user while keeping the functionality intact in a scenario 
where a trusted central authority is required.

For decentralised \acp{is} we present in \cref{article:thesis:passwords-peer-to-peer} 
a scheme for usable privacy-preserving user authentication in a similar manner than 
what most \acp{is} use nowadays, by means of user-password credentials. We also 
design functionality that allows credentials recovery or even maintain authentication 
details stored in a secure manner across devices with a negligible impact on performance.

In a specific use case of decentralised \acp{is} within the realm of \acp{osn}, 
\acp{dosn}, we demonstrate how it is possible to have a coordination and cooperation 
mechanism such as the organisation of events without the need of a \acl{ttp} in \cref{article:thesis:events-invitations-dosns}. 
Along with standard cryptographic primitives we describe some privacy-enhancing 
tools that allow storage location indirection --- control of the ability to access 
an encrypted object and possibly to decrypt it as well, or controlled cipher-text 
inferences --- allowing certain leak information such as the size of an encrypted 
object but not its content --- and define a ``commit-disclose'' protocol --- where 
some information is disclosed to a certain set of users who have fulfilled some 
requirement.

We also tackled the problem of anonymity in centralised \ac{is} with a prototype 
implementation of a document submission and grading system in \cref{article:thesis:document-submission-system} 
without compromising the correctness of the process. By means of standard cryptographic 
primitives such as blind signatures we propose a protocol that guarantees ``forward 
unlinkability'' of a document with the author over time --- an examiner will not 
be able to link the author of a document with the document itself even after reporting 
the grade for that document --- and, ``provable linkability'' of an author's identity 
with the grade for its document --- an author of a document is guaranteed to receive 
a grade if and only the author submitted the document.

Finally, we also aim at bringing back --- once more --- the debate on privacy because 
the envisioned ``Big Brother'' society that George Orwell pictured is not far from 
being part of our daily lives \cite{Orwell49}. The solutions to some of the security 
and privacy problems we have presented in the previous \namecrefs{chapter:thesis:introduction} 
are some examples of how we can actually achieve some compromises to resolve the 
conflict between the need and value for information and the cost of personal privacy.

\section{Future work}
    \label{section:thesis:future-work}
There are many directions in which centralised and decentralised systems can be 
developed further in a privacy-preserving manner. From the purely decentralised 
approach such as \ac{p2p} networks to mixed models where some of the services are 
supplied --- partially or fully ---- by means of centralised service providers. 
For example, there may be a need to trust a third party such as a governmental agency 
--- the police --- to provide users' identification capabilities to some decentralised
services that would need to know that each user has a verifiable and unique identity 
for the purpose of complying with governmental regulations --- such as tax laws.

In the following list we describe some of the possible lines of work we have thought 
this thesis can be followed up with,
\begin{itemize}[topsep=\parskip, parsep=\parskip, itemsep=\parskip]
    
    \item stricter formalisation and evaluation of the security and privacy properties.
    
    In our work we have evaluated the security and privacy properties of the protocols 
    we have proposed. However, we have done so in a loosely manner --- pseudo-formally 
    --- and, up to a certain degree, in isolated scenarios.
    
    Such features would benefit from a strict formalisation --- for example, by 
    means of formal systems such as \emph{join}-calculus \cite{FournetG96}, \(\lambda\)-calculus 
    \cite{Church36} or \(\pi\)-calculus \cite{MilnerPW92} --- and a deeper evaluation 
    for increased soundness.
    
    Moreover, combining the features in the same system may reveal inconsistencies 
    and conflicts between the properties or even uncover new properties. Assessing 
    the compatibility would be another important task to include when formalising 
    the properties.
    
    \item wider range of functionalities.
    
    We have shown some cases of functionalities in centralised and decentralised 
    \acp{is} with the ideal goal of achieving a fully decentralised \ac{is} using 
    \acp{osn} as our main scenario. Though we have barely scratched the surface 
    of the area because there are many open challenges to address in a more privacy-preserving 
    manner.
    
    For example, it is not trivial to retrieve the content of the connections of 
    a user in such a decentralised scenario and it gets more complicated when considering 
    the question of posting content on someone else's profile or simply commenting. 
    
    Another open problem is the realisation of the so called albums (of pictures) 
    because it is not only a matter of distributed storage of such information but 
    mostly search and access control of this content --- and other questions such 
    as allowing the identification of individuals in the stored content.
    
    \item expansion of the decentralisation approach to other areas.
    
    Although our focus in this work has been on decentralisation of centralised 
    \acp{is}, such as those where there can be various actors --- for example, in 
    \acp{osn}, there are also other scenarios where decentralisation could be of 
    interest
    
    For example, in home care\footnote{Medical care and safety are some controversial 
    matters when it comes to ideal privacy. Consider an unconscious person in a 
    life threatening situation, should her medical history be accessed without her 
    consent? It would be technologically possible to protect such sensitive personal 
    data but it simply does not make sense. However, it is clear it would be a privacy 
    violation allowed in many legislations because the preservation of the human 
    life is an unquestioned obligation.} --- where a nurse carries some sensitive 
    \acl{pii} in order to provide the right care to a patient and eventually returns 
    to the primary care facility with some new personal data about that patient, 
    in content-sharing or even in public ledging of transactions without a \acl{ttp} 
    --- such as those happening in the blockchain powering the crypto-currency \Bitcoin.
    
    \item privacy-preserving analytics.
    
    There are some solutions to do simple privacy-preserving centralised analytics, 
    but there are no straightforward ways or solutions to do complex ones. Moreover, 
    doing such an analysis in a decentralised scenario, where it is not easy to have 
    a complete picture of the network at the time of analysis, makes it a challenge 
    on its own.
    
    However, in the coming years, privacy-preserving analytics not only will be 
    key for the success of many businesses and enterprises but also a requirement 
    \cite{LuZLLS14}. Bad publicity and wrong-doing of past actions --- such as the 
    ones after the Snowden revelations --- are among the reasons for which corporations 
    may want to minimize the data footprint at its source --- the user --- and 
    process it in a more privacy-preserving manner. 
    
\end{itemize}

Ultimately, information-theoretically secure solutions are the only ones that can 
guarantee the effectiveness of privacy in whichever system they are implemented because, 
by design, these solutions assume an adversary with unbounded computing power. 
For example, the confidentiality of personal data encrypted with encryption algorithms 
that are information-theoretically secure is guaranteed to be resilient to any future 
development in computing power such as quantum computing. 

Some of the information-theoretically secure solutions that exist today and that 
could be of use to the scenarios we have described in this thesis are either too 
complex to implement or somewhat computationally inefficient in a distributed scenario. 
However, any further work must aim at such an ideal of information-theoretic secure, 
usable and performing solutions in a decentralised scenario, and so do we.
