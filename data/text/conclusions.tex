% -*- mode: TeX -*-
% -*- coding: utf-8 -*-

\chapter{Conclusions}
    \label{chapter:thesis:conclusions}
\requote{Space: the final frontier. These are the voyages of the starship Enterprise. Its 
continuing mission: to explore strange new worlds, to seek out new life and new 
civilizations, to boldly go where no one has gone before}{Star Trek: The Next Generation}

\lettrine{\textcolor[gray]{.25}{W}}{hen} Tim Berners-Lee invented the \Ac{www} in 
1989, he envisioned a large-scale and decentralised \ac{is} where everybody would 
be able to have their own website and plug-in a server from which they could offer 
their content. Such idea never worked out and, instead, personal data ended up under 
the control of major technological companies in huge data centers --- ``silos'' 
--- distributed all over the world.

% https://archive.org/details/DWebSummit2016_Keynote_Tim_Berners_Lee
% https://web.archive.org/web/20161106081045/https://techcrunch.com/2016/10/09/a-decentralized-web-would-give-power-back-to-the-people-online/
The idea of bringing back the initial design of a decentralised \ac{www} has been 
in the minds of researchers and practitioners for a while now --- ``bring back the 
power to the people'' as Tim Berners-Lee said during his keynote ``Re-decentralizing 
the web -- some strategic questions'' at the ``Decentralized Web Summit'' in 2016. 

Decentralisation of \acp{is} on the \Internet --- or re-decentralisation --- has 
gained popularity in recent years due to increasing concerns on the risks of personal 
data misuse by major technological companies as they oversee a large portion of 
the data circulating the \Internet.

Until the turn of the 21st century the focus was on the privacy concerns of individuals 
as consumers and the policy making in that direction --- see some examples at \cite{MilbergBSK95}, 
\cite{Rindfleisch97}, \cite{Clarke99} and \cite{NamSPI06}. However, in 2013, the 
focus shifted to governmental mass surveillance after the disclosures of Edward 
Snowden --- a former contractor of the \aca{usa} government --- on the existence 
of a global surveillance apparatus led by the \aca{usa} government \Ac{nsa} in cooperation 
with other intelligence organisations and in partnership with major technological 
players such as \Google, \Facebook or \Apple. 

Such idea of decentralisation is what motivates this thesis as means to achive 
privacy as data control in similar terms as those stated by Allen \cite{Allen99}. 
Though overruling the power of a central authority like the service provider of 
an \ac{osn} entails sharing the duties and responsibilities among a subset of federated 
servers or among all the users of the network in a \ac{p2p} fashion. In other words, 
there are many non-trivial trade-offs to consider which are worth the effort of 
being looked at.

With this thesis we aim at contributing to the field of centralised \acp{is} by 
presenting solutions showing that decentralisation is not only possible but also 
improves the control that a user has over her personal data --- her privacy is enhanced. 
We specify and design some protocols showing how some common functionality in 
generic \acp{is} can be mimicked in a decentralised scenario, such as a \ac{p2p} 
network, or in a more specific one, such as a \ac{dosn}. We also demonstrate how 
in some conventional process where privacy is not seen that important it is possible 
to enhance the privacy of a user while keeping the functionality intact in a scenario 
where a trusted central authority is required.

For decentralised \acp{is} we present in \cref{article:thesis:passwords-peer-to-peer} an 
scheme for usable privacy-preserving user authentication in a similar manner --- 
via user-password credentials --- than what most \acp{is} use nowadays with key 
functionality to recover the credentials or even maintain the authentication details 
stored in a secure manner across devices with a negligible impact on performance.

In a specific use case of decentralised \acp{is} within the realm of \acp{osn}, 
\acp{dosn}, we demonstrate how it is possible to have a coordination and cooperation 
mechanism such as the organisation of events without the need of a \ac{ttp} in \cref{article:thesis:events-invitations-dosns}. 
Along with standard cryptographic primitives we describe some privacy-enhancing 
tools that allow storage location indirection --- control of the ability to access 
an encrypted object and possibly to decrypt it as well, or controlled cipher-text 
inferences --- allowing certain leak information such as the size of an encrypted 
object but not its content --- and define a ``commit-disclose'' protocol --- where 
some information is disclosed to a certain set of users who have fulfilled some 
requirement.

We also tackled the problem of anonymity in centralised \ac{is} with a prototype 
implementation of a document submission and grading system in \cref{article:thesis:document-submission-system} 
without compromising the correctness of the process. By means of standard cryptographic 
primitives such as blind signatures we propose a protocol that guarantees ``forward 
unlinkability'' of a document with the author over time --- an examiner will not 
be able to link the author of a document with the document itself even after reporting 
the grade for that document --- and, ``provable linkability'' of an author's identity 
with the grade for its document --- an author of a document is guaranteed to receive 
a grade if and only the author submitted the document.

% TODO: Replace chapter with macro command
Finally, we also aim at bringing back --- once more --- the debate on privacy because 
the envisioned ``Big Brother'' society that George Orwell pictured is not far from 
being part of our daily lives \cite{Orwell49}. The solutions to some of the security 
and privacy problems we have presented in the previous \namecrefs{chapter:thesis:introduction} 
are some examples on how we can actually achieve some compromises to resolve the 
conflict between the need and value for information and, the cost of personal privacy.

\section{Future directions}
    \label{section:future-directions}
There are many directions in which \acp{cbis} can be developed further in a privacy-preserving 
manner. From the purely decentralised approach such as \ac{p2p} networks to mixed 
models where some of the services are supplied --- partially or fully ---- by means 
of centralised service providers. For example, a trusted third party such as a governmental 
agency --- the police --- could provide a national \ac{pki} on which some decentralised
services that would need a real identity could rely on to uniquely identify their 
users. We list some of the possible lines of work that this thesis could be followed 
up with,
\begin{itemize}
    
    \item stricter formalisation and evaluation of security and privacy properties,\\
    we have looked at these loosely in isolated scenarios (the functionality itself)
    but these features would benefit from a deeper evaluation that if formalised (pi-calculus) 
    would be able to guarantee even more.
    At the same time, a combination of the security and privacy properties of the 
    different features that we have looked at may conflict with each other, hence 
    assessing the compatibility would also be needed (besides, they will come together 
    anyways)
    
    \item wider range of functionalities,\\
    TODO
    
    \item expand decentralisation approach to other areas,\\
    our focus has always been on centralised systems in which there are multiple actors
    for example, in osns. There are other scenarios where decentralisation could 
    be interesting, for example, in home care (a nurse transports some personal sensitive data
    to help a patient and eventually returns with some new data about that patient)
    
    \item privacy-preserving analytics,\\
    although there are some solutions to do some analysis in a privacy-preserving 
    manner in centralised, there are not simple ways or solutions to do complex ones
    morevoer, moving such analysis in a decentralised scenario, where there it is 
    not easy to have a complete picture of the network at the time of analysis, makes
    it a challenge on its own.
    However, in the coming years, privacy-preserving analytics are going to be key 
    for the success of many business and enterprises as these do not want to risk 
    bad publicity such as the one gained after Snowden, and they may want to minimize 
    the data footprint already in the source of it (the user)
    
    \item more,\\
    TODO

\end{itemize}

