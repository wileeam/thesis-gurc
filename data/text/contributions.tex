% -*- mode: TeX -*-
% -*- coding: utf-8 -*-

\requote{Science is always worth it because its discoveries, sooner or later, are 
always put to use}{Severo Ochoa de Albornoz}

% P2P
\section{\usebibentry{KreitzBGRB12}{title}}
\begingroup\centering
\begin{ppBox}
    \bibentry{KreitzBGRB12}
\end{ppBox}
\endgroup

\subsection{Summary}
% TODO Summarise or insert abstract
% -*- mode: TeX -*-
% -*- coding: utf-8 -*-

One of the differences between typical peer-to-peer (P2P) and client-server
systems is the existence of user accounts. While many P2P applications, like
public file sharing, are anonymous, more complex services such as
decentralized online social networks require user authentication. In these,
the common approach to P2P authentication builds on the possession of
cryptographic keys. A drawback with that approach is usability when users
access the system from multiple devices, an increasingly common scenario.

In this work, we present a scheme to support logins based on users knowing a
username-password pair. We use passwords, as they are the most common
authentication mechanism in services on the Internet today, ensuring strong
user familiarity. In addition to password logins, we also present supporting
protocols to provide functionality related to password logins, such as
resetting a forgotten password via e-mail or security questions. Together,
these allow P2P systems to emulate centralized password logins. The
results of our performance evaluation  indicate that incurred delays
are well within acceptable bounds.


% Benny's thesis
% The problem of how to transfer credentials in a usable and secure way from one device to another is one of the concerns of Article B contained in this thesis. The proposed solution emulates the traditional username/password-login paradigm in a decentralized system.

\subsection{Contributions}
    \label{subsection:contributions-p2p}
I contributed actively in the discussions we held for the design of the protocols 
with ideas and improvements, for example, I helped with the formalisation of some 
of the protocols such as the password recovery mechanism. 

\subsection{Changes for the thesis}
TODO: Include comments from Mats Näslund (via B's thesis)

% EI
\section{\usebibentry{RodriguezCanoGB14}{title}}
\begingroup\centering
\begin{ppBox}
    \bibentry{RodriguezCanoGB14}
\end{ppBox}
\endgroup

\subsection{Summary}
% TODO Summarise or insert abstract
% -*- mode: TeX -*-
% -*- coding: utf-8 -*-

% 5 questions:
%
% What is the problem:
% Why is it a problem: 
% Why should we care: 
% What is our approach: 
% What are the findings: 

\Acp*{osn} have an infamous history of privacy and security issues. One approach 
to avoid the massive collection of sensitive data of all users at a central point 
is a decentralized architecture.

An event invitation feature -- allowing a user to create an event
and invite other users who then can confirm their attendance --
is part of the standard functionality of \acsp*{osn}. 
%
We formalize security and privacy properties of such a feature 
like allowing different types
of information related to the event (\eg how many people are
invited/attending, who is invited/attending) to be shared with 
different groups of users (\eg only invited/attending users).
%or only attending users to see an additional private event description.

Implementing this 
feature in a Privacy-Preserving \Acl*{dosn} is non-trivial because there is 
no fully trusted broker to guarantee fairness to all parties involved. 
%
We propose a secure decentralized protocol for implementing
this feature, 
using tools such as storage location indirection, ciphertext inferences
and a disclose-secret-if-committed mechanism, derived from standard
cryptographic primitives.

The results can be applied in the context of Privacy-Preserving \acsp*{dosn}, but 
might also be useful in other domains that need mechanisms for cooperation and coordination, 
\eg \Acl*{cwe} and the corresponding collaborative-specific tools, \ie groupware, 
or \Acl*{cscl}.


\subsection{Contributions}
I developed, with Benjamin Greschbach, the problem statement as well as the formalisations 
of the security and privacy properties, and the protocols we devised. I also contributed 
with Benjamin in the evaluation of the implementation and the tools we devised.

\subsection{Changes for the thesis}
TODO: Include comments from Mats Näslund (via B's thesis)
% Clarifications:
% % Here, I think it gets contradictory to say the services have "very bad reputation", yet they are "popular".
% - Infamous vs popularity:
%   The `bad reputation' of OSNs services is a fact that media has covered widely
%   with numerous cases of intenational and unintentaional data leakages, censorship,
%   and even collaboration with national governments to access personal data without
%   consent. See the introduction for some of those news covered by media worldwide.
%   However, such examples of privacy threats don't seem to affect that much the success
%   of such services. Every year, the numbers reported by some of these services providers
%   to their stakeholders do nothing but just increase.
%   Although, while the success of such networks cannot be questioned given that popularity,
%   the users are more aware of their own privacy and data sharing activities in these
%   networks.
%
% - Communication between participants:
%   The `abstract' exchanges of information among the participants in the protocols,
%   that is, organiser sends invitiations to invitees, are assumed to be protected
%   against techniques such as traffic analysis (by reason that if they were not protected,
%   a third party could infer who are the invitees just by analysing the outgoing
%   traffic from the organised when she sends the issued invitiations to them).
%   For example, a trarffic analysis protection such as XXX can be used in our scenario.
  


% DSS
\section{\usebibentry{GreschbachREB15}{title}}
\begingroup\centering
\begin{ppBox}
    \bibentry{GreschbachREB15}
\end{ppBox}
\endgroup

\subsection{Summary}
% TODO Summarise or insert abstract
% -*- mode: TeX -*-
% -*- coding: utf-8 -*-

% What is the problem: Deficient anonimity and unlinkability in grading 
% Why is it a problem: Subjective grading 
% Why should we care: Regain privacy?
% What is our approach: Protocols based on standard crypto tools: blind digital signatures
% What are the findings: Anonimity and unlinkability can be achieved with minor trade offs

%Privacy regulations mandate the collection of \Acl*{pii}, \eg full name or date 
%of birth, to be proportionate and necessary for the purpose of the offered service, 
%for example, registration at an online e-commerce website. However, in some instances, 
%the amount and type of such personal data does not match the actual needs nor the intended 
%usage while in other cases the availability of non-essential information may lead 
%to undesired biased assessments.

Document submission and grading systems are commonly used in educational institutions.
They facilitate the hand-in of assignments
by students, the subsequent grading by the course teachers and the 
management of the submitted documents and corresponding grades.
But they might also undermine the privacy of students, especially
when documents and related data are stored long term with the risk of
leaking to malicious parties in the future. 
%
%Discriminatory
%judgement can be another issue in these systems when teachers during grading
%know the identity of the student who authored a certain document.
%
We propose a protocol for a privacy-preserving, anonymous document
submission and grading system based on blind signatures.
Our solution guarantees the unlinkability of
a document with the authoring student even after her grade has been
reported, while the student can prove that she received the grade
assigned to the document she submitted. 
%
We implemented a prototype of the proposed protocol to show its
feasibility and evaluate its privacy and security properties.

% In such scenario, our proposed protocol guarantees what we defined as ``forward
% unlinkability'' of a document with the author --- a document and its author's identity
% must remain unlinkable even after receiving a grade --- and ``provable linkability''
% of an author with a grade --- the system must guarantee that an author receives
% a grade if and only if the author submitted a document for grading that was actually
% graded.

\subsection{Contributions}
Together with Benjamin Greschbach, I developed the problem statement, the formalisation 
of the security and privacy properties, and the analysis and evaluation of our solution. 
Tomas Ericsson implemented a proof-of-concept of the problem statement and contributed 
actively in the formalisation \cite{Ericsson15}.

%\subsection{Changes for the thesis}
