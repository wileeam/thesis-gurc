% -*- mode: TeX -*-
% -*- coding: utf-8 -*-


    % What is the problem?
    % Why is it a problem?
    % Why should we care?
    % What is our approach?
    % What are our findings?

    % Fix for bad grupin in lettrine package. See http://tex.stackexchange.com/questions/134371/lettrine-in-abstract-strange-behaviour
    % What is the problem?
    % Threats (to privacy) posed by centralised ISs to their users (and their data)
    % Why is it a problem?
    % (a) Massive collection and control of personal (and sensitive) data under one single player (or an oligpole)
    {
    \lettrine{\textcolor[gray]{.25}{P}}{rivacy enhancing technologies}\csname@restorepar\endcsname\everypar{} 
    have proven to be a beneficial area of research aiming at lessening the threats 
    to the privacy of users' personal information in centralised \aclp*{is} such 
    as \aclp*{osn}. In consequence, decentralised solutions have been proposed to 
    extend the control that users have over their data as opposed to the centralised 
    massive collection of personal and sensitive data.\par
    }
    Integritets främjande teknik har visat sig vara en positiv forskningsområde 
    som syftar till att minska hoten mot den personliga integriteten av användarnas 
    personuppgifter i centraliserade informationssystem som online sociala nätverk. 
    Följaktligen har decentraliserade lösningar föreslagits för att förlänga den 
    kontroll som användare har över sina data i motsats till den centraliserade 
    massiv samling av personliga och känsliga data.
    
    % Why should we care? (Why is it a problem?)
    % (b) Power of centralised service provider 
    % (c) Lack of control
    % (d) Threat to the right to be left alone
    The power that the service provider has in centralised \aclp*{is} has been shown 
    to diminish the user's privacy with cases of misuse, censorship or data leakage. 
    Moreover, the disclosures in 2013 of a global surveillance program led by public 
    intelligence institutions in collaboration with some of the service providers 
    of such centralised \aclp*{is} has accelerated the debate on how to take action 
    to counteract the threats to privacy. In particular, the threat to the legal 
    ``right to be let alone'' as defined by Samuel Warren and Louis Brandeis in 
    1890 in their influential law review article ``The Right to Privacy''.
    
    % What is our approach?
    % Decentralisation by means of privacy-preserving decentralised systems
    % Trade-offs: feature replication and security/privacy protection
    Privacy-preserving decentralised systems are plausible solutions to such threats 
    and one of the most common alternatives used nowadays. However, the removal 
    of the central authority comes with two main trade-offs to be tackled, mimicking 
    the features of the centralised \acl*{is} in a usable way and taking over the 
    supervision of the security and privacy threats that once were a responsibility 
    of the central authority.
    
    % What are our findings? (What is our approach?)
    % Privacy-preserving decentralised systems by means of examples developing the functionality and showing the concepts of decentralisation and security and privacy properties.
    % Brief explanation of the three solutions
    In our thesis, we propose the use of privacy-preserving decentralised systems 
    and develop three solutions to centralised \aclp*{is} in terms of decentralisation, 
    functionality and, achievable security and privacy. For decentralised \aclp*{is} 
    in general we show a concrete mechanism for user authentication via standard 
    user-password credentials with comparable usability to standard centralised 
    applications. Within the realm of practical decentralised systems we show a 
    specific example in the domain of \aclp*{dosn} implementing a coordination and 
    cooperation mechanism to organise events without the need of a trusted third 
    party. Finally, we step back to those centralised systems where the presence 
    of the central authority is still required and, instead, improve one of the 
    aspects of the user's privacy: anonymity, by showing an implementation of a 
    system to submit and grade documents anonymously in the academic sphere in a 
    generic centralised privacy-preserving system.

    % What are our findings?
    % Privacy-preserving protocols in centralised and decentralised systems that mitigate the dangers to personal privacy
    Our solutions are some concrete examples of how privacy as data control, as 
    the paradigm envisioned by Anita Allen, can be achieved to varying degrees in 
    privacy-preserving centralised and decentralised \aclp*{is}. Nonetheless, we 
    hope that the privacy-preserving protocols we propose and the evaluation of 
    the security and privacy properties can be useful in other scenarios as such 
    to mitigate the diverse dangers to personal privacy that we are facing at the 
    present times.
