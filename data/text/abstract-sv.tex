% -*- mode: TeX -*-
% -*- coding: utf-8 -*-


    % What is the problem?
    % Why is it a problem?
    % Why should we care?
    % What is our approach?
    % What are our findings?

    % Fix for bad grupin in lettrine package. See http://tex.stackexchange.com/questions/134371/lettrine-in-abstract-strange-behaviour
    % What is the problem?
    % Threats (to privacy) posed by centralised ISs to their users (and their data)
    % Why is it a problem?
    % (a) Massive collection and control of personal (and sensitive) data under one single player (or an oligpole)
    {
    % \lettrine{\textcolor[gray]{.25}{P}}{rivacy enhancing technologies}\csname@restorepar\endcsname\everypar{}
    % have proven to be a beneficial area of research aiming at lessening the threats
    % to the privacy of users' personal information in centralised \aclp*{is} such
    % as \aclp*{osn}. In consequence, decentralised solutions have been proposed to
    % extend the control that users have over their data as opposed to the centralised
    % massive collection of personal and sensitive data.\par

    \lettrine{\textcolor[gray]{.25}{I}}{ntegritets främjande teknik}\csname@restorepar\endcsname\everypar{} 
    --- på engelska, \aclp*{pet} --- har visat sig vara en positiv forskningsområde 
    som syftar till att minska hoten mot den personliga integriteten av användarnas 
    personuppgifter i centraliserade informationssystem som online sociala nätverk 
    --- på engelska, \aclp*{osn}. Följaktligen har decentraliserade lösningar föreslagits 
    för att förlänga den kontroll som användare har över sina data i motsats till 
    den centraliserade massiv samling av personliga och känsliga data.\par
    }
    
    % Why should we care? (Why is it a problem?)
    % (b) Power of centralised service provider 
    % (c) Lack of control
    % (d) Threat to the right to be left alone
    Den kraften som tjänsteleverantören har i centrala informationssystem har visat 
    sig minska användarens integritet vid fall av missbruk, censur eller dataläckage. 
    Vidare har upplysningarna 2013 av ett globalt övervakningsprogram som leds av 
    offentliga efterlysningsinstitutioner i samarbete med några av tjänsteleverantörerna 
    av sådana centraliserade informationssystem påskyndat debatten om hur man vidtar 
    åtgärder för att motverka hot mot integritet. I synnerhet hotet mot den lagliga 
    ``rätten att bli ensam'' --- på engelska, ``right to be let alone'', som definierats 
    av Samuel Warren och Louis Brandeis år 1890 i sin inflytelserika laggransknings 
    artikel ``The Right to Privacy''..
    
    % What is our approach?
    % Decentralisation by means of privacy-preserving decentralised systems
    % Trade-offs: feature replication and security/privacy protection
    Sekretessskyddande decentraliserade system är trovärdiga lösningar på sådana 
    hot och ett av de vanligaste alternativen som används idag. Avlägsnandet av 
    den centrala auktoriteten kommer emellertid med två huvudsakliga kompromisser 
    att åtgärdas, efterlikna funktionerna i det centraliserade informationssystemet 
    på ett användbart sätt och överta övervakningen av säkerhets- och hoten som 
    en gång var ett centralt ansvar för centralen auktoritet.
    
    % What are our findings? (What is our approach?)
    % Privacy-preserving decentralised systems by means of examples developing the functionality and showing the concepts of decentralisation and security and privacy properties.
    % Brief explanation of the three solutions
    I vår avhandling vi användningen av decentraliserade system för integritetsskydd 
    och utvecklar tre lösningar för centraliserade informationssystem när det gäller 
    decentralisering, funktionalitet och uppnåelig säkerhet och integritet. I decentraliserade 
    informationssystem generellt visar vi en konkret mekanism för användarautentisering 
    via standard användar-lösenordsuppgifter med jämförbar användbarhet för standardiserade 
    centraliserade applikationer. Inom ramen för praktiska decentraliserade system 
    visar vi ett specifikt exempel på domänen för decentraliserade online sociala 
    nätverk --- på engelska, \aclp*{dosn} --- som implementerar en samordnings- 
    och samarbetsmekanism för att organisera händelser utan att behöva ha en betrodd 
    tredje part. Slutligen går vi tillbaka till de centraliserade systemen där närvaron 
    av den centrala myndigheten fortfarande krävs och i stället förbättrar en av 
    aspekterna av användarens integritet: anonymitet genom att visa en implementering 
    av ett system för att skicka in och klassificera dokument anonymt i akademisk 
    sfär i ett generiskt centraliserat system för integritetsskydd.
    
    % What are our findings?
    % Privacy-preserving protocols in centralised and decentralised systems that mitigate the dangers to personal privacy
    Våra lösningar är några konkreta exempel på hur integritet som datakontroll, 
    som det paradigm som Anita Allen förutser, kan uppnås i varierande grad i centraliserade 
    och decentraliserade informationssystem för integritetsskydd. Ändå hoppas vi 
    att de integritetsskydd protokollen vi föreslår och utvärderingen av säkerhets- 
    och sekretessegenskaperna kan vara användbara i andra scenarier för att mildra 
    de olika farorna för personlig integritet som vi står inför för närvarande.
