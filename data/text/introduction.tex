% -*- mode: TeX -*-
% -*- coding: utf-8 -*-

\requote{Intellectual beauty is sufficient unto itself, and only for it rather than 
for the future good of humanity does the scholar condemn himself to arduous and 
painful labors}{Santiago Ram{\'o}n y Cajal}

\lettrine{\textcolor[gray]{.25}{I}}{n} 1989, while at the European Particle Physics 
Laboratory, at the \acf{cern}, Sir Tim Berners-Lee invented the \Ac{www}. No one 
at that time would have imagined that such an invention would be central to the rapid 
development of the digital age that we are currently witnessing.

Such a period in human history, mainly characterised by an economy based on information 
processing as opposed to the traditional manufacturing of the industrial revolution, 
has widened the availability of uncountable services to the general population with 
access to the Internet. Not only new business opportunities have stemmed as technological 
developments matured but also services have been overhauled to make them available 
in the digital world.

To give an example, interpersonal communication --- the passing of information between 
entities --- has evolved throughout history as communication advances developed. 
The traditional exchange of paper letters as a means of communication between two 
parties, which still prevails today, has evolved into a much more sophisticated 
and complex form in the information age: electronic mail (e-mail), which is simply 
the parallelism in the digital era to such physical letters.

Such an evolution in the means of communication among individuals and other entities 
has also happened at other societal levels. For example, the social structures that 
individuals form when sharing similar interests or activities, namely social networks, 
have also seen a transposition in the digital era. The \Internet and the \ac{www} 
have taken the leading role as tools to convey different types of information from 
one party to another, expanding the boundaries of the concept of social networks, 
heavily based on tangible activity between individuals and entities.

While the concept of \acl{sn} is a theoretical term mainly used in the realm of 
social sciences to study and describe the social structures determined by the relationships 
between individuals, groups, and other types of entities such as societies; the 
technological developments of the digital era have popularised its virtual counterpart 
term: \aclp{osn}.

\Acp{osn} are online based applications where the users --- individuals and groups 
--- create profiles of themselves, connect with others by creating relationships, 
generate heterogeneous content, and share and exchange such content among their 
connections and other users in the network. These networks are, up to a certain 
extent, a complement to the established interactions that individuals already have 
in their daily lives \cite{SubrahmanyamRWE08}. Convenience of building and maintaining 
relationships and enjoyment are among the main motivations driving individuals to 
disclose personal information in such platforms \cite{KrasnovaSKH10}.

These \acp{is} started becoming popular in the second half of the 90s 
with services such as \LiveJournal or \Friendster evolving\footnote{For a comprehensive 
but concise survey on the history and evolution of these networks the reader is 
advised to read \cite{boydE07} and \cite{HeidemannKP12}} to the most popular ones 
nowadays such as \Facebook, \LinkedIn or \Twitter.

The building design describing such systems is very similar among them. Their design 
is usually monolithic with a data model that is centralised\footnote{With centralised 
we mean that the control of the data is under the control of one entity, such as 
the \ac{osn} service provider, regardless of how the data is stored physically, 
for example, in a distributed or centralised manner.} and a system model consisting 
of some variant of cloud computing architectures \cite{PallisZD11}. 

The main business model is also similar among most \acp{osn} providers. Instead 
of charging the users for the services of the \ac{osn}, the revenue model is based 
on selling advertisements to businesses and companies that want to reach a subset 
of the users in the platform. In fact, some argue that \acp{osn} have become an 
ecosystem for marketing rather than for information \cite{HannaRC11}, and ignoring 
these social media platforms may be the difference between prosperity or failure 
for business success \cite{HarrisR09}.

Such business revenue model based on advertising requires in most cases accepting 
some \ac{tos} and \acp{pp} where the users of the \ac{osn} give up certain rights 
on the data contained in the ``free'' profiles and any other data stemming from 
the interactions among them. For example, \Facebook states in its \ac{tos} that 
the user grants a ``non-exclusive, transferable, sub-licensable, royalty-free and 
worldwide license'' of any intellectual property content posted in the \ac{osn}, 
\LinkedIn and \Twitter state similar \ac{tos} as \Facebook~--- see \cref{chapter:thesis:excerpts-of-tos-and-pp} 
for some excerpts of the \ac{tos}.

Users are also more aware of the profit that \ac{osn} service providers get from 
their data, for example, via advertising. Moreover, the concern of users' data privacy 
in \acp{osn} has also increased in recent years particularly when it comes to configuring 
who can access their content. Some studies show that users' engagement with the 
\ac{osn} is linked to a higher awareness and more frequent changes of the privacy 
settings in the service \cite{boydH10}.

However, raising awareness among \ac{osn} users of the privacy settings or how they 
can better use the service to protect their privacy or, requesting more transparent 
policies from the \ac{osn} service provider is necessary but still not sufficient. 
Because an important aspect of personal data privacy is actually maintaining control 
over such information --- informational privacy \cite{Cavoukian96}.

Privacy in \acp{osn} has become a topic of ongoing discussions in the current century 
due to the business model of the service providers of \acp{osn}. Nevertheless, the 
focus of the discussions on privacy in \acp{osn} has shifted towards the idea of 
data control --- privacy was already envisioned as data control in the late 90s 
\cite{Allen99}.

\acp{osn} have been on the news for different unfortunate reasons related to privacy, 
to name a few, coercion into self-identification\footnote{See ``Facebook Locks Out Thousands, Now Wants Photo ID'' at \url{https://web.archive.org/web/20150921075721/http://www.conspiracyclub.co/2015/03/25/facebook-asking-gov-id-to-verify-account/}}, 
user data leakages\footnote{See ``Passwords for 32M Twitter accounts may have been hacked and leaked'' at \url{https://web.archive.org/web/20161102165137/https://techcrunch.com/2016/06/08/twitter-hack/}}, 
censorship\footnote{See ``LinkedIn Considers Changes After China Censorship Revealed'' at \url{https://web.archive.org/web/20150927040931/http://blogs.wsj.com/digits/2014/09/03/linkedin-considers-changes-after-china-censorship-exposed/}}, 
intentional misuse\footnote{See ``WhatsApp data sharing with Facebook must be stopped until it can be proved legal, European Union warns'' at \url{https://web.archive.org/web/20161029115621/http://www.independent.co.uk/life-style/gadgets-and-tech/news/whatsapp-data-sharing-facebook-eu-european-union-privacy-safety-opt-out-a7384586.html}} 
and even refusal to the right to be forgotten\footnote{See ``European Court Lets Users Erase Records on Web'' at \url{https://web.archive.org/web/20161013113612/http://www.nytimes.com/2014/05/14/technology/google-should-erase-web-links-to-some-personal-data-europes-highest-court-says.html}}

Moreover, \acp{osn} service providers, among other major technological companies, 
have been known to be in the thick of a global surveillance program --- a program 
known as \Ac{prism}\footnote{See ``NSA Prism program taps in to user data of Apple, Google and others'' at \url{https://web.archive.org/web/20161112033822/https://www.theguardian.com/world/2013/jun/06/us-tech-giants-nsa-data}} --- 
after the revelations\footnote{Such disclosures should not come as a striking surprise 
because a former intelligence analyst of the \aca{usa} Navy already testified in 
the 1970s about the illegal eavesdropping on \aca{usa} citizens \cite{Bamford14}.} 
of a former contractor's employee of the \ac{usa} government in 2013. For example, 
not only \Facebook and \Google, had full knowledge of having their users' privacy 
breached without their knowledge and consent but also assisted in such violations 
without much legal opposition on the abusive and indiscriminate demands of law enforcement 
agencies.

The aforementioned scenario certainly pictures a quite pessimistic and probable 
future as it seems that privacy is pretty much extinct, if it was somewhat possible 
in the past. Even though there are other plausible scenarios that can address the 
use of \acp{osn} in a privacy-preserving manner. A trivial one would be simply not 
using any technology that can pose a threat to privacy but this is impractical in 
the current digital era and irrational. Another solution could be some sort of institutional 
transparent control of service providers but this would allow for collusions between 
service providers and institutions as proven with the \ac{prism} program.

In fact, there is not a perfect solution that addresses the problem. However, the 
idea of privacy as data control is feasible enough to consider as a starting point. 
Because the user should be the one who decides what happens with her data, how it 
is dealt with and by whom. There is already technology aiding the users keep some 
control while allowing business models to operate without fully compromising users' 
privacy. For example, \Apple recently unveiled how it is embracing differential 
privacy to collect statistics of its users anonymously\footnote{See ``Apple's `Differential Privacy' Is About Collecting Your Data --- But Not Your Data'' at \url{https://web.archive.org/web/20160901192334/https://www.wired.com/2016/06/apples-differential-privacy-collecting-data/}} 
--- however, still not fully privacy-preserving because users can be identified 
in the broad sense of categories without getting the chance to decide whether they 
want to allow that or not. 

% DSS BEGIN
An example of a privacy-preserving solution in an \ac{is} is part of 
this thesis in \cref{article:thesis:document-submission-system}. In the context of an academic 
institution where personal information about individuals is collected --- for example, 
students' examination documents --- for further processing by a central authority 
--- such as a grading examiner --- we provide a privacy-preserving protocol based 
on blind signatures for anonymous document submission and grading --- the process 
of anonymous grading is popularly known as blind grading in fact. 

Although our solution provides a procedure to do blind grading in a privacy-preserving 
manner guaranteeing the correctness of the grading process, it relies on a central 
authority to process the data --- for example, to issue and link the credentials with 
the identities of the participants. Such authority --- an academic institution --- 
is a trusted party, but this cannot always be a reasonable assumption in other scenarios 
--- recall the governmental program \ac{prism}.
% DSS END

For these reasons, decentralisation\footnote{In the context of this thesis we adopt 
the definition by Rohit Khare where a decentralised system is ``one which requires 
multiple parties to make their own independent decisions''.} has been proposed as 
alternative to some centralised \acp{is} such as \acp{osn} --- namely \acp{dosn} --- 
as one of the foremost steps to minimise the impact on users' privacy by the central 
service provider. Yet decentralisation in \acp{is} does not come for 
free nor it is trivial to achieve because new challenges arise when there is no 
central provider to rely on for basic services such as storage, communication or 
identification.

The main idea in decentralised systems, such as \acp{dosn}, is to provide the services 
of the centralised counterpart without relying on the central authority which was 
assumed to be trusted --- fully or partially --- and making decisions on behalf 
of everyone in the system. This can be done by fully opening the system in such 
a way that every party --- known as peer --- is autonomous, hence interacting directly 
with other parties, for example, sharing information, and, entering or leaving the 
system freely --- anyone can do so at any time. A typical example of an information 
system that is completely decentralised is a \ac{p2p} network.

Such arbitrariness comes with trade-offs because the parties may have conflicting 
interests and goals making the decentralised system susceptible to additional attacks 
from malicious parties. Every party has now some responsibility because there is 
not a central authority where any party can turn to in case of misbehaviors. For 
example, impersonation, denial of service, \Sybil attacks, collusion or deception 
are some of the security threats that a decentralised scenario has to face besides 
other technical ones such as lack of cooperation, network bootstrapping or, who and 
how to trust \cite{BucheggerA09}.

Consequently, decentralisation shifts the primary focus of the \ac{is} 
to privacy, security and data portability --- and usable functionality thereafter. 
As the power of the central authority is overruled and distributed among the parties 
in the decentralised system new threats arise in the decentralised context. Without 
any privacy-preserving approach addressing these challenges, the personal data, 
that is now unlocked from the centralised ``silo'', is at the hand of everyone in 
a virtually ungoverned environment.

% P2P BEGIN
Privacy-preserving decentralised \acp{is} is the other focus of this thesis. On 
one side, we show how the classical functionality of user authentication can be 
implemented in a decentralised manner in \cref{article:thesis:passwords-peer-to-peer} 
for an \ac{is} in general --- in fact, for a fully decentralised system such as 
a \ac{p2p} network --- with the use case of \acp{osn} in mind. We supplement the 
basic login functionality with other features such as enabling a user to change 
her password or even recover a forgotten one via her e-mail or some security questions 
or, store her login credentials on some other device --- in a secure form --- for 
example, a mobile phone, so that she does not have to log in every time she wants 
to use it.

Scalable and usable privacy-preserving user authentication in decentralised systems 
allows for a wider adoption and transition to decentralised \acp{is}. 
For example, the feature of recovering the password via e-mail is a popular solution 
used by many vendors in centralised systems --- including \acp{osn} service providers 
\cite{Kuzma11}. However, such a recovery functionality fails its primary function 
when the service provider is not available, or if it becomes malicious and refuses 
to use the e-mail originally provided by the user. 

In our solution we propose a \((n, k)\)-secret sharing approach to overcome the absence 
of a central authority. The user chooses $n$ peers who can help her on the recovery 
task and among those, $k$ peers that should be available when recovering the password. 
The user is in control of both parameters --- $n$ and $k$ ---  at all times, unlike 
in the centralised scenario.
% P2P END

% EI BEGIN
On the other side, we show how the basic functionality of organising an event, usually 
available in centralised \acp{osn}, can be implemented in a privacy-preserving manner 
in \acp{dosn} in \cref{article:thesis:events-invitations-dosns}. Our proposed solution 
allows a user to create events and invite other users who can confirm their attendance. 
While this functionality is rather simple when there is a trusted third party, such as 
the service provider in an \ac{osn}, implementing such a feature without a trusted 
broker that guarantees fairness in the process is not trivial. For example, keeping 
track of who is invited or showing the total number of attendees to those invitees 
who are attending.

Our mechanism uses standard cryptographic primitives allowing other features such 
as revealing some event specific information to a group of users --- to those who 
commit to attend an event --- or even detect any misbehaving party --- for example, 
an invitee can prove that she accepted an invitation and did not get further details 
about the event from a malicious organiser.

With our solution we aim at a broader spectrum of privacy-preserving applications 
outside the context of \acp{dosn}. The cooperation and coordination characteristics 
of our specific use case and our secure decentralised solution suggest for applications 
in other domains where there is some collaboration involved between the parties 
and a trusted third party is not ideal or possible at all.
% EI END

Finally, we acknowledge the urgent need for more privacy in \acp{is}. 
In this thesis, we take the approach of decentralising such systems as a way to 
achieve the ideal of privacy as data control as envisioned by Allen \cite{Allen99}. 
We contribute with some insights on how to enhance the control of the user on her 
data in general centralised and decentralised systems, and in particular ones such 
as \acp{dosn}. We also suggest usable privacy-preserving protocols for some specific 
features and show their utility in some concrete scenarios.

\section{Motivation}
    \label{section:thesis:motivation}
Our rationale for the line of work of this thesis is data control in the broad sense 
and in line with the notion of ``privacy as data control'' in \cite{Allen99}. We 
drive our reasoning on the three following core user personal data rights\footnote{The 
reader is advised to read more on the three main rights we have in mind in the \Ac{udm} 
at \url{https://web.archive.org/web/20161010132230/https://userdatamanifesto.org/}.}: 
``control over user data access'', ``knowledge of how data is stored'' and which 
laws or jurisdictions apply and, ``freedom to choose a platform'' without experiencing 
any vendor lock-in.

% The rights stated in the \ac{udm} are, as a matter of fact, a modern version of the
% three notions of `privacy' that \citeauthor{Allen99} stated in \cite{Allen99}: ``personal
% data control or rights of data control'', ``right of personal data control'' and,
% ``enhancing personal data control by individuals is the optimal end of privacy regulation''.

Data control in the essence of data privacy and information security is a 
subject that has gained a lot of attention in the last decade. Particularly from 
major technological corporations such as \Google, \Apple or \Facebook because they 
have been identified as necessary collaborators of illegal and indiscriminate institutional\footnote{See 
``EU's highest court delivers blow to UK snooper's charter'' at \url{http://web.archive.org/web/20161221123617/https://www.theguardian.com/law/2016/dec/21/eus-highest-court-delivers-blow-to-uk-snoopers-charter}} 
surveillance of citizens world-wide \cite{Lyon14}.

However, users of \acp{osn} owned by some of these major corporations are much more 
aware than ever of such a collaboration of service providers and governmental institutions 
for mass surveillance \cite{Madden14}. Moreover, users are more willing than before 
to do something to protect their privacy although in a seemingly paradox they are 
also more eager to share more personal information to an even wider audience \cite{StutzmanGA13}. 

User privacy-related behavior is usually context-dependent because different types 
of information prompt different degrees of sensitivity. For example, social security 
numbers are considered much more sensitive than some habits such as purchasing or 
browsing the \ac{www}. But such privacy concerns and preferences are subject to 
manipulation as those interested parties, such as governments or corporations, distort 
the context to their interests \cite{AcquistiBL15}.

A well articulated security and privacy strategy is necessary but it is not sufficient 
for a corporation (or governmental institution) to protect the data of its customers 
and users (or citizens) because the data itself is still out of the control of the 
user. Not only does the user not possess such data physically but also not 
control who accesses it or even if and how it gets transferred to third parties --- 
in many cases, unauthorised.

On the one hand, we aim at improving users' data control as much as it is technologically 
possible, leaving legal policies such as \ac{tos} or \ac{pp} and, the ethical debate 
out of the scope of this thesis. We believe that removing the central provider in 
a centralised \ac{is} such as an \ac{osn} is one of the very first steps towards 
returning the control of the data back to the user. However, there are trade offs 
to consider when getting rid off such centralised entities. For example, whatever 
guarantees that two parties can communicate between each other without a third one 
overseeing ought to be identified, analysed, researched, designed and implemented.

On the other hand, the security and privacy we may achieve by removing such a central 
provider will always be limited to the trust that we put in the machinery that runs 
any privacy-preserving protocol. For example, the communication application \Signal 
provides some strong guarantees for secure communication between two or more parties, 
but confidence in such assurances --- which have been publicly scrutinised as 
the source code of the application is available for anyone to look at --- are ultimately 
dependent on the trust that the user has on the hardware device and the software platform 
--- particularly when they come from the same vendor.

Finally, we also have the desire to reduce the impact of personal data privacy in 
people's lives by raising awareness of the topic with our solutions. We want to 
broaden and deepen the ongoing political debate and legal discussions on the topic 
of privacy not only in \acp{osn} but also in many other centralised \acp{is}. However, 
we reckon there is still a long road ahead on the quest of privacy and this is barely 
the beginning.

\subsection{The utopia of privacy}
    \label{subsection:thesis:utopia-of-privacy}
Our philosophy for privacy should be understood as an ideal that we aim at. Our line 
of work where the individual exercises as much control as possible on her data can 
be perceived as an utopia and, to a limited extent, we must agree. Because we can 
not ignore the reality of living in a law-abiding world and the need for common 
sense in daily life. 

The same democracy that is giving us the freedom of free speech and allows us to 
write these lines, accepted to have laws to regulate itself while also setting certain 
boundaries to freedom. For example, it is common in many jurisdictions that service 
providers, such as those operating telecommunication and telephone networks, are 
legally obliged to help law enforcement in some situations, when the ruling of a 
court says so, and breach the privacy of one or more suspect individuals --- the 
so called \ac{li}.

However, our approach in this work aims at providing solutions to reduce the impact 
of abusive, intrusive and indiscriminate demands for personal information. For example, 
the surveillance that law enforcement may demand of a specific suspect must be supported 
with some proof to be presented at a court for a judge to approve is not the same 
as the demands of laws to make legal for law enforcement to circumvent the courts 
to wiretap on any individual at will.

It is precisely the court ruling situation that service providers cannot avoid in 
most cases and we do not dispute nor criticise. In fact, the solution that service 
providers seem to be taking nowadays is similar to the one we advocate in this 
work: ``privacy as data control''. Service providers are trying to reduce their 
liability when it comes to personal data storage. In such cases, they are aiming  
at becoming information proxies and minimise access capabilities on the personal 
data of their users that they store, either fully or partially depending on their 
business model.

Some service providers claim that they cannot access the information they hold of 
a specific user because such information has been encrypted with a key for which 
the user is the only holder. Therefore, even in the case of a court requirement 
to access the personal information of a user, the best the service provider can 
do is to provide a copy of the encrypted data --- hence, complying with the court request 
--- but rendering the desires of law enforcement to access personal information 
futile\footnote{Note that the request of laws to forbid encryption mechanisms or even require 
service providers to keep copies of encryption keys is still a demand of some legislators. 
Fortunately, such requests encounter heavy criticism from other legislators and 
freedom of speech and human rights activists because it would render democracy no 
different than any of the still existing authoritarians modern dictatorships, at 
least when it comes to privacy.}. 

Possibly striking and somewhat in conflict with what we advocate in this thesis, 
we accept the need for vetoing the desires for privacy in some some situations such 
as those where human life is at risk. For example, advancements in electronics have 
allowed for the embedment of \ac{gnss} chipsets in modern mobile phones whose capabilities 
are used every day by emergency services to locate and save people's lives.

We are still far from a digital dystopia where no secret is safe but more work is 
needed to avoid reaching that frightening point. We aim at providing some privacy-enhancing
tools so that sharing personal data is possibly at the terms of the owner of such 
data and within the bounds of a democratic society, ethics and common sense. Mass 
and indiscriminate surveillance will never be acceptable.

\section{Outline}
    \label{section:thesis:outline}
After putting the reader in the context of \acp{osn} and the need for privacy-preserving 
approaches by means of decentralisation of such online based applications --- \acp{dosn} 
--- in the current \namecref{chapter:thesis:introduction}, we continue with a background 
introduction to the main concepts that we base our work on in \cref{chapter:thesis:background}. 
We provide a succinct overview of concepts within \acp{is} in \cref{section:thesis:information-systems-on-the-importance-of-capabilities}, 
centralised and decentralised \acp{osn} in \cref{section:thesis:osns-centralisation-vs-decentralisation} 
and privacy in \cref{section:thesis:privacy-a-never-ending-battle}. And we finalize 
the \namecref{chapter:thesis:background} with a brief review of the state of the art 
on decentralised \acp{is} with an emphasis on \acp{dosn} in \cref{section:thesis:related-work}.

We continue in \cref{chapter:thesis:our-research-problem} with a description of our research 
problem and the challenges we faced to detail our research methodology in \cref{section:thesis:methodology}. 
Our contributions are listed in \cref{chapter:thesis:our-contributions} along with 
a summary of each article for the reader's convenience. We complete this thesis 
with our conclusions and future directions in \cref{chapter:thesis:conclusions}.

We finally include a re-print of each publication in the corresponding \namecref{part:published-articles}. 
Namely, \usebibentry{KreitzBGRB12}{title}, \usebibentry{RodriguezCanoGB14}{title}, 
and \usebibentry{GreschbachREB15}{title} (\cref{article:thesis:passwords-peer-to-peer,article:thesis:events-invitations-dosns,,article:thesis:document-submission-system} 
respectively). Moreover, and for the reader's reference, we have also included selected 
excerpts of \ac{tos} from three popular \ac{osn} service providers in \cref{chapter:thesis:excerpts-of-tos-and-pp}.


\section{Conventions}
    \label{section:thesis:conventions}
In this thesis we are using the following conventions in the language used,
\begin{itemize}
    \item The gender used throughout the text while referring to a human being using 
    a computer --- user --- may unintentionally differ, although we tried to be 
    consistent with the gender `she'. However, this thesis is neutral towards gender 
    but to ease the reading we chose such gender --- except in the abstract.

    \item Data and information can be considered synonyms in the broader sense --- 
    the definition of data is a piece of information. In the context of this thesis 
    we will use the latter --- information --- as the abstract fact about something, 
    while the former --- data --- as the encoded form of such information that can 
    be transferred over a network and interpreted by a computer.
    
    \item All the \acp{url} in this thesis --- except the ones in \cref{chapter:thesis:excerpts-of-tos-and-pp} ---
    are provided via the \Wayback machine, a digital archive of the \ac{www} and 
    other information on the \Internet created by the \InternetArchive non-profit 
    digital library organisation. The reader has the possibility of accessing a cached 
    snapshot of the version that was used when this thesis was written and also 
    the original \ac{url}. In order to access the original \ac{url}, the reader 
    is advised to remove the timestamped prefix that precedes the original \ac{url} 
    of the form \url{https://web.archive.org/web/YYYYMMDDHHMMSS/}. For example, 
    given the original \ac{url}, \url{https://en.wikipedia.org/wiki/Privacy}, the 
    \Wayback machine created the following snapshot and its \ac{url} as of the time 
    of this writing, \url{https://web.archive.org/web/20161111044417/https://en.wikipedia.org/wiki/Privacy}.
\end{itemize}
