% -*- mode: TeX -*-
% -*- coding: utf-8 -*-

\requote{Poetry is a succession of questions which the poet constantly poses}{Vicente 
P{\'i}o Marcelino Cirilo Aleixandre y Merlo}

\lettrine{\textcolor[gray]{.25}{W}}{e} aim at providing some privacy-preserving 
methods to improve user data control in the realm of centralised and decentralised 
\acp{is}, and in a particular instance of \acp{cbis}, \acp{dosn}.

Centralised \acp{is} are the most popular systems on the \Internet as the business 
strategy of most enterprises owning such systems is fundamentally built around control. 
For example, \Facebook, which has been accused\footnote{See ``Zuckerberg: Facebook will develop tools to fight fake news'' at \url{https://web.archive.org/web/20161120010511/http://money.cnn.com/2016/11/19/technology/mark-zuckerberg-facebook-fake-news-election/}} 
recently of not doing enough to prevent fake news in its network, would probably 
benefit from sharing some details on how its news filtering processes work for the 
research community to assess, but doing so would likely expose details of its internal 
mechanisms to its competitors.

In this thesis we aim at reducing the amount of control that a service provider 
can do in a centralised \ac{is}. Either by improving the privacy of some steps of 
a process in a centralised scenario --- if we are constrained to keep such authority, 
for example, in a semi-trusted scheme, or by analysing what happens when the central 
authority is taken away, and designing and evaluating privacy-enhancing alternatives 
instead.

\section{Challenges addressed}
    \label{section:thesis:challenges-addressed}
Among the wide spectrum of problems and questions that decentralisation of centralised 
\acp{is} such as \acp{osn} poses, we have first focused our efforts on keeping the 
functionality of some features in decentralised systems as close as possible to 
what these would be in the centralised scenario.

In parallel, such decentralisation triggers our second concern in terms of privacy.
Not only do we consider this topic in such a setup where the central authority does 
not exist or it plays a minimal role but also in a more generic fashion for an \ac{is}. 
When reducing the tasks of the central authority to the minimum, the users in the 
system are exposed to some security and privacy threats that should be analysed 
and addressed.

Each one, decentralisation and privacy, raise more questions than any thesis or 
research will ever be able to answer. Therefore, what follows is a list of the main 
questions that we believe this thesis is answering in the aforementioned domains with 
the corresponding reference to the articles that we believe shed some light on the 
answer. There may be additional questions we have not explicitly considered but 
the reader finds an answer to in our work. Moreover, the reader may expect answers 
to some questions that we have not even thought of at the time of this writing --- 
though this is the beauty of research.

\begin{itemize}
    \item Can we decentralize an \ac{is}? If so, how?\\
    \Cref{article:thesis:passwords-peer-to-peer,article:thesis:events-invitations-dosns}
    \item Can we mimic some of the features of centralised \acp{is} in a decentralised setup?\\
    \Cref{article:thesis:passwords-peer-to-peer,article:thesis:events-invitations-dosns}
    \item What are the trade-offs when protecting the privacy of the user in a decentralised \ac{is}?\\
    \Cref{article:thesis:passwords-peer-to-peer,article:thesis:events-invitations-dosns}
    \item Can centralised \ac{is} be more privacy-preserving?\\
    \Cref{article:thesis:document-submission-system}
\end{itemize}

\section{Delimitations}
    \label{section:delimitations}
In the realm of this thesis, centralised and decentralised \acp{is}, including a 
user case of the latter ones, \acp{dosn}, there are some aspects that are considered 
in a limited manner or topics that are not considered at all --- because they are 
out of the scope of this work. We describe them as follows,

\begin{itemize}
    \item We are not aiming at building, nor do we have, a functional system as per 
    the definition of \ac{cbis} in \cref{section:thesis:information-systems-on-the-importance-of-capabilities}.

    \item We do not get into the debate or in depth analyse the trade-offs between 
    centralised and decentralised \acp{is} --- regardless of our preference towards 
    the later ones. In fact, we acknowledge that this is an interesting open question 
    that will likely bring new challenges in the future given the concentration 
    of power that cloud services providers are experiencing nowadays.
\end{itemize}

\section{Methodology}
    \label{section:thesis:methodology}
Computer science is the study of the phenomena that surrounds computers by means 
of the theory, experiments and design that leads to the actual construction of computers. 
The computer is not only a collection of interconnected pieces of hardware but also 
a programmed ``living'' device --- as defined by Newell\etal the studied ``organism'' 
--- \cite{NewellS76}.

As such, and derived from the natural sciences as a continuous process, we take 
the scientific method as the main approach to address our research problem by systematic 
observation, measurement, experimentation, induction and formulation of hypotheses, 
testing of our deductions, and possibly the modification of such hypotheses \cite{Oxford14}.

Among the most prominent research methodologies used nowadays --- quantitative and 
qualitative --- we do not side with any of them and instead take a mixed approach. 
Mixed methods research gives a greater flexibility when answering research questions 
rather than constraining approaches or ideas. We try to find research questions 
that offer the best chance to obtain useful and meaningful answers. And we think 
that a combined approach achieves the best results \cite{JohnsonO04}.

Such mixed methods approach leads us into design science research methodologies 
for \acp{is}, which consist of six main steps: identification and motivation of the 
problem, definition of the objectives for a solution, design and development of 
the solution, demonstration, evaluation and finally, communication of the research 
\cite{PeffersTRC07}. We use the previous criteria to classify and describe the works 
included in this thesis in \cref{table:papers-methodologies}.

% { % restricting the table to be always on top of a page to this case (local not global)
% \makeatletter
% \setlength{\@fptop}{0pt}
% \setlength{\@fpbot}{0pt plus 1fil}
% \makeatother
{
\newcolumntype{L}{>{\centering}m{3.5cm}}
\begin{sidewaystable*}[t!]
    \centering
    \caption{Summary of research design steps of the articles included in this thesis}
    \begin{tabu}{>{\arraybackslash}m{3.55cm}LLL@{}m{0pt}@{}}
        \toprule
        % See http://tex.stackexchange.com/questions/156219/proper-centering-with-cmidrule-and-multi-row-and-column for the reason of the [4] optional argument in multirow
        \multirow{2}[4]{*}{\textsc{Design Research Step}} & \multicolumn{3}{c}{\textsc{Articles}} & \\
        \cmidrule(lr){2-4}
         & \ref{article:thesis:passwords-peer-to-peer} & \ref{article:thesis:events-invitations-dosns} & \ref{article:thesis:document-submission-system} & \\
        \midrule
        Problem identification, motivation & decentralised authentication & decentralised events organisation & anonymous document submission & \\[1em]
        Definition of objectives & simple, usable & fair, \ac{ttp}-free & correct, fair & \\[1em]
        Solution design, development & protocols, experimentation & protocols, auditing & security properties, system design & \\[1em]
        Demonstration & - & - & prototype (see \cite{Ericsson15})& \\[1em]
        Evaluation & security, privacy, and performance analysis & security and privacy analysis & security and privacy analysis & \\[1em]
        Communication & See \cite{KreitzBGRB12} & See \cite{RodriguezCanoGB14} & See \cite{GreschbachREB15} & \\[1em]
        \bottomrule
        \multicolumn{4}{l}{\footnotesize \ref{article:thesis:passwords-peer-to-peer} :: \usebibentry{KreitzBGRB12}{title} } \\
        \multicolumn{4}{l}{\footnotesize \ref{article:thesis:events-invitations-dosns} :: \usebibentry{RodriguezCanoGB14}{title} } \\
        \multicolumn{4}{l}{\footnotesize \ref{article:thesis:document-submission-system} :: \usebibentry{GreschbachREB15}{title} } \\
    \end{tabu}
    \label{table:papers-methodologies}
% \begin{landscape}
% \begin{adjustbox}{angle=90}
% \end{adjustbox}
% \end{landscape}
\end{sidewaystable*}
}

