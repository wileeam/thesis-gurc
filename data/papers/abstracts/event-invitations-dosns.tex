% -*- mode: TeX -*-
% -*- coding: utf-8 -*-

% 5 questions:
%
% What is the problem:
% Why is it a problem: 
% Why should we care: 
% What is our approach: 
% What are the findings: 

\Acp*{osn} have an infamous history of privacy and security issues. One approach 
to avoid the massive collection of sensitive data of all users at a central point 
is a decentralized architecture.

An event invitation feature -- allowing a user to create an event
and invite other users who then can confirm their attendance --
is part of the standard functionality of \acsp*{osn}. 
%
We formalize security and privacy properties of such a feature 
like allowing different types
of information related to the event (\eg how many people are
invited/attending, who is invited/attending) to be shared with 
different groups of users (\eg only invited/attending users).
%or only attending users to see an additional private event description.

Implementing this 
feature in a Privacy-Preserving \Acl*{dosn} is non-trivial because there is 
no fully trusted broker to guarantee fairness to all parties involved. 
%
We propose a secure decentralized protocol for implementing
this feature, 
using tools such as storage location indirection, ciphertext inferences
and a disclose-secret-if-committed mechanism, derived from standard
cryptographic primitives.

The results can be applied in the context of Privacy-Preserving \acsp*{dosn}, but 
might also be useful in other domains that need mechanisms for cooperation and coordination, 
\eg \Acl*{cwe} and the corresponding collaborative-specific tools, \ie groupware, 
or \Acl*{cscl}.
