% -*- mode: TeX -*-
% -*- coding: utf-8 -*-

% What is the problem: Deficient anonimity and unlinkability in grading 
% Why is it a problem: Subjective grading 
% Why should we care: Regain privacy?
% What is our approach: Protocols based on standard crypto tools: blind digital signatures
% What are the findings: Anonimity and unlinkability can be achieved with minor trade offs

%Privacy regulations mandate the collection of \Acl*{pii}, \eg full name or date 
%of birth, to be proportionate and necessary for the purpose of the offered service, 
%for example, registration at an online e-commerce website. However, in some instances, 
%the amount and type of such personal data does not match the actual needs nor the intended 
%usage while in other cases the availability of non-essential information may lead 
%to undesired biased assessments.

Document submission and grading systems are commonly used in educational institutions.
They facilitate the hand-in of assignments
by students, the subsequent grading by the course teachers and the 
management of the submitted documents and corresponding grades.
But they might also undermine the privacy of students, especially
when documents and related data are stored long term with the risk of
leaking to malicious parties in the future. 
%
%Discriminatory
%judgement can be another issue in these systems when teachers during grading
%know the identity of the student who authored a certain document.
%
We propose a protocol for a privacy-preserving, anonymous document
submission and grading system based on blind signatures.
Our solution guarantees the unlinkability of
a document with the authoring student even after her grade has been
reported, while the student can prove that she received the grade
assigned to the document she submitted. 
%
We implemented a prototype of the proposed protocol to show its
feasibility and evaluate its privacy and security properties.