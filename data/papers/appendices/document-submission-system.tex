% -*- mode: TeX -*-
% -*- coding: utf-8 -*-


\section{Clarifications}
    \label{section:thesis:appendix:dss:clarifications}
In the following list we try to further explain some words or sentences as published 
in the original text that otherwise could be misleading or even contradictory,
\begin{itemize}
    \item Stylometry\\
    In \cref{section:document-submission-system:anonymous-document-submission-system} 
    we assume that an adversary does not have the resources to achieve authorship 
    attribution by analysis of the linguistic style of writing --- hence, our protocol 
    does not protect against stylometry.
    
    In our scenario, we believe that such assumption is realistic enough because 
    our protocol is agnostic to the content of the submitted document. Reducing 
    the disclosure of information is out of the scope of our solution and a problem 
    on its own.
    
    We believe that the privacy-preserving and anonymisation protocol that we propose 
    is a necessary first step though we acknowledge that it is not sufficient.
    
    \item \Ac{dos} attacks\\
    There is a possibility for \ac{dos} attacks on the server side while the submission 
    of documents is in progress because the server does not know who submits what 
    due to blinding nor from where due to the anonymous communication channel.
    
    Although this is future work, there are some solutions to mitigate selective 
    \ac{dos} attacks in our scenario of anonymous communication channels. For example, 
    in \cite{DasB13}, or even in \cite{JansenTJS14} a 
    
    We have yet to explore mitigating solutions in our scenario of a centralised 
    storage where the stored files are unidentifiable during a period of time because 
    the server's storage is subject to flooding in such case.
    
    \item Popularity of blind signatures\\
    Blind signatures are widely used in the academic area of cryptography, for example, 
    in voting protocols or digital currencies. 
\end{itemize}

\section{Assumptions}
    \label{section:thesis:appendix:dss:assumptions}
In the following list we make explicit some assumptions we implied in the original 
text as published or supplement some of those we did state in the text,
\begin{itemize}
    \item Compromising capabilities of the adversary\\
    In some of our scenarios, we assume that the adversary has the capability to 
    compomise the server where the proposed \Ac{adss} is running. However, such 
    compomising capabilities are limited to the application level where our proposed 
    solution lies. For example, the adversary cannot affect the functionality or 
    security of the underlying \ac{os} primitives used by our system such as the 
    entropy of the \acp{rng}.
    
    \item Randomness of the random blinding factors\\
    It is in the student's best interest to randomly choose the random blinding 
    factors --- $b_{pass}$ and $b_{fail}$, otherwise a student who is willing to 
    disclose her identity may choose such factors in a less random manner, for example, 
    to obtain some more favorable grade.
    
\end{itemize}


