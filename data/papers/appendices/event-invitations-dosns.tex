% -*- mode: TeX -*-
% -*- coding: utf-8 -*-


\section{Clarifications}
    \label{section:thesis:appendix:ei:clarifications}
In the following list we try to further explain some words or sentences as published 
in the original text that otherwise could be misleading or even contradictory,
\begin{itemize}
    \item ``\acp{osn} have an infamous history of [...] issues...'' 
    and ``...popularity of these services...''\\
    \Acp{osn} have been on the frontpage of numerous media as responsible for intentional 
    and unintentional cases of data leakages, censorship and collaboration with 
    national governments to access personal data without consent --- see some examples 
    in the introduction. 
    
    However, such presumable bad publicity on management of their users' privacy 
    does not seem to affect their success. Every year, the service providers of 
    the major \acp{osn} report record numbers on their growth.
    
    Though publicity of such privacy breaches has made users of these services more 
    aware of their own privacy and implications of their data sharing activities 
    in the network.
    
    \item Number of public/private keypairs $rk_1/rk_1^S, \dots, rk_n/rk_n^S$\\
    The number of keypairs $n$ generated by the organiser \o{} to encrypt the 
    entries on the commit-list is sufficiently large to allow every invitee to be 
    able to commit to attend the event \eo{} --- in other words, $n >> |\I|$.
    
    Note that $n$ can not be the number of invited users, $|\I|$, because any invited 
    user could simply learn such number by counting the stored keys in the event 
    object \eo{}. 
    
\end{itemize}

\section{Assumptions}
    \label{section:thesis:appendix:ei:assumptions}
In the following list we make explicit some assumptions we implied in the original 
text as published or supplement some of those we did state in the text,
\begin{itemize}
    \item Communication among participants\\
    In our threat model described in \cref{subsection:event-invitations-dosns:threat-model}, 
    the exchanges of information --- messages --- among the participants in the 
    protocols should be assumed to be protected, for example, against traffic analysis. 
    
    If the traffic was not protected, a third party could easily infer who are the 
    invitees by simply observing the outgoing messages of the organiser when she 
    was issuing the invitations to the subset of invited users.
    
    Some traffic analysis countermeasure could be masking the communication channel 
    with dummy traffic, besides the communication channel itself being encrypted.

    \item Organiser trust\\
    %TODO Check the threat model section for type of malicious: honest-but-curious perhaps?
    Although our protocols are \ac{ttp}-free, we can not protect against an organiser 
    not following the protocol (or the participants for that matter). Therefore, there is some 
    certain trust implied in either party, and reason for us to provide our commit-disclose 
    protocol technique to, at least, identify misbehavior --- but unfortunately, 
    we cannot prevent them because there is no technology that can coerce someone 
    into doing something they don't want to do.

    \item Size of protocols' objects\\
    The largest object in our protocol is the event \eo{}, which accounts for various 
    data as depicted in \cref{figure:event-invitations-dosns:overview-objects-actions}. Namely, 
    publick keys: \e{}, \o{} and $rk_1/rk_1^S, \dots, rk_n/rk_n^S$, descriptions: 
    \dP{} and \dS{}, links: \ILL{}, \ILK{}, \CLL{} and \DLL{}, and privacy settings.
    
    The largest output of a secure and commonly used hash function, SHA-2, 
    is 512 bits. Moreover, a common key size\footnote{As of January 2016, the recommended 
    key size by the \ac{nsa} when using the RSA algorithm for digital signatures 
    is 3048 bits --- see \cite{iad16}} used in public/private cryptosystems such 
    as RSA is 2048 bits.
    
    Therefore, if we consider a key size for the public/private keypairs of 2048 bits, 
    we assume that the links are represented by a hash of 512 bits in our storage 
    system, and the descriptions and privacy settings are also represented as links, 
    we have that the size of the event object \eo{} is, at least, $(n + 3)*2048 + 3*512 = (n + 2)*2048 + 4*512 + 2*512 + 512$ 
    bits, excluding file format encoding and encryption overhead.
    
    For example, for $n = 100$, an \eo{} is less than 27 kilobytes. Even if we would 
    embed the descriptions and privacy settings into the object and consider overheads, 
    the size of \eo{} is rather low. Similarly, the size of the different lists 
    can be calculated in the same manner and expected to be low.
    
\end{itemize}

\section{Corrections}
    \label{section:thesis:appendix:ei:corrections}
\begin{itemize}
    \item Where in \cref{subsection:event-invitations-dosns:security-properties} 
    says ``Property IIP and ICP are closely related in the sense that if IIP holds 
    for a certain set of users, then...'' it should say ``Property IIP and ICP are 
    closely related in the sense that if IIP holds relative to a certain set of 
    users, then...''
\end{itemize}
