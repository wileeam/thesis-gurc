% -*- mode: TeX -*-
% -*- coding: utf-8 -*-

\documentclass[showtrims,oldfontcommands]{kthesis}

%\usepackage[sfdefault,lf]{carlito}
\usepackage[lf]{carlito}
%% The 'lf' option for lining figures
%% The 'sfdefault' option to make the base font sans serif
\usepackage[T1]{fontenc}
\renewcommand*\oldstylenums[1]{\carlitoOsF #1}

\usepackage{textcomp}
\usepackage{lmodern}
\usepackage[utf8]{inputenc}
\usepackage[swedish, english, french, spanish]{babel}

\usepackage[svgnames]{xcolor}
%\usepackage{tikz}

% *** disable reset of footnote counter every chapter
\usepackage{chngcntr}
\counterwithout{footnote}{chapter}

% *** enables the checkmark commands: \ding{51} and \ding{52}
\usepackage{pifont}

% *** enables changing the label in enumerate environments directly
\usepackage{enumitem}

% *** for the boxes in the appendix alfa
\usepackage{framed}

% *** enables full bibliographic entries within text ***
\usepackage{bibentry}
\nobibliography*

% *** citation numbers will be sorted and properly "compressed/ranged" ***
\usepackage[noadjust]{cite}
% *** allowed for \citeauthor and \citeyear commands ***
\usepackage[authoryear, numbers, square]{natbib}
\usepackage{usebib}
\bibliographystyle{acm}
\setcitestyle{authoryear,aysep={},open={(},close={)}}
\bibinput{data/bibliography} % <--- the same argument as in \bibliography

\usepackage{lettrine}

\usepackage{booktabs}
\usepackage{multirow}

% DEBUG
% *** provides random text to fill up boxes, etc ***
\usepackage{lipsum}
\listfiles

\usepackage[explicit]{titlesec}
\usepackage[many]{tcolorbox}


\usepackage{mfirstuc}
\usepackage[single, hyperref=true]{acro}

% *** capitalisation config for acronyms ***
\acsetup{uc-cmd=\capitalisewords}
% *** acronyms file loading ***
% -*- mode: TeX -*-
% -*- coding: utf-8 -*-


\DeclareAcronym{abc4trust}{
	short        = ABC4Trust,
	long         = attribute-Based credentials for trust
}

\DeclareAcronym{cbis}{
    short           = CBIS,
    short-plural    = 's,
    long            = computer-based information system
}

\DeclareAcronym{cc}{
    short       = CC,
    long        = creative commons
}

\DeclareAcronym{cern}{
    short       = CERN,
    long        = Conseil Europ{\'e}en pour la Recherche Nucl{\'e}aire,
    foreign     = European Organization for Nuclear Research,
    foreign-lang = english
}

\DeclareAcronym{cscw}{
	short        = CSCW,
	long         = computer-Supported collaborative work
}

\DeclareAcronym{cscl}{
	short        = CSCL,
	long         = computer-Supported collaborative learning
}

\DeclareAcronym{cwe}{
	short        = CWE,
	long         = collaborative working environment
}

\DeclareAcronym{dht}{
    short       = DHT,
    long        = distributed hash table
}

\DeclareAcronym{dosn}{
    short       = DOSN,
    long        = decentralised online social network
}

\DeclareAcronym{fcfs}{
	short        = FCFS,
	long         = first-come{,} first-served
}

\DeclareAcronym{ibm}{
	short        = IBM,
	long         = international business machines corp.
}

\DeclareAcronym{irma}{
    short        = IRMA,
    long         = i reveal my attributes
}

\DeclareAcronym{is}{
    short           = IS,
    short-plural    = 's,
    long            = information system,
    long-indefinite = an
}

\DeclareAcronym{it}{
    short       = IT,
    long        = information technology
}

\DeclareAcronym{mooc}{
	short        = MOOC,
	long         = massively open online course
}

\DeclareAcronym{nsa}{
    short       = NSA,
    long        = national security agency
}

\DeclareAcronym{osn}{
    short       = OSN,
    long        = online social network,
    long-indefinite = an
}

\DeclareAcronym{pii}{
	short        = PII,
	long         = personal identifiable information
}

\DeclareAcronym{p2p}{
    short       = P2P,
    long        = peer-to-peer
}

\DeclareAcronym{pecole}{
	short        = PECOLE,
	long         = Peer-to-pEer COLlaborative Environment
}

\DeclareAcronym{pet}{
    short               = PET,
    long                = privacy enhancing technology,
    long-plural-form    = privacy enhancing technologies
}

\DeclareAcronym{pgp}{
    short       = PGP,
    long        = pretty good privacy
}

\DeclareAcronym{pki}{
    short       = PKI,
    long        = public key infrastructure
}

\DeclareAcronym{pp}{
    short               = PP,
    long                = privacy policy,
    long-plural-form    = privacy policies
}

\DeclareAcronym{prism}{
    short       = PRISM,
    long        = personal record information system methodology
}

\DeclareAcronym{ror}{
    short       = RoR,
    long        = Ruby on Rails
}

\DeclareAcronym{smis}{
    short           = SMIS,
    short-plural    = 's,
    long            = social media information system
}

\DeclareAcronym{sn}{
    short       = SN,
    long        = social network
}

\DeclareAcronym{sns}{
    short           = SNS,
    short-plural    = 's,
    long            = social network system
}

\DeclareAcronym{spi}{
	short        = SPI,
	long         = sensitive personal information
}

\DeclareAcronym{ssf}{
    short       = SSF,
    long        = Stiftelsen f{\"o}r Strategisk Forskning,
    foreign     = Swedish Foundation for Strategic Research,
    foreign-lang = english
}

\DeclareAcronym{taler}{
    short        = Taler,
    long         = taxable anonymous libre electronic reserves
}

\DeclareAcronym{tor}{
    short        = Tor,
    long         = the onion router
}

\DeclareAcronym{tos}{
    short           = TOS,
    short-plural    = 's,
    long            = terms of service
}

\DeclareAcronym{ttp}{
    short       = TTP,
    long        = trusted third party
}

\DeclareAcronym{udm}{
    short       = UDM,
    long        = user data manifesto
}

\DeclareAcronym{un}{
    short       = UN,
    long        = united nations
}

\DeclareAcronym{url}{
    short       = URL,
    long        = unified resource locator
}

\DeclareAcronym{usa}{
    short       = USA,
    long        = United States of America,
    alt         = US
}

\DeclareAcronym{vr}{
    short       = VR,
    long        = Vetenskapsr{\aa}det,
    foreign     = Swedish Research Council,
    foreign-lang = english
}

\DeclareAcronym{wis}{
    short           = WIS,
    short-plural    = 's,
    long            = web information system
}

\DeclareAcronym{www}{
    short       = WWW,
    long        = world wide web
}

\DeclareAcronym{ycab}{
	short        = YCab,
	long         = YCab
}

\DeclareAcronym{zkp}{
	short        = ZKP,
	long         = zero knowledge proof
}

\DeclareAcronym{zkpp}{
	short        = ZKPP,
	long         = zero-knowledge password proof
}

%%% EI %%%
\DeclareAcronym{cr}{
	short        = CR,
	long         = commitment reliability
}

\DeclareAcronym{iip}{
	short        = IIP,
	long         = invitee identity privacy
}

\DeclareAcronym{icp}{
	short        = ICP,
	long         = invitee count privacy
}

\DeclareAcronym{aip}{
	short        = AIP,
	long         = attendee identity privacy
}

\DeclareAcronym{acp}{
	short        = ACP,
	long         = attendee count privacy
}

\DeclareAcronym{air}{
	short        = AIR,
	long         = attendee-only information reliability
}

\DeclareAcronym{cdp}{
	short        = CDP,
	long         = commit-disclose protocol
}

%%% DSS %%%
\DeclareAcronym{adss}{
    short        = ADSS,
    long         = anonymous document submission system
}

\DeclareAcronym{adsgs}{
    short        = ADSGS,
    long         = anonymous document submission and grading system
}




% *** adds a space unless the macro is followed by certain punctuation characters ***
\usepackage{xspace}

% tocloft is already being loaded
%\usepackage{tocloft}
% \renewcommand{\cftchapterfont}{\scshape}
% \renewcommand{\cftsecfont}{\bfseries}
% \renewcommand{\cftfigfont}{Figure }
% \renewcommand{\cfttabfont}{Table }
% \renewcommand{\cftsectionpresnum}{SOMETHING }

%\usepackage{algorithm}

% colored dogear marks
\usepackage[pages=some,contents={},opacity=1.0,scale=1,angle=90]{background}
\usepackage{tikz}
\newcommand*\VerBar[1]{%
	\begin{tikzpicture}
	\fill[#1] (0,0)--(1,0)--(1,1)--(0,1)--cycle;
	\draw[white] (0.5,0.5)node{\rotatebox{-90}{\Huge \textbf \thechapter}};
	\end{tikzpicture}
}
\newcounter{DogearHShift}
\newcommand{\dogear}[1]{%
  \backgroundsetup{position={current page.north east},vshift=0.5cm,%
    hshift=-\theDogearHShift cm,contents={\VerBar{#1}}}%
  \BgThispage%
  \addtocounter{DogearHShift}{1}
}
\newcommand{\dogearRGB}[1]{%
  \definecolor{dogearColor}{RGB}{#1}%
  \dogear{dogearColor}%
}
%usage: \dogear{COLOR}, e.g. \dogear{red}
%   or: \dogearRGB{RED,GREEN,BLUE}, e.g. \dogearRGB{255,0,0}
% ------------------------------------


\usepackage{hyperxmp}
% == load hyperref last ==
%\usepackage[hidelinks, pagebackref=true]{hyperref}
% if you only want the page number to be clickable in tock: linktocpage=true
\usepackage[colorlinks=true, pagebackref=true, linktocpage=true]{hyperref}

\usepackage{cleveref}
\crefname{appsec}{appendix}{appendices}
\crefname{appchp}{appendix}{appendices}
\crefname{artsec}{article}{articles}


% *** customisation for pagebackref option in hyperref package ***
% http://tex.stackexchange.com/questions/38149/removing-double-entries-from-hyperrefs-pagebackref
\renewcommand*{\backreflastsep}{, }
\renewcommand*{\backreftwosep}{, }
\renewcommand*{\backref}[1]{}
\renewcommand*{\backrefalt}[4]{%
  \ifcase #1 %
    No citations.% use \relax if you do not want the "No citations" message
  \or
    (Cited in page #2).%
  \else
    (Cited in pages #2).%
  \fi%
}


%%
%% This is file `kth-bibl.tex',
%% generated with the docstrip utility.
%%
%% The original source files were:
%%
%% kthesis.dtx  (with options: `biblio')
%% 
%% IMPORTANT NOTICE:
%% 
%% For the copyright see the source file.
%% 
%% Any modified versions of this file must be renamed
%% with new filenames distinct from kth-bibl.tex.
%% 
%% For distribution of the original source see the terms
%% for copying and modification in the file kthesis.dtx.
%% 
%% This generated file may be distributed as long as the
%% original source files, as listed above, are part of the
%% same distribution. (The sources need not necessarily be
%% in the same archive or directory.)
\title{Practical Methods for Privacy Preservation}
\subtitle{A centralised and decentralised approach}
\author{Guillermo Rodr{\'i}guez Cano}
\date{maj 2003}
\thesistype{Doctoral Thesis}
\imprint{Stockholm, Sweden 2003}
\examen{teknologie doktorsexamen i datalogi}
\disputationsdatum{torsdagen den 17 maj 2003 klockan 10.00}
\disputationslokal{Kollegiesalen, Administrationsbyggnaden,
  Kungl Tekniska h{\"o}gskolan, Valhallav{\"a}gen~79, Stockholm}
\isbn{ISBN x-xxxx-xxx-x}
\issn{ISSN xxxx-xxxx}
\isrn{ISRN KTH/xxx/xx-{}-yy/nn-{}-SE}
\trita{TRITA xxx yyyy-nn}
\publisher{Universitetsservice US AB}
\address{KTH School of Computer Science and Communication\\
  SE-100 44 Stockholm\\
  SWEDEN}
\kthlogo{kth_svv_comp_science_comm}


\newcommand{\thesisauthor}{\theauthor}                                        % Nombre y apellidos del autor
\newcommand{\thesiscontactemail}{gurc@kth.se}
\newcommand{\thesissupervisor}{TBD}                              % Nombre y apellidos del tutor/a o tutores
\newcommand{\thesissupervisorcover}{\thesissupervisor}                         % Nombre y apellidos del tutor/a o tutores

\newcommand{\thesistitle}{\thetitle}                  % Título del trabajo
\newcommand{\thesistitleshort}{\thesistitle}                                                    % Título del trabajo (para meta-datos PDF)
\newcommand{\thesislanguage}{EN}
\newcommand{\thesiscopyright}{Copyright (C) 2016, \thesisauthor}                                % Copyright
\newcommand{\thesislicense}{Creative Commons (by-nc-sa) 3.0 Sweden}                             % Nombre de la licencia
\newcommand{\thesislicenseurl}{http://creativecommons.org/licenses/by-nc-sa/3.0/se/legalcode}   % Enlace a los términos de la licencia
\newcommand{\thesiskeywords}{thesis, degree, computer science, Uppsala,
                             model-driven development, product lines, feature model,
                             variability, validation, configuration, hyper-graph, traversal}    % Palabras clave (para incrustar en PDF)
\newcommand{\thesiskeywordsabstractes}{desarrollo dirigido por modelos, l{\'i}neas de productos,
                                       modelo de caracter{\'i}sticas, variabilidad,
                                       validaci{\'o}n, configuraci{\'o}n, hiper-grafo,
                                       recorrido}                                               % Palabras clave en castellano (para resumen)
\newcommand{\thesiskeywordsabstracten}{model-driven development, product lines, feature model,
                                       variability, validation, configuration, hyper-graph,
                                       traversal}                                               % Palabras clave en inglés (para abstract)
\endinput
%%
%% End of file `kth-bibl.tex'.


\newtcolorbox{ppBox}[1][]{
    breakable,
    enhanced,
    skin=enhancedmiddle,
    frame hidden,
    interior hidden,
    title=#1,
    arc=0pt,
    outer arc=0pt,
    top=0pt,
    bottom=0pt,
    boxsep=1mm,
    borderline={1.5pt}{0mm}{black},
    fonttitle=\sffamily\footnotesize,
    fontupper=\sffamily\footnotesize,
    fontlower=\sffamily\footnotesize,
    colframe=black,
    colback=white,
    colbacktitle=white,
    coltitle=black,
    fonttitle=\bfseries,
    boxrule=0pt,
    bottomrule=0pt,
    toprule=0pt,
  % leftrule=3pt,
  % rightrule=3pt,
    titlerule=0pt,
    width=8\linewidth/10,
    valign=center,
}

%%%
% CONFIG: Meta-data properties for PDF and XMP
%
% Notes: Load after kth-bibl (as variables used come from there)
%
\hypersetup{
    pdftitle        = {\thesistitle},
    pdfauthor       = {\thesisauthor},
%    pdfauthortitle={},%
    pdfsubject      = {Wannabe thesis},
%    pdfsubject      = {\thesistype~(\degree),~\department,~\university},
    pdfkeywords     = {\thesiskeywords},
    pdfcreator      = {\thesisauthor~(Supervised~by~\thesissupervisor)},
    pdfproducer     = {\thesisauthor~powered~by~\LaTeX},
%    pdfcontactaddress={},
%    pdfcontactcity={Lentini (SR)},
%    pdfcontactpostcode={},
%    pdfcontactcountry={Sweden},
%    pdfcontactphone={},
    pdfcontactemail={\thesiscontactemail},
%    pdfcontacturl={},
%    pdfcaptionwriter={},
    pdflang={\thesislanguage},
%    pdfcopyright    = {\thesiscopyright},
    pdflicenseurl   = {\thesislicenseurl},
    pdfstartview    = FitB,
    colorlinks=true,
    breaklinks,
    linkcolor={blue},
    citecolor={red},
    urlcolor={blue}
}

% *** 
% -*- mode: TeX -*-
% -*- coding: utf-8 -*-

% *** definitions ***
\newcommand{\etal}{\xspace~et\xspace~al.\xspace}
\newcommand{\eg}{e.\xspace~g.,\xspace} % note the trailing comma (recommended by http://grammar.quickanddirtytips.com/ie-eg-oh-my.aspx )
\newcommand{\Eg}{E.\xspace~g.,\xspace}
\newcommand{\ie}{i.\xspace~e.,\xspace}
\newcommand{\Ie}{I.\xspace~e.,\xspace}
\newcommand{\vs}{vs.\xspace}

% *** proper circled letters command ****
\newcommand*\circled[1]{
	\tikz[baseline=(char.base)]{\node[shape=circle,draw,inner sep=1.5pt] (char) {#1};}}

% *** customisation for pagebackref option in hyperref package ***
% http://tex.stackexchange.com/questions/38149/removing-double-entries-from-hyperrefs-pagebackref
\renewcommand*{\backref}[1]{}
\renewcommand*{\backrefalt}[4]{%
  \ifcase #1 %
    No citations.% use \relax if you do not want the "No citations" message
  \or
    (Cited in pp. #2).%
  \else
    (Cited in pp. #2).%
  \fi%
}

%%%
% DOC: Automatized formatting for quotes :) Usually it will be
%      one quote/citation per chaper, and it should be located
%      right after the chapter's title, \chapter{bla bla bla}
%      Usage: \requote{frase}{autor}
%
%      For instance: \requote{To be... or not to be: that is the question}{William Shakespeare}
%
\newcommand{\requote}[2]{
{\fontfamily{\carlitofamily}\selectfont
    \begin{flushright}
        \begin{minipage}[b][][t]{0.45\paperwidth}
            \small{
                \itshape{\textbf{\normalsize ``}#1\textbf{\normalsize ''}}
                \begin{flushright}
                    \textbf{\textsc{#2}}
                \end{flushright}
            }
        \end{minipage}
        \bigskip
    \end{flushright}
}}

% ***
% -*- mode: TeX -*-
% -*- coding: utf-8 -*-

\newcommand{\Internet}{{\itshape Internet}\xspace}

\newcommand{\Apple}{{\itshape Apple}\xspace}
\newcommand{\AppletInc}{{\itshape Apple Inc.}\xspace}

\newcommand{\Microsoft}{{\itshape Microsoft}\xspace}
\newcommand{\MicrosoftInc}{{\itshape Microsoft Inc.}\xspace}

\newcommand{\Google}{{\itshape Google}\xspace}
\newcommand{\GoogleInc}{{\itshape Google Inc.}\xspace}
\newcommand{\GooglePlus}{{\itshape Google+}\xspace}
\newcommand{\GoogleCircles}{{\itshape Google Circles}\xspace}

\newcommand{\YouTube}{{\itshape YouTube}\xspace}

\newcommand{\Orkut}{{\itshape Orkut}\xspace}

\newcommand{\Facebook}{{\itshape Facebook}\xspace}
\newcommand{\FacebookInc}{{\itshape Facebook Inc.}\xspace}

\newcommand{\Twitter}{{\itshape Twitter}\xspace}
\newcommand{\TwitterInc}{{\itshape Twitter Inc.}\xspace}

\newcommand{\Flickr}{{\itshape Flick.r}\xspace}
\newcommand{\Yahoo}{{\itshape Yahoo! Inc.}\xspace}

\newcommand{\LinkedIn}{{\itshape LinkedIn}\xspace}

\newcommand{\LiveJournal}{{\itshape LiveJournal}\xspace}

\newcommand{\Friendster}{{\itshape Friendster}\xspace}

\newcommand{\Cyworld}{{\itshape Cyworld}\xspace}

\newcommand{\Sybil}{{\itshape Sybil}\xspace}

\newcommand{\Wayback}{{\itshape Wayback}\xspace}

\newcommand{\InternetArchive}{{\itshape Internet Archive}\xspace}


\makechapterstyle{myell}{%
  \chapterstyle{veelo}
  \renewcommand*{\chapnumfont}{\normalfont\huge\bfseries}
  \renewcommand*{\chaptitlefont}{\normalfont\Huge\bfseries}
  \settowidth{\chapindent}{\chapnumfont 111}
  \renewcommand*{\chapterheadstart}{\begingroup
    \vspace*{\beforechapskip}%
    \begin{adjustwidth}{}{-\chapindent}%
    \hrulefill
    \smash{\rule{0.4pt}{15mm}}
    \end{adjustwidth}\endgroup}
  \renewcommand*{\printchaptername}{}
  \renewcommand*{\chapternamenum}{}
  \renewcommand*{\printchapternum}{%
    \begin{adjustwidth}{}{-\chapindent}
    \hfill
    \raisebox{10mm}[0pt][0pt]{\chapnumfont \thechapter}%
                              \hspace*{1em}
    \end{adjustwidth}\vspace*{-3.0\onelineskip}}
  \renewcommand*{\printchaptertitle}[1]{%
    \vskip\onelineskip
    \raggedleft {\chaptitlefont ##1}\par\nobreak}}

% http://tex.stackexchange.com/questions/254318/chapter-style-with-tcolorbox/254328
% http://tex.stackexchange.com/questions/301185/modify-fancy-chapter-heading-to-show-chapter-name
\definecolor{titlebgdark}{RGB}{0,163,243}
\definecolor{titlebglight}{RGB}{191,233,251}
\titleformat{\chapter}[display]
  {\normalfont\huge\bfseries}
  {}
  {20pt}
  {%
    \begin{tcolorbox}[
      enhanced,
      colback=titlebgdark,
      boxrule=0.25cm,
      colframe=titlebglight,
      arc=0pt,
      outer arc=0pt,
      leftrule=0pt,
      rightrule=0pt,
      fontupper=\color{white}\sffamily\bfseries\huge,
      enlarge left by=-1in-\hoffset-\oddsidemargin, 
      enlarge right by=-\paperwidth+1in+\hoffset+\oddsidemargin+\textwidth,
      width=\paperwidth, 
      left=1in+\hoffset+\oddsidemargin, 
      right=\paperwidth-1in-\hoffset-\oddsidemargin-\textwidth,
      top=0.6cm, 
      bottom=0.6cm,
      overlay={
        \node[
          fill=titlebgdark,
          draw=titlebglight,
          line width=0.15cm,
          inner sep=0pt,
          text width=1.7cm,
          minimum height=1.7cm,
          align=center,
          font=\color{white}\sffamily\bfseries\fontsize{30}{36}\selectfont
        ] 
        (chapname)
        at ([xshift=-1in]frame.north east)
        {\thechapter};
        \node[font=\small,anchor=south,inner sep=2pt] at (chapname.north)
        {\MakeUppercase\chaptertitlename};  
      } 
    ]
    #1
    \end{tcolorbox}%
  }
\titleformat{name=\chapter,numberless}[display]
  {\normalfont\huge\bfseries}
  {}
  {20pt}
  {%
    \begin{tcolorbox}[
      enhanced,
      colback=titlebgdark,
      boxrule=0.25cm,
      colframe=titlebglight,
      arc=0pt,
      outer arc=0pt,
      remember as=title,
      leftrule=0pt,
      rightrule=0pt,
      fontupper=\color{white}\sffamily\bfseries\huge,
      enlarge left by=-1in-\hoffset-\oddsidemargin, 
      enlarge right by=-\paperwidth+1in+\hoffset+\oddsidemargin+\textwidth,
      width=\paperwidth, 
      left=1in+\hoffset+\oddsidemargin, 
      right=\paperwidth-1in-\hoffset-\oddsidemargin-\textwidth,
      top=0.6cm, 
      bottom=0.6cm, 
    ]
    #1
    \end{tcolorbox}%
  }
\titlespacing*{\chapter}
  {0pt}{0pt}{40pt}
\makeatother

%\chapterstyle{myell}

% http://tex.stackexchange.com/questions/159551/customizing-part-style-with-tikz
\definecolor{mybluei}{RGB}{0,173,239}
\definecolor{myblueii}{RGB}{63,200,244}
\definecolor{myblueiii}{RGB}{199,234,253}
\renewcommand\partnumfont{% font specification for the number
  \fontsize{380}{130}\color{myblueii}\selectfont%
}

\renewcommand\partnamefont{% font specification for the name "PART"
  \normalfont\color{white}\scshape\small\bfseries 
}

\titleformat{\part}
  {\normalfont\huge\filleft}
  {}
  {20pt}
  {\begin{tikzpicture}[remember picture,overlay]
  \fill[myblueiii] 
    (current page.north west) rectangle ([yshift=-13cm]current page.north east);   
  \node[
      fill=mybluei,
      text width=2\paperwidth,
      rounded corners=6cm,
      text depth=18cm,
      anchor=center,
      inner sep=0pt] at (current page.north east) (parttop)
    {\thepart};%
  \node[
      anchor=south east,
      inner sep=0pt,
      outer sep=0pt] (partnum) at ([xshift=-20pt]parttop.south) 
    {\partnumfont\thepart};
  \node[
      anchor=south,
      inner sep=0pt] (partname) at ([yshift=2pt]partnum.south)   
  {\partnamefont PART};
  \node[
      anchor=north east,
      align=right,
      inner xsep=0pt] at ([yshift=-0.5cm]partname.east|-partnum.south) 
  {\parbox{.7\textwidth}{\raggedleft#1}};
  \end{tikzpicture}%
  }
  

\makeatletter
\newcommand*{\greek}[1]{%
   \expandafter\@greek\csname c@#1\endcsname
}
\newcommand*{\@greek}[1]{%
   $\ifcase#1\or\alpha\or\beta\or\gamma\or\delta\or\varepsilon
     \or\zeta\or\eta\or\theta\or\iota\or\kappa\or\lambda
     \or\mu\or\nu\or\xi\or o\or\pi\or\varrho\or\sigma
     \or\tau\or\upsilon\or\phi\or\chi\or\psi\or\omega
     \else\@ctrerr\fi$%
}
\makeatother

\newcommand\firstToLow[1]{%
 {%
   \renewcommand{\mfirstucMakeUppercase}{\MakeLowercase}%
   \makefirstuc{#1}%
 }%
}

\makeatletter
\newcommand*{\lcnameref}[1]{%
 \begingroup 
   \let\label\@gobble
   \NR@setref{#1}\lc@thirdoffive{#1}%
  \endgroup
}
\newcommand{\lc@thirdoffive}[5]{\firstToLow{#3}}
\makeatother

\counterwithin*{chapter}{part}



\begin{document}

\frontmatter
% \maketitle
% \input{template/doc/latex/kthesis/kth-abs}
% \clearpage
% \selectlanguage{swedish}
% \begin{abstract}
%   Denna fil ger ett avhandlingsskelett.
%   Mer information om \LaTeX-mallen finns i
%   dokumentationen till paketet.
% \end{abstract}
% \selectlanguage{spanish}
% \begin{abstract}
%  Me gusta mucho esto
% \end{abstract}
 \selectlanguage{english}
\begin{abstract}
    \label{abstract}
    % What is the problem?
    % Why is it a problem?
    % Why should we care?
    % What are our findings?
    %
    Privacy-enhancing technologies may help lessen the threats to the privacy of 
    users' personal information in centralised systems like \aclp*{osn}. As such, 
    decentralised solutions have been proposed to extend user's control over its 
    data.
    
    current solutions are made of building blocks, we show some of them in the categories
    of decentralized systems in general, particular systems like osns and more generic
    privacy-preserving systems.
    
    Decentralization: P2P
    
    Particular case of decentralization (\acp{dosn}): EI
    
    Privacy-Preserving systems: DSS
\end{abstract}
\clearpage
% \chapter*{Acknowledgements}
% I acknowledge the acknowledgement of having written this :)
%
% \section*{Funding}
% The research leading to this thesis has been supported by the following projects:
% \begin{itemize}
%     \item Protection of personal information in social networks (Skydd av personlig information f{\"o}r sociala n{\"a}tverk)\\
%     Funded by the \ac{ssf} grant: SSF FFL09-0086.\\
%     \url{http://stratresearch.se/en/research/ongoing-research/framtidens-forskningsledare-4/project/4048/}
%     \item Privacy-preserving social and community networks\\
%     Funded by the \ac{vr} grant: VR 2009-3793.\\
%     \url{http://vrproj.vr.se/detail.asp?arendeid=69587}
% \end{itemize}
%
% We are grateful to the Swedish taxpayers for their commitment to research and development.
%
\clearpage
\addtocontents{toc}{~\hfill\textbf{Page}\par}
\tableofcontents

\mainmatter

\part*{}

\chapter{Introduction}
    \label{chapter:introduction}

\requote{Intellectual beauty is sufficient unto itself, and only for it rather than 
for the future good of humanity does the scholar condemn himself to arduous and 
painful labors}{Santiago Ram{\'o}n y Cajal}
    
\lettrine{\textcolor[gray]{.25}{I}}{n} 1989, while at the European Particle Physics 
Laboratory, at the \acf{cern}, Tim Berners-Lee invented the \Ac{www}. No one at that 
time would have imagined that such invention has been centric to the rapid development 
of the digital age that we are currently witnessing.

Such period in human history, mainly characterised by an economy based on information 
processing as opposed to the traditional industry of the industrial revolution, 
has widen the availability of uncountable services to the general population with 
access to the Internet. Not only new business opportunities have stemmed as technological 
developments matured but also services have been overhauled to make them available 
in the digital world.

To give an example, interpersonal communication --- the passing of information between 
entities --- has evolved throughout history as communication advances developed. 
The traditional exchange of paper letters as a means of communication between two 
parties, which still prevails today, has evolved into a much more sophisticated 
and complex form in the information age: electronic mail (e-mail), which is simply 
the parallelism in the digital era to such physical letters.

Such evolution in the means of communication among individuals and other entities 
has also happened at other societal levels. For example, the social structures that 
individuals form when sharing similar interests or activities, namely social networks, 
have also seen a transposition in the digital era. The \Internet and the \ac{www} 
have taken the leading role as tools to convey different types of information from 
one party to another, expanding the boundaries of the concept of social networks, 
heavily based on tangible activity between individuals and entities.

While the concept of \acl{sn} is a theoretical term mainly used in the realm of 
social sciences to study and describe the social structures determined by the relationships 
between individuals, groups, and other types of entities such as societies; the 
technological developments of the digital era have popularised its virtual counterpart 
term: \aclp{osn}.

\Acp{osn} are \Internet based applications where the audience --- individuals and 
groups --- create profiles of themselves, connect with others by creating relationships, 
generate heterogeneous content and, share and exchange such content among the connections 
and other users in the network. These networks are, up to a certain extent, a complement 
to the established interactions that individuals already have in their daily lives 
\cite{SubrahmanyamRWE08}. Convenience of building and maintaining relationships 
and enjoyment are among the main motivations driving individuals to disclose personal 
information in such platforms \cite{KrasnovaSKH10}.

These \acp{is} started becoming popular in the second half of the 90s 
with services such as \LiveJournal or \Friendster evolving\footnote{For a comprehensive 
but concise survey on the history and evolution of these networks the reader is 
advised to read \cite{boydE07} and \cite{HeidemannKP12}} to the most popular ones 
nowadays such as \Facebook, \LinkedIn, \Twitter or \GooglePlus.

The building design describing such systems is very similar among them. Their design 
is usually monolithic with a data model that is centralised\footnote{With centralised 
we mean that the control of the data is under the jurisdiction of one entity, usually 
the \ac{osn} service provider, regardless of how such data is stored physically, 
for example, in a distributed or centralised manner.} and a system model consisting 
of some variant of cloud computing architectures \cite{PallisZD11}. 

The main business model is also similar among most \acp{osn} providers. Instead 
of charging the users for the services of the \ac{osn}, the revenue model is based 
on selling advertisements to businesses and companies that want to reach a subset 
of the users in the platform. In fact, some argue that \acp{osn} have become an 
ecosystem for marketing rather than for information \cite{HannaRC11}, and ignoring 
these social media platforms may be the difference between prosperity or failure 
\cite{HarrisR09}.

Such business revenue model based on advertising requires in most cases accepting 
some \ac{tos} and \acp{pp} where the users of the \ac{osn} give up certain rights 
on the data contained in the ``free'' profiles and any other data stemming from 
the interactions among them. For example, \Facebook states in its \ac{tos} that 
the user grants a ``non-exclusive, transferable, sub-licensable, royalty-free and 
worldwide license'' of any intelectual property content posted in the \ac{osn}, 
\LinkedIn and \Twitter state similar \ac{tos} as \Facebook~--- see \cref{chapter:excerpts-of-tos-and-pp} 
for some excerpts of the \ac{tos}.

Users are also more aware of the profitability that \ac{osn} service providers do 
of their content via advertising. Moreover, the concern of users' data privacy in 
\acp{osn} has also increased in recent years particularly when it comes to configuring 
who can access their content. Some studies show that users' engagement with the 
privacy settings is more frequent nowadays and even in line with the use of the 
\ac{osn} \cite{boydH10}.

However, raising awareness among \ac{osn} users of the privacy settings or how they 
can better use the service to protect their privacy or, requesting more transparent 
policies from the \ac{osn} service provider is necessary but still not sufficient. 
Because an important aspect of personal data privacy is actually maintaining control 
over such information --- informational privacy \cite{Cavoukian96}.

Privacy in \acp{osn} has become a topic of ongoing discussions in the current century 
due to the business model of the service providers of \acp{osn}. Nevertheless, the 
focus of the discussions on privacy in \acp{osn} has shifted towards the idea of 
data control --- privacy was already envisioned as data control in the late 90s 
\cite{Allen99}.

\acp{osn} have been on the news for different unfortunate reasons related to privacy, 
to name a few, coercion into self-identification\footnote{See ``Facebook Locks Out Thousands, Now Wants Photo ID'' at \url{https://web.archive.org/web/20150921075721/http://www.conspiracyclub.co/2015/03/25/facebook-asking-gov-id-to-verify-account/}}, 
user data leakages\footnote{See ``Passwords for 32M Twitter accounts may have been hacked and leaked'' at \url{https://web.archive.org/web/20161102165137/https://techcrunch.com/2016/06/08/twitter-hack/}}, 
censorship\footnote{See ``LinkedIn Considers Changes After China Censorship Revealed'' at \url{https://web.archive.org/web/20150927040931/http://blogs.wsj.com/digits/2014/09/03/linkedin-considers-changes-after-china-censorship-exposed/}}, 
intentional misuse\footnote{See ``WhatsApp data sharing with Facebook must be stopped until it can be proved legal, European Union warns'' at \url{https://web.archive.org/web/20161029115621/http://www.independent.co.uk/life-style/gadgets-and-tech/news/whatsapp-data-sharing-facebook-eu-european-union-privacy-safety-opt-out-a7384586.html}} 
and even veto the right to be forgotten\footnote{See ``European Court Lets Users Erase Records on Web'' at \url{https://web.archive.org/web/20161013113612/http://www.nytimes.com/2014/05/14/technology/google-should-erase-web-links-to-some-personal-data-europes-highest-court-says.html}}

Moreover, \acp{osn} service providers, among other major technological companies, 
have been known to be in the thick of a global surveillance program --- a program 
known as \Ac{prism}\footnote{See ``NSA Prism program taps in to user data of Apple, Google and others'' at \url{https://web.archive.org/web/20161112033822/https://www.theguardian.com/world/2013/jun/06/us-tech-giants-nsa-data}} --- 
after the revelations of a former contractor's employee of the \ac{usa} government 
in 2013. For example, not only \Facebook and \Google, had full knowledge of having 
their users' privacy breached without their knowledge and consent but also assisted 
and collaborated willingly in such violation.

The aforementioned scenario certainly pictures a quite pessimistic and probable 
future as it seems that privacy is pretty much extinct --- if it was somewhat possible 
in the past. Even though there are other plausible scenarios that can address the 
use of \acp{osn} in a privacy-preserving manner. A trivial one would be simply not 
using any technology that can pose a threat to privacy but this is impractical in 
the current digital era and irrational. Another solution could be some sort of institutional 
transparent control of service providers but this would allow for collusions between 
service providers and institutions as proven with the \ac{prism} program.

In fact, there is not a perfect solution that addresses the problem. However, the 
idea of privacy as data control is feasible enough to consider as a starting point. 
Because it should be the user the one who decides what happens with her data, how 
it is dealt with and by whom. There is already technology aiding the users keep 
that control while allowing business models to operate without compromising users' 
privacy. For example, \Apple recently unveiled how it is embracing differential 
privacy to collect statistics of its users anonymously\footnote{See ``Apple's `Differential Privacy' Is About Collecting Your Data --- But Not Your Data'' at \url{https://web.archive.org/web/20160901192334/https://www.wired.com/2016/06/apples-differential-privacy-collecting-data/}}.

% DSS BEGIN
An example of a privacy-preserving solution in an \ac{is} is part of 
this thesis in \cref{article:document-submission-system}. In the context of an academic 
institution where personal information about individuals is collected --- for example, 
students' examination documents --- for further processing by a central authority 
--- such as a grading examiner --- we provide a privacy-preserving protocol based 
on blind signatures for anonymous document submission and grading --- the process 
of anonymous grading is popularly known as blind grading in fact. 

Although our solution provides with a procedure to do blind grading in a privacy-preserving 
manner guaranteeing the correctness of the grading process, it relies on a central 
authority to process the data --- for example, to issue and link the credentials with 
the identities of the participants. Such authority --- an academic institution --- 
is a trusted party, but this cannot always be a reasonable assumption in other scenarios 
--- recall the governmental program \ac{prism}.
% DSS END

For these reasons, decentralisation\footnote{In the context of this thesis we adopt 
the definition by Rohit Khare where a decentralised system is ``one which requires 
multiple parties to make their own independent decisions''.} has been proposed in 
some centralised \acp{is} such as \ac{osn} --- namely \acp{dosn} --- 
as one of the foremost steps to minimise the impact on users' privacy by the central 
service provider. Yet decentralisation in \acp{is} does not come for 
free nor it is trivial to achieve because new challenges arise when there is no 
central provider to rely on for basic services such as storage, communication or 
identification.

The main idea in decentralised systems, such as \acp{dosn}, is to provide the services 
of the centralised counterpart without relying on the central authority which was 
assumed to be trusted --- fully or partially --- and making decisions on behalf 
of everyone in the system. This can be done by fully opening the system in such 
a way that every party --- known as peer --- is autonomous, hence interacting directly 
with other parties, for example, sharing information, and, entering or leaving the 
system freely --- anyone can do so at any time. A typical example of an information 
system that is completely decentralised is a \ac{p2p} network.

Such arbitrariness comes with trade-offs because the parties may have conflicting 
interests and goals making the decentralised system susceptible to additional attacks 
from malicious parties. Every party has now some responsibility because there is 
not a central authority where any party can turn to in case of misbehaviour. For 
example, impersonation, denial of service, \Sybil attacks, collusion or deception 
are some of the security threats that a decentralised scenario has to face besides 
other technical ones such as lack of cooperation, network bootstraping or, who and 
how to trust \cite{BucheggerA09}.

Consequently, decentralisation shifts the primary focus of the \ac{is} 
to privacy, security and data portability --- and usable functionality thereafter. 
As the power of the central authority is overruled and distributed among the parties 
in the decentralised system new threats arise in the decentralised context. Without 
any privacy-preserving approach addressing these challenges, the personal data, 
that is now unlocked from the centralised ``silo'', is at the hand of everyone in 
a virtually ungoverned environment.

% P2P BEGIN
Privacy-preserving decentralised \acp{is} is the other focus of this 
thesis. On one side, we show how the classical functionality of user authentication 
can be implemented in a decentralised manner in \cref{article:passwords-peer-to-peer} 
for an \ac{is} in general --- in fact, for a fully decentralised system 
such as a \ac{p2p} network. We supplement the basic login functionality with other 
features such as enabling a user to change her password or even recover a forgotten 
one via her e-mail or some security questions or, store her login credentials on 
some other device --- in a secure form --- for example, a mobile phone, so that 
she does not have to log in every time she wants to use it.

Scalable and usable privacy-preserving user authentication in decentralised systems 
allows for a wider adoption and transition to decentralised \acp{is}. 
For example, the feature of recovering the password via e-mail is a popular solution 
used by many vendors in centralised systems --- including \acp{osn} service providers 
\cite{Kuzma11}. However, such recovery functionality fails its primary function 
when the service provider is not available, or if it becomes malicious and refuses 
to use the e-mail originally provided by the user. 

In our solution we propose a $(n, k)$-secret sharing approach to overcome the absence 
of a central authority. The user chooses $n$ peers who can help her on the recovery 
task and among those, $k$ peers that should be available when recovering the password. 
The user is in control of both parameters --- $n$ and $k$ ---  at all times, unlike 
in the centralised scenario.
% P2P END

% EI BEGIN
On the other side, we show how the basic functionality of organising an event, usually 
available in centralised \acp{osn}, can be implemented in a privacy-preserving manner 
in \acp{dosn} in \cref{article:events-invitations-dosns}. Our proposed solution 
allows a user to create events and invite other users who can confirm their attendance. 
While this functionality is rather simple when there is a trusted third party, such as 
the service provider in an \ac{osn}, implementing such feature without a trusted 
broker that guarantees fairness in the process is not trivial. For example, keeping 
track of who is invited or showing the total number of attendess to those invitees 
who are attending.

Our mechanism uses standard cryptographic primitives allowing other features such 
as revealing some event specific information to a group of users --- to those who 
commit to attend an event --- or even detect any misbehaving party --- for example, 
an invitee can prove that accepted an invitation and did not get further details 
about the event from a malicious organiser.

With our solution we aim at a broader spectrum of privacy-preserving applications 
outside the context of \acp{dosn}. The cooperation and coordination characteristics 
of our specific use case and our secure decentralised solution suggest for applications 
in other domains where there is some collaboration involved between the parties 
and a trusted third party is not ideal or possible at all.
% EI END

Finally, we acknowledge the urgent need for more privacy in \acp{is}. 
In this thesis, we take the approach of decentralising such systems as a way to 
achieve the ideal of privacy as data control as envisioned by Allen \cite{Allen99}. 
We contribute with some insights on how to enhance the control of the user on her 
data in general centralised and decentralised systems, and in particular ones such 
as \acp{dosn}. We also suggest usable privacy-preserving protocols for some specific 
features and show their utility in some concrete scenarios.

% % Social networks
%  is an inherently interdisciplinary academic field which emerged from social psychology, sociology, statistics, and graph theory.
%
%
%
% Social networks, as a bla bla bla, and in particular, online social networks, as
% their equivalent in the digital world (but also as a complement to the tangible
% one) are super awesome and
%
% % Next paragraph is wrong
%
% Such process of information dissemination is one of the basis when forming relations
% among individuals and other entities --- social structures --- resulting in a set
% of social relations commonly known as social networks.
%
%
% people to build social networks or social relations with other people who share similar personal or career interests,[1] activities, backgrounds or real-life connections. The variety of stand-alone and built-in

\section{Motivation}
    \label{section:motivation}
Our rationale for the line of work of this thesis is data control in the broad sense 
and in line with the notion of ``privacy as data control'' in \cite{Allen99}. We 
drive our reasoning on the three following core user personal data rights\footnote{The 
reader is advised to read more on the three main rights we have in mind in the \Ac{udm} 
at \url{https://web.archive.org/web/20161010132230/https://userdatamanifesto.org/}.}: 
``control over user data access'', ``knowledge of how data is stored'' and which 
laws or jurisdictions apply and, ``freedom to choose a platform'' without experiencing 
any vendor lock-in.

% The rights stated in the \ac{udm} are, as a matter of fact, a modern version of the
% three notions of `privacy' that \citeauthor{Allen99} stated in \cite{Allen99}: ``personal
% data control or rights of data control'', ``right of personal data control'' and,
% ``enhancing personal data control by individuals is the optimal end of privacy regulation''.

Data control —-- in the essence of data privacy and information security —-- is a 
subject that has gained a lot of attention in the last decade. Particularly from 
major technological corporations such as \Google, \Apple or \Facebook because they 
have been identified as necessary collaborators of institutional surveillance of 
citizens world-wide \cite{Lyon14}.

Although the users of \acp{osn} owned by some of these major corporations are much more 
aware than ever of such collaboration of service providers and governmental institutions 
for mass surveillance \cite{Madden14}. Moreover, users are more willing than before 
to do something to protect their privacy despite the fact that different types of 
information prompt different degrees of sensitivity, for example, social security 
numbers are considered much more sensitive than some habits such as purchasing or 
browsing the \ac{www}.

A well articulated security and privacy strategy is necessary but it is not sufficient 
for a corporation (or governmental institution) to protect the data of its customers 
and users (or citizens) because the data itself is still out of the control of the 
user. Not only the user does not possess such data physically but also does not 
control who access it or even if, and how, gets transferred to third parties --- 
in many cases, unauthorised.

Therefore, we aim at improving users' data control as much as it is technologically 
possible, leaving legal policies such as \ac{tos} or \ac{pp} and, the ethical debate 
out of the scope of this thesis. We believe that removing the central provider in 
a centralised \ac{is} such as an \ac{osn} is one of the very first steps towards 
returning that control of the data back to the user. Although the trade offs of 
getting rid off such entity, for example, who guarantees that two parties can communicate 
between each other without a third one overseeing, ought to be identified, analysed, 
researched and worked out.

Finally, we also have the desire to reduce the impact of personal data privacy in 
people's lives by raising awareness on the topic with our solutions. We want to 
broaden and deepen the ongoing political debate and legal discussions on the topic 
of privacy not only in \acp{osn} but also in many other centralised \acp{is}. However, 
we reckon there is still a long road ahead on the quest of privacy and this is barely 
the beginning.

\section{Outline}
    \label{section:outline}
After putting the reader in the context of \acp{osn} as \acp{is} and the need for 
privacy-preserving approaches by means of decentralisation of such \Internet based 
applications --- \acp{dosn} --- in the current \namecref{chapter:introduction}, 
we continue with a background introduction to the main concepts that we base our 
work on in \cref{chapter:background}. We provide a brief overview of concepts within 
\acp{is} in \cref{section:information-systems-on-the-importance-of-capabilities}, 
centralised and decentralised \acp{osn} in \cref{section:osns-centralisation-vs-decentralisation} 
and privacy in \cref{section:privacy-a-never-ending-battle}.

We continue in \cref{chapter:our-research-problem} with a description of our research 
problem and the challenges we faced to detail our research methodology 
in \cref{section:methodoloy}. Our contributions are listed in \cref{chapter:our-contributions} 
along with a summary of each article for the reader's convenience. We complete our 
thesis with our conclusions and future directions in \cref{chapter:conclusions} 
and \cref{section:future-directions} respectively. 

For completeness, we include a re-print of each publication that we refer in \cref{chapter:our-contributions} 
in \cref{part:published-articles}: \usebibentry{KreitzBGRB12}{title}, \usebibentry{RodriguezCanoG14}{title} 
and \usebibentry{GreschbachREB15}{title} (\cref{article:passwords-peer-to-peer,article:events-invitations-dosns,,article:document-submission-system} 
respectively). Moreover, and for the reader's reference, we have also included selected 
excerpts of \ac{tos} from three popular \ac{osn} service providers in \cref{chapter:excerpts-of-tos-and-pp}.


\section{Conventions}
    \label{section:conventions}
In this thesis we are using the following conventions in the language used,
\begin{itemize}
    \item The gender used throughout the text while referring to a human being using 
    a computer --- user --- may unintentionally differ, although we tried to be 
    consistent with the gender `she'. However, this thesis is neutral towards gender 
    but to ease the reading we chose such gender --- except in the abstract.

    \item Data and information can be considered synonyms in the broader sense --- 
    the definition of data is a piece of information. In the context of this thesis 
    we will use the latter --- information --- as the abstract fact about something, 
    while the former --- data --- as the encoded form of such information that can 
    be transferred over a network and interpreted by a computer.
    
    \item All the \acp{url} in this thesis --- except the ones in \cref{chapter:excerpts-of-tos-and-pp} ---
    are provided via the \Wayback machine, a digital archive of the \ac{www} and 
    other information on the \Internet created by the \InternetArchive non-profit 
    digital library organisation. The reader has the possibility of accesing a cached 
    snapshot of the version that was used when this thesis was written and also 
    the original \ac{url}. In order to access the original \ac{url}, the reader 
    is advised to remove the timestamped prefix that precedes the original \ac{url} 
    of the form \url{https://web.archive.org/web/YYYYMMDDHHMMSS/}. For example, 
    given the original \ac{url}, \url{https://en.wikipedia.org/wiki/Privacy}, the 
    \Wayback machine created the following snapshot and its \ac{url} as of the time 
    of this writing, \url{https://web.archive.org/web/20161111044417/https://en.wikipedia.org/wiki/Privacy}.
\end{itemize}


\chapter{Background}
    \label{chapter:background}

\requote{Science is always worth because its discoveries, sooner or later, are always 
put to use}{Severo Ochoa de Albornoz}

\lettrine{\textcolor[gray]{.25}{T}}{he} field of \acp{sn} is an extremely vast domain 
of knowledge and research. It has influenced dozens of domains, spanning from the 
most traditional ones such as ethnography or anthropology to more contemporanean 
ones such as genomics or astrophysics.

% TODO: Do I need to include something here to make the transititon from SNs to OSNs? Otherwise, maybe rephrase the next paragraph and start with such transition?

With the explosion of large scale \acp{osn} new challenges have emerged in connection 
with the sudden availability of new types of data. The inherent increase in the 
amount of data poses a technological challenge because there is a need to cope with 
such size. Moreover, there is a need to investigate and develop efficient methods 
that can answer questions in such large space of data within a reasonable amount 
of time.

% http://dictionary.cambridge.org/dictionary/english/be-two-sides-of-the-same-coin
At the same time, the collection of such amount of data in \acp{osn} is having a 
great impact on the privacy of the users who use these networks to share information 
among them. Data and privacy seem to be the two sides of the same coin: \acp{osn} 
\cite{BelkinC92}. On the one side, the retrieval of information --- data --- is 
a necessity when it comes to providing some service, for example, allowing people 
to comment on a picture, and on the other side, the collection of such data opens 
a lot of possibilities on how the data can be used beyond its initial intended purpose 
in other realms, for example, advertising.

As the data is usually stored by a single centralised entity --- the owner of the 
\ac{osn} --- there is also a risk for unintentional wrong-doing of the data without 
the knowledge of the users such as misuse, data leakages or even censorship. Therefore, 
when designing such \acp{is} with social network capabilities \cite{Abrams06, Lunt06, Lunt07, Zhu08, Lunt09}, 
considering privacy is not simply an additional feature to have in the \ac{osn} 
and it does become a requirement. 

Moreover, providing privacy-preserving capabilities is as important as educating 
the users into using the settings built for data control. The risks of exposing 
personal information by not configuring properly the visibility of the data is something 
that is not commonly considered and pose different threats, for example, identity 
theft \cite{GrossAH05, BrandtzaegLS10}. Though the topic of user behaviour with 
privacy-preserving features in \acp{osn} is out of the scope of this thesis.

% TODO: Use cleverref for name of chapter instead of hardcoding background
In the following \namecrefs{section:information-systems-on-the-importance-of-capabilities}
we introduce the reader to the main concepts that this thesis is based on --- however,
the reader should should not consider this chapter a survey because it is not aimed 
at being that and rather an introductory and concise summary of relevant concepts. 
%We provide a brief introduction to \acp{is} in \cref{section:information-systems-on-the-importance-of-capabilities},
We provide a brief introduction to \acp{is} in the following section,
centralised and decentralised \acp{osn} in \cref{section:osns-centralisation-vs-decentralisation}
and privacy in \cref{section:privacy-a-never-ending-battle}.

%\section{\Aclp*{is}: on the importance of capabilities}
\section{Information Systems: on the importance of capabilities}
    \label{section:information-systems-on-the-importance-of-capabilities}

Stair\etal define an \ac{is} as a ``set of interrelated elements or components 
that collect --- input ---, manipulate --- process --- , store, and disseminate 
--- output --- data and information, and provide a corrective reaction --- feedback 
mechanism --- to meet an objective'' \cite{StairR15}. And narrow down the previous 
definition in the domain of \ac{it} with \ac{cbis} defined as a ``single set of 
hardware, software, databases, telecommunications, people, and procedures configured 
to collect, manipulate, store, and process data into information''. 

\Ac{cbis} are usually divided into six parts, which we briefly describe in the context 
of \acp{osn},
\begin{itemize}
    \item Hardware\\
    Possibly the most obvious one in any \ac{cbis}. It can be a single machine or 
    many. In the context of \acp{osn} --- and in the current state of the \Internet 
    --- is common to use server farms in data centers all over the world acting 
    as one and, providing a fast, reliable and redundant service. Users usually 
    communicate with one of the farms that is geographically close. In case of failure 
    of any component in one or more servers of the farm --- or even an entire server 
    farm --- the other machines can take over seamlessly --- or another server farm.
    
    \item Software\\
    A
    
    \item Communications\\
    A
    
    \item Data\\
    A
    
    \item Processes\\
    A
    
    \item Users\\
    A
    
\end{itemize}


%Social Media Information systems - david kroenke :https://online.columbiasouthern.edu/CSU_Content/Courses/Business/BBA/BBA3551/12K/UnitV_Chapter8Presentation.pdf?targ
There are also pther approaches to \acp{osn} in \acp{is} as \acp{smis} or even web 
finromation ssytems
Make a footnote on sticking on CBIS in this thesis to be agnostic of terms and/or mediums (web is fine but now social networks happen on the mobile hence social media or even mobile apps work best)

% https://www.techwalla.com/articles/what-are-the-six-elements-of-an-information-system

% Reference: Social Networks and Information Systems: Ongoing and Future Research Streams
% Short summary: The IS research drawing on social networks can be divided into the following streams: 1) network awareness at both individual and organizational levels, 2) uses of social network analysis related to IS use, and 3) conceptual and technological change in the fast evolving platforms to manage social networks

%\subsection{\Aclp*{is} in this thesis}
\subsection{Information Systems in this thesis}
    \label{subsection:information-system-in-this-thesis}

\url{https://en.wikipedia.org/wiki/Information_system}

A computer(-based) information system is essentially an IS using computer technology to carry out some or all of its planned tasks. The basic components of computer-based information systems are:

Hardware- these are the devices like the monitor, processor, printer and keyboard, all of which work together to accept, process, show data and information.
Software- are the programs that allow the hardware to process the data.
Databases- are the gathering of associated files or tables containing related data.
Networks- are a connecting system that allows diverse computers to distribute resources.
Procedures- are the commands for combining the components above to process information and produce the preferred output.

Information systems theoretical foundations: \url{http://www.sciencedirect.com/science/article/pii/0306437981900235}

Managing trust in a peer-2-peer information system: \url{http://dl.acm.org/citation.cfm?id=502638}


%\section{\Aclp*{osn}: centralisation \vs decentralisation}
\section{Online Social Networks: centralisation \vs decentralisation}
    \label{section:osns-centralisation-vs-decentralisation}
While there is a wide understanding among researchers and practitioners on what 
a \ac{sn} is and, by extension, an \ac{osn}, there is not a canonical definition 
of neither of them in the literature. Authors seem to define their own ``interpretation'' 
as they see fit to their particular problem to solve or topic to discuss without 
much general consensus --- see, for example, \cite{AdamicA05}, \cite{DwyerHP07}, \cite{SchneiderFKW09} 
and \cite{RichterRB11}.

In our work, we have opted for the definition by boyd\etal because we believe that 
it is the most representative for recent works in the field of \acp{osn} --- an 
\ac{osn} site is a ``web-based service that allows individuals to construct a public 
or semi-public profile within a bounded system, articulate a list of other users 
with whom they share a connection, and view and traverse their list of connections 
and those made by others within the system'' --- \cite{boydE07}. Although this definition 
has been critiqued for being too broad as well \cite{Beer08}.

% https://web.archive.org/web/20161121154722/http://www.bbc.co.uk/programmes/p02swnrx
% https://web.archive.org/web/20131215060028/http://downloads.bbc.co.uk/podcasts/fivelive/pods/pods_20131112-0401a.mp3
The two most important parts of the previous definition are the means of communication 
--- the \ac{www} --- and the administration of the users' profiles and their connections 
in a system. This system is what we need to clarify in fact because the \ac{www} 
is, by construction, an interconnected set of systems thus inherently\footnote{
\Internet traffic nowadays is increasingly concentrating through specific nodes, 
owned by major technological corporations such as \Amazon or \Google, offering cloud 
services and fostering a centralisation of the data in the \ac{www} which is seen 
as a dangerous concentration of `power' by some activists \cite{Bolychevsky13}} 
distributed.

Popular centralised \acp{osn} such as \Facebook, \Twitter or \GooglePlus take different 
approaches in the way they have designed their users' profiles and the mechanisms 
for which users share and traverse the connections among them and, exchange information. 
However, most of the data is still stored, processed, administered and managed by 
the owner of the \ac{osn}\footnote{The owner may use a distributed architecture 
to provide the service while still keep full control of all the nodes in such setup.}.

Traditional \acp{osn}, as \acp{is}, are controlled by a single authority --- the 
service provider--- regardless of the chosen infrastructure of the \ac{cbis} --- 
for example, in a distributed manner --- to provide the service to the users. There 
are many reasons for which many \acp{osn} designs are centralised with the most 
prominent ones lying on the side of data control\footnote{Do not confuse data control 
with data ownership. The latter is still a right that the user who uploaded the content 
to the \ac{osn} usually keeps. See \cref{chapter:excerpts-of-tos-and-pp} for some 
examples of \acp{tos} with the legal wording on this matter.} towards business value 
by means of data mining \cite{DomingosR01}.

However, such centralisation by a single authority poses many threats to the owner 
of the data --- the user --- because she does not know what happens with her data 
after such data is transferred into the \ac{osn}. Even if the specific uses and 
purposes are specified in the \ac{tos} and \ac{pp}, the user will not know how much 
effort the service provider will put into abiding by them. There are many unfortunate 
examples of misuse in the media as we highlighted in the \lcnameref{chapter:introduction}.
Such instances are factual breaches of security of the \ac{osn} infrastructure but 
more importantly, probable violations of the privacy of \ac{osn} users' personal 
information --- see \cite{CutilloMS10} or \cite{GaoHHWC11} for examples of attacks 
against \acp{osn} affecting the privacy, integrity and availability of personal 
data of the users.

For that reason, decentralisation is the most common solution proposed nowadays 
as a solution to the central authority that amasses all the personal data shared 
by the users in centralised \acp{osn}. However, decentralisation is not an easy 
task and usually there is not one single solution, for example, \cite{ShakimovVCC09} 
describes three alternative architectures to decentralised \acp{osn} --- namely, 
cloud based, desktop based with socially informed replication and a hybrid of the 
former two --- differing in privacy, costs and availability trade-offs. 

Past research has focused on removing the central provider and guaranteeing the 
confidentiality of users' personal data --- via cryptographic algorithms and protocolos 
--- but little has been done to address other major trade-offs when distributing 
the responsibilities of the central authority among the users of the \ac{osn} \cite{GreschbachKB12} 
--- see \cref{section:related-work} for \lcnameref{section:related-work} on \acp{dosn}.

When decentralising an \ac{osn} --- and other similar centralised \acp{is} --- one 
has to consider how to address topics such as identification of the users, storage 
of personal data and, access control policies --- and their enforcement --- for 
the stored data as well information accountability.

\textit{Identification}. In traditional \acp{osn} --- and most centralised \acp{is} 
--- authentication of users for the purpose of accessing the system is done by the 
service provider itself or via some authorised trusted third party. The central 
authority keeps and maintains a registry of records with diverse information about 
the user for the purpose of authenticating the user after identifying her. The authority 
usually has some other identifying information should the user forgets her credentials 
and needs to gain access to the system again.

When decentralising such mechanisms of identification and authentication there is 
not a central authority to which rely on to guarantee unique identifiers or recovering 
the credentials. One of the solutions for a user to identifying herself in such 
scenario would be using public key cryptography because the user could use the public 
key as identifier while authenticate herself since she holds the paired private key. 

Although cryptographic identifiers are not memorable for the users nor portable 
enough for the dynamism of \acp{osn} and the different devices where they are used 
nowadays. A solution to this would be using a \ac{pki} for which there are decentralised 
solutions such as the web of trust ---  decentralised networks of \ac{p2p} certification 
--- of \ac{pgp} \cite{Stallings95, Abdul97}. But they do not prevent a user from 
registering a public key under the identity of an already registered user --- identity 
retention. 

% TODO: Mention Twister as example?
Such consistency guarantee is now possible thanks to leveraging on the blockchain 
that enables cryptocurrencies such as \Bitcoin \cite{Nakamoto08}. The blockchain 
is a distributed sequential database of growing list of records --- blocks of transactions. 
Because blocks can only be appended it is not possible to alter already registered 
transactions and this characteristic --- among others of this technology --- can 
be used to impose such consistency guarantee for identifiers in a \ac{dosn} setup.

\textit{Data storage}. 

\textit{Access control policies for the stored data}. 

\textit{Information accountability}. 
dl.acm.org/citation.cfm?id=1349043



talk about possibility of business models in a decentralised (refer to apple's differential 
privacy but perhpaps look for another one from facebook or twitter if they even exists...)
---
Give works on decentralisation
 - Privacy, Cost, and Availability Tradeoffs in Decentralized OSNs -> Decentralization alternatives
   \cite{ShakimovVCC09}
---
% Sample of centralised vs decentralised system: https://en.wikipedia.org/wiki/File:Decentralization_diagram.svg
% How most designs are centralised (and why)
% Cite "Centralized information systems and the legal right to privacy"
% What it means to be decentralised
% How we can decentralise such centrali
% Trade off (this leads to the main point of privacy that comes now)
---


%\subsection{\Aclp*{dosn} in this thesis}
\subsection{Decentralised Online Social Networks in this thesis}
    \label{subsection:dosns-in-this-thesis}

\section{Privacy: a never ending battle}
    \label{section:privacy-a-never-ending-battle}

Privacy as a fundamental human right recognised in the \ac{un} Declaration of Human 
Rights, the International Convenant on Civil and Political Rights and in many other 
international treaties, can be defined as ``the right to be let alone'' \cite{Westin70}.

Although there have been many attempts throughout history to define privacy from different
perspectives, for example, as part of a legal framework or as a social value.
---
One definition of interest that englobes all this stuff from Westin who describes 
four states of privacy: solitude, intimacy, anonymity, and reserve. These states must balance participation against norms:

% Data control is a complex paradigm that hinders conflicting interests. For example,
% while it seems reasonable to let individuals keep control of their financial data,
% as it is rather sensitive personal information, certain obligations, such as tax
% laws, oblige individuals to share this type of information to governmental agencies
% and even society \cite{Allen99}.

% Each individual is continually engaged in a personal adjustment process in which
% he balances the desire for privacy with the desire for disclosure and communication
% of himself to others, in light of the environmental conditions and social norms
% set by the society in which he lives. - Alan Westin, Privacy and Freedom, 1968[48]


What privacy is


What it implies
Why is it important (and in our context)
How it can be breached or attacked or diminished
What can we do: crypto for access control
---

% \section{Graph modeling: there is so much beyond one-to-one relationships}
% TODO: Develop for PhD kappa
%
% \section{Analytics: from raw data to information}
% TODO: Develop for PhD kappa
%
% Raw data is a set of unorganized facts about something that when processed, organized
% and structured can be presented in some meaningful manner resulting in information.
% Such information is key to get conclusions via a process: analysis.
%
% Analysis of data in OSNs is important from a business perspective but doing so will
% necessarily breach the boundaries of user privacy. Can we do business in a privacy-preserving
% manner? (This is the key question).

\subsection{Privacy aspects in this thesis}
    \label{subsection:privacy-aspects-in-this-thesis}


\section{Related work}
    \label{section:related-work}

%NOTE: Undecided, whether this best fit as a subsection or part of the introduction or simply not exist.

Here I could talk about Diaspora, Twister, perhaps Ello, but also the works that try to 
use OSNs in a privacy-preserving manner (that is, using an overlay of encryption 
or perhaps a link to a third party that is under the user's control?).
Vis-a-Vis project
Check GreschbachKB12 for citations of Safebook and other projects.



\chapter{Our research problem}
    \label{chapter:our-research-problem}

\requote{Poetry is a succession of questions which the poet constantly poses}{Vicente 
P{\'i}o Marcelino Cirilo Aleixandre y Merlo}

\lettrine{\textcolor[gray]{.25}{W}}{e} aim at providing some methods to improve 
user data control in the realm of centralised and decentralised \ac{is} 
and, in a particular scenario: \aclp{dosn}.


Main idea of the problem we are trying to solve (or shortlist of them but try to come up with a general idea first)

Privacy-Preserving methods for services in decentralised systems in general and applied ones


\section{Challenges addressed}
    \label{section:challenges-addressed}
Among the wide spectrum of problems and questions that decentralisation poses in 
\acp{is} such as \acp{osn} we have first focused our efforts on keeping 
the functionality of some features in decentralised systems as close as possible to 
what these would be in a centralised scenario.

In parallel, such decentralisation triggers our second concern in terms of privacy.
Not only we consider this topic in such setup where the central provider does not 
exist or it plays a minimal role but also in a more generic fashion for an information 
system.

Each one, decentralisation and privacy, raise so many questions that any thesis 
or research will ever be able to answer. Therefore, we list the main questions that 
we believe this thesis answers in the aforementioned domains. There may be additional 
questions we have not explicitly considered but have an answer in our work. Moreover, 
the reader may expect some questions that we have not even thought as of the time 
of this writing --- though this is the beauty of 
tthere are other expected questions the reader does not find an answer because 
we did not think about them at the time of of this writing.
\begin{itemize}
    \item Can we decentralize an \ac{is}? If so, how?\\
    \Cref{article:passwords-peer-to-peer,article:events-invitations-dosns}
    \item Can we mimic some of the features of centralised \acp{is} in a decentralised setup?\\
    \Cref{article:passwords-peer-to-peer,article:events-invitations-dosns}
    \item What are the trade-offs when protecting the privacy of the user in an \ac{is}?\\
    \Cref{article:passwords-peer-to-peer,article:events-invitations-dosns,,article:document-submission-system}
\end{itemize}

---
Keep functionality of OSNs (in the decentralised scenario) while provide such functionality 
in a privacy-preserving manner.

Practice is nothing without theory and viceversa (but there are trade offs when leaning to either side).

Talk about delimitations as well
---

\section{Methodology}
    \label{section:methodoloy}
Computer science is the study of the phenomena that surrounds computers by means 
of the theory, experiments and design that leads to the actual construction of computers. 
That is, the computer is not only a collection of interconnected pieces of hardware 
but also a programmed ``living'' device --- the studied ``organism'' --- \cite{NewellS76}.

As such, and derived from the natural sciences as a continuous process, we take 
the scientific method as the main approach to address our research problem by systematic 
observation, measurement, experimentation, induction and formulation of hypotheses, 
testing of our deductions, and possibly the modification of such hypotheses \cite{Oxford14}.

Among the most prominent research methodologies used nowadays --- quantitative and 
qualitative --- we do not side with any of them and instead take a mixed approach. 
Mixed methods research gives a greater flexibility when answering research questions 
rather than constraining approaches or ideas. We try to find research questions 
that offer the best chance to obtain useful and meaningful answers. And we think 
that a combined approach achieves the best results \cite{JohnsonO04}.

Such mixed methods approach leads us to design science research methodologies for 
\acp{is}, which consist of six main steps: identification and motivation of the 
problem, definition of the objectives for a solution, design and development of 
the solution, demonstration, evaluation and finally, communication of the research 
\cite{PeffersTRC07}. We use the previous criteria to classify and describe the works 
included in this thesis in \cref{table:papers-methodologies}.

{ % restricting the table to be always on top of a page to this case (local not global)
\makeatletter
\setlength{\@fptop}{0pt}
\setlength{\@fpbot}{0pt plus 1fil}
\makeatother
\begin{table}[ht!]
    \centering
    \begin{tabular}{llll}
        \toprule
        \multirow{2}{*}{Design Research Step} & \multicolumn{3}{c}{Articles} \\
        \cmidrule{2-4}
         & \ref{article:passwords-peer-to-peer} & \ref{article:events-invitations-dosns} & \ref{article:document-submission-system} \\
        \midrule
        Problem identification and motivation & \ding{51} & \ding{51} & \ding{51} \\
        Definition of objectives & \ding{51} & \ding{51} & \ding{51} \\
        Solution design and development & \ding{51} & \ding{51} & \ding{51} \\
        Demonstration & - & - & \ding{51} \\
        Evaluation & \ding{51} & \ding{51} & \ding{51} \\
        Communication & \ding{51} & \ding{51} & \ding{51} \\
        \bottomrule
    \end{tabular}
    \caption{Pong}
    \label{table:papers-methodologies}
\end{table}
}

% ---
% The following gets answered with the previous table
% 
% Types of information used/analyzed/gathered/studied and source of such data (random, dataset, etc)
%
% Parts of systems designed and/or analyzed
%
% Prototyping or such
%
% Break down these parts (analysis, design, protocol/theory, prototyping) for each
% paper?
% ---


\chapter{Our contributions}
    \label{chapter:our-contributions}
\renewcommand\thesection{\Alph{section}}
% P2P
\section{\usebibentry{KreitzBGRB12}{title}}
\begingroup\centering
\begin{ppBox}
    \bibentry{KreitzBGRB12}
\end{ppBox}
\endgroup

\subsection{Summary}

% Benny's thesis
% The problem of how to transfer credentials in a usable and secure way from one device to another is one of the concerns of Article B contained in this thesis. The proposed solution emulates the traditional username/password-login paradigm in a decentralized system.

\subsection{Contributions}
%\subsection{Changes for the thesis}

% EI
\section{\usebibentry{RodriguezCanoG14}{title}}
\begingroup\centering
\begin{ppBox}
    \bibentry{RodriguezCanoG14}
\end{ppBox}
\endgroup

\subsection{Summary}
\subsection{Contributions}
%\subsection{Changes for the thesis}

% DSS
\section{\usebibentry{GreschbachREB15}{title}}
\begingroup\centering
\begin{ppBox}
    \bibentry{GreschbachREB15}
\end{ppBox}
\endgroup

\subsection{Summary}
% In such scenario, our proposed protocol guarantees what we defined as ``forward
% unlinkability'' of a document with the author --- a document and its author's identity
% must remain unlinkable even after receiving a grade --- and ``provable linkability''
% of an author with a grade --- the system must guarantee that an author receives
% a grade if and only if the author submitted a document for grading that was actually
% graded.

\subsection{Contributions}
%\subsection{Changes for the thesis}

\renewcommand\thesection{\thechapter.\arabic{section}}
\chapter{Conclusions}
    \label{chapter:conclusions}
\requote{Space: the final frontier. These are the voyages of the starship Enterprise. Its 
continuing mission: to explore strange new worlds, to seek out new life and new 
civilizations, to boldly go where no one has gone before}{Star Trek: The Next Generation}

\lettrine{\textcolor[gray]{.25}{W}}{hen} Tim Berners-Lee invented the \Ac{www} in 1989, he envisioned a large-scale 
and decentralised \ac{is} where everybody would be able to have their own website 
and plug-in a server from which they could offer their content. Such idea never 
worked out and, instead, personal data ended up under the control of major technological 
companies in huge data centers --- ``silos'' --- distributed all over the world.

% https://archive.org/details/DWebSummit2016_Keynote_Tim_Berners_Lee
% https://web.archive.org/web/20161106081045/https://techcrunch.com/2016/10/09/a-decentralized-web-would-give-power-back-to-the-people-online/
The idea of bringing back the initial design of a decentralised \ac{www} has been 
in the minds of many researchers and practitioners for a while now --- ``bring back 
the power to the people'' as Tim Berners-Lee said during his keynote ``Re-decentralizing 
the web -- some strategic questions'' at the ``Decentralized Web Summit'' in 2016. 

The decentralisation of \acp{is} on the \Internet has gained some popularity in recent 
years due to concerns on what the personal data was being used for but it took a major turnaround
after the revelations by Snowden on worldwide scale governmental surveillance.
%Gaining crucial relevance after the revelations by Snowden on worlwide scale governmental surveillance
% Here talk about how privacy on the internet was mostly focused on the business side
% till the end of 20th century and how all the power of technological companies made
% that a concern, and after snowden revelations things changed not just because of the
% governmetal surveillance but also because of the collaboration of these companies.

Such idea of decentralisation is what motivates this thesis as means to achive 
privacy as data control in similar terms as those by Allen et al \cite{Allen99}. Though overruling 
the power of a central authority like the service provider of an \ac{osn} entails 
sharing the duties and responsibilities among a subset of federated servers or among
all the users of the network in a \ac{p2p} fashion.




\section{Future directions}
    \label{section:future-directions}
- More features/functionality
- Other scenarios where decentralisation fits
- More formal security and privacy evaluation
- Privacy-preserving analytics
- Protection of the social network graph from within


%\chapter{Glossary of acronyms and abbreviations}
\chapter{Glossary of acronyms}
\acsetup{list-caps=true, list-heading=none, extra-style=comma}
% Reset all acronyms
%\acresetall
% Mark all acronyms as used
%\acuseall
\printacronyms


\part{Published articles}
    \label{part:published-articles}
\renewcommand\thechapter{\Alph{chapter}}
\renewcommand{\chaptername}{Article}

\chapter{\usebibentry{KreitzBGRB12}{title}}
    \label[artsec]{article:passwords-peer-to-peer}


\chapter{\usebibentry{RodriguezCanoG14}{title}}
    \label[artsec]{article:events-invitations-dosns}


\chapter{\usebibentry{GreschbachREB15}{title}}
    \label[artsec]{article:document-submission-system}


\newenvironment{quote_tos}{%
    \definecolor{formalshade}{rgb}{0.95,0.95,1}
    \setlength{\parindent}{0pt}
    \def\FrameCommand{%
        \hspace{1pt}%
        {\color{DarkBlue}\vrule width 2pt}%
        {\color{formalshade}\vrule width 4pt}%
        \colorbox{formalshade}%
    }%
    \MakeFramed{\advance\hsize-\width\FrameRestore}%
    \noindent\hspace{-4.55pt}% disable indenting first paragraph
    \begin{adjustwidth}{}{7pt}%
        \vspace{2pt}\vspace{2pt}%
}
{%
    \vspace{2pt}\end{adjustwidth}\endMakeFramed%
}

%\appendix
\begin{appendices}
    \addappheadtotoc
    %\appendixpage
    \renewcommand\thechapter{\greek{chapter}}
    \renewcommand\thesection{\thechapter.\greek{section}}
    \crefalias{section}{appsec}
    \crefalias{chapter}{appchp}

    %\chapter{Excerpts of \aclp*{tos} and \aclp*{pp}}
    \chapter{Excerpts of Terms of Services and Privacy Policies}
        \label{chapter:excerpts-of-tos-and-pp}
    
    % TODO: replace the word sections with a cleverref command
    \lettrine{\textcolor[gray]{.25}{I}}{n} the following sections we quote some 
    selected excerpts of the \ac{tos} and \ac{pp} of a few \ac{osn} service providers 
    that we thought --- as computer scientists --- interesting to be aware of in 
    our realm of research.
    
    We include the last update date stated by the service provider of the \ac{tos} 
    and \ac{pp} as well as the corresponding \acp{url} where we originally found 
    the terms and policies --- however, the wording and even the \ac{url} are likely 
    to change at any time.

    \section[\Facebook]{\Facebook (\FacebookInc)}
        \label{section:excerpts-facebook}
    % \cite{Facebook15}
    \Facebook's \acs{tos} are called ``Statement of Rights and Responsibilities
    '' and available at \url{https://www.facebook.com/terms}. The last revision as of 
    this writing is dated on January 30th, 2015. The following is an excerpt of some 
    selected sections of the \ac{tos}.

    \begin{quote_tos}
        \[...\]
        \textbf{Privacy}

        Your privacy is very important to us. We designed our Data Policy to make important 
        disclosures about how you can use Facebook to share with others and how we collect 
        and can use your content and information. We encourage you to read the Data Policy, 
        and to use it to help you make informed decisions. 

        \vspace{\baselineskip}

        \textbf{Sharing Your Content and Information}

        You own all of the content and information you post on Facebook, and you can control 
        how it is shared through your privacy and application settings. In addition:

        \begin{enumerate}
            \item For content that is covered by intellectual property rights, like photos 
            and videos (IP content), you specifically give us the following permission, 
            subject to your privacy and application settings: you grant us a non-exclusive, 
            transferable, sub-licensable, royalty-free, worldwide license to use any IP 
            content that you post on or in connection with Facebook (IP License). This IP 
            License ends when you delete your IP content or your account unless your content 
            has been shared with others, and they have not deleted it.
    
            \item When you delete IP content, it is deleted in a manner similar to emptying 
            the recycle bin on a computer. However, you understand that removed content 
            may persist in backup copies for a reasonable period of time (but will not be 
            available to others).
    
            \item When you use an application, the application may ask for your permission 
            to access your content and information as well as content and information that 
            others have shared with you.  We require applications to respect your privacy, 
            and your agreement with that application will control how the application can 
            use, store, and transfer that content and information.  (To learn more about 
            Platform, including how you can control what information other people may share 
            with applications, read our Data Policy and Platform Page.)
    
            \item When you publish content or information using the Public setting, it means 
            that you are allowing everyone, including people off of Facebook, to access 
            and use that information, and to associate it with you (i.e., your name and 
            profile picture).
    
            \item We always appreciate your feedback or other suggestions about Facebook, 
            but you understand that we may use your feedback or suggestions without any 
            obligation to compensate you for them (just as you have no obligation to offer 
            them).
        \end{enumerate}
        \[...\]
    \end{quote_tos}

    \section[\LinkedIn]{\LinkedIn (\LinkedInCorp)}
        \label{section:excerpts-linkedin}
    % \cite{LinkedIn14}
    \LinkedIn's \acs{tos} are called ``User Agreement'' and available at \url{https://www.linkedin.com/legal/user-agreement}. 
    The last revision as of this writing is dated on October 23rd, 2014. The following 
    is an excerpt of some selected sections of the \ac{tos}.

    \begin{quote_tos}
        \[...\]
        \textbf{Your License to LinkedIn}
        
        As between you and LinkedIn, you own the content and information that you 
        submit or post to the Services and you are only granting LinkedIn the following 
        non-exclusive license: A worldwide, transferable and sublicensable right 
        to use, copy, modify, distribute, publish, and process, information and 
        content that you provide through our Services, without any further consent, 
        notice and/or compensation to you or others. These rights are limited in 
        the following ways:
        \begin{enumerate}[label=\alph*]
            \item You can end this license for specific content by deleting such 
            content from the Services, or generally by closing your account, except 
            (a) to the extent you shared it with others as part of the Service and 
            they copied or stored it and (b) for the reasonable time it takes to 
            remove from backup and other systems.
            \item We will not include your content in advertisements for the products 
            and services of others (including sponsored content) to others without 
            your separate consent. However, we have the right, without compensation 
            to you or others, to serve ads near your content and information, and 
            your comments on sponsored content may be visible as noted in the Privacy 
            Policy.
            \item We will get your consent if we want to give others the right to 
            publish your posts beyond the Service. However, other Members and/or 
            Visitors may access and share your content and information, consistent 
            with your settings and degree of connection with them.
            \item While we may edit and make formatting changes to your content 
            (such as translating it, modifying the size, layout or file type or 
            removing metadata), we will not modify the meaning of your expression.
            \item Because you own your content and information and we only have 
            non-exclusive rights to it, you may choose to make it available to others, 
            including under the terms of a Creative Commons license.
        \end{enumerate}
        You agree that we may access, store and use any information that you provide 
        in accordance with the terms of the Privacy Policy and your privacy settings.

        By submitting suggestions or other feedback regarding our Services to LinkedIn, 
        you agree that LinkedIn can use and share (but does not have to) such feedback 
        for any purpose without compensation to you.

        You agree to only provide content or information if that does not violate 
        the law nor anyone's rights (e.g., without violating any intellectual property 
        rights or breaching a contract). You also agree that your profile information 
        will be truthful. LinkedIn may be required by law to remove certain information 
        or content in certain countries.
        \[...\]
        \textbf{Limits}
        
        LinkedIn reserves the right to limit your use of the Services, including 
        the number of your connections and your ability to contact other Members. 
        LinkedIn reserves the right to restrict, suspend, or terminate your account 
        if LinkedIn believes that you may be in breach of this Agreement or law 
        or are misusing the Services (e.g. violating any Do and Don'ts).

        LinkedIn reserves all of its intellectual property rights in the Services. 
        For example, LinkedIn, SlideShare, LinkedIn (stylized), the SlideShare and 
        “in” logos and other LinkedIn trademarks, service marks, graphics, and logos 
        used in connection with LinkedIn are trademarks or registered trademarks 
        of LinkedIn. Other trademarks and logos used in connection with the Services 
        may be the trademarks of their respective owners
        \[...\]
    \end{quote_tos}

    \section[\Twitter]{\Twitter (\TwitterInc)}
        \label{section:excerpts-twitter}
    % \cite{Twitter16}
    \Twitter's \acs{tos} are called ``Twitter Terms of Service'' and available at 
    \url{https://twitter.com/tos}. The last revision as of 
    this writing is dated on September 30th, 2016. The following is an excerpt of some 
    selected sections of the \ac{tos}.

    \begin{quote_tos}
        \[...\]
        \textbf{Your Rights}
        
        You retain your rights to any Content you submit, post or display on or 
        through the Services. What’s yours is yours — you own your Content (and 
        your photos and videos are part of the Content).

        By submitting, posting or displaying Content on or through the Services, 
        you grant us a worldwide, non-exclusive, royalty-free license (with the 
        right to sublicense) to use, copy, reproduce, process, adapt, modify, publish, 
        transmit, display and distribute such Content in any and all media or distribution 
        methods (now known or later developed). This license authorizes us to make 
        your Content available to the rest of the world and to let others do the 
        same. You agree that this license includes the right for Twitter to provide, 
        promote, and improve the Services and to make Content submitted to or through 
        the Services available to other companies, organizations or individuals 
        for the syndication, broadcast, distribution, promotion or publication of 
        such Content on other media and services, subject to our terms and conditions 
        for such Content use. Such additional uses by Twitter, or other companies, 
        organizations or individuals, may be made with no compensation paid to you 
        with respect to the Content that you submit, post, transmit or otherwise 
        make available through the Services.

        Twitter has an evolving set of rules for how ecosystem partners can interact 
        with your Content on the Services. These rules exist to enable an open ecosystem 
        with your rights in mind. You understand that we may modify or adapt your 
        Content as it is distributed, syndicated, published, or broadcast by us and 
        our partners and/or make changes to your Content in order to adapt the Content 
        to different media. You represent and warrant that you have all the rights, 
        power and authority necessary to grant the rights granted herein to any 
        Content that you submit.
        \[...\]
    \end{quote_tos}

\end{appendices}


%\chapter*{References}
%\addcontentsline{toc}{chapter}{Bibliography}
%\dogearRGB{118,118,118}

\bibliography{data/bibliography}

\end{document}
\endinput

