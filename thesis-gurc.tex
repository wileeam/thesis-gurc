% -*- mode: TeX -*-
% -*- coding: utf-8 -*-

\documentclass[showtrims]{kthesis}
\usepackage[T1]{fontenc}
\usepackage{textcomp}
\usepackage{lmodern}
\usepackage[utf8]{inputenc}
\usepackage[swedish, english, french, spanish]{babel}

%\usepackage{xcolor}
%\usepackage{tikz}

% *** enables full bibliographic entries within text ***
\usepackage{bibentry}
\nobibliography*

% *** citation numbers will be sorted and properly "compressed/ranged" ***
\usepackage[noadjust]{cite}
% *** allowed for \citeauthor and \citeyear commands ***
\usepackage[authoryear,numbers,square]{natbib}

% DEBUG
% *** provides random text to fill up boxes, etc ***
\usepackage{lipsum}

\usepackage{mfirstuc}
\usepackage{acro}

% *** capitalisation config for acronyms ***
\acsetup{uc-cmd=\capitalisewords}
% *** acronyms file loading ***
% -*- mode: TeX -*-
% -*- coding: utf-8 -*-

\DeclareAcronym{ssf}{
    short       = SSF,
    long        = Stiftelsen f{\"o}r Strategisk Forskning,
    foreign     = Swedish Foundation for Strategic Research,
    foreign-lang = english
}

\DeclareAcronym{vr}{
    short       = VR,
    long        = Vetenskapsr{\aa}det,
    foreign     = Swedish Research Council,
    foreign-lang = english
}

\DeclareAcronym{un}{
    short       = UN,
    long        = united nations
}

\DeclareAcronym{usa}{
    short       = USA,
    long        = United States of America,
    alt         = US
}

\DeclareAcronym{www}{
    short       = WWW,
    long        = world wide web
}

\DeclareAcronym{prism}{
    short       = PRISM,
    long        = personal record information system methodology
}

\DeclareAcronym{cern}{
    short       = CERN,
    long        = Conseil Europ{\'e}en pour la Recherche Nucl{\'e}aire,
    foreign     = European Organization for Nuclear Research,
    foreign-lang = english
}

\DeclareAcronym{sn}{
    short       = SN,
    long        = social network
}

\DeclareAcronym{osn}{
    short       = OSN,
    long        = online social network,
    long-indefinite = an
}

\DeclareAcronym{dosn}{
    short       = DOSN,
    long        = decentralised online social network
}

\DeclareAcronym{p2p}{
    short       = P2P,
    long        = peer-to-peer
}

\DeclareAcronym{tos}{
    short       = TOS,
    long        = terms of service
}

\DeclareAcronym{pp}{
    short               = PP,
    long                = privacy policy,
    long-plural-form    = privacy policies
}

\DeclareAcronym{udm}{
    short       = UDM,
    long        = user data manifesto
}

\DeclareAcronym{url}{
    short       = URL,
    long        = unified resource locator
}


% *** adds a space unless the macro is followed by certain punctuation characters ***
\usepackage{xspace}

\usepackage{cleveref}

%\usepackage{algorithm}

\usepackage{hyperxmp}
% == load hyperref last ==
%\usepackage[hidelinks, pagebackref=true]{hyperref}
\usepackage[colorlinks=true, pagebackref=true, linktocpage=true]{hyperref}

% *** customisation for pagebackref option in hyperref package ***
% http://tex.stackexchange.com/questions/38149/removing-double-entries-from-hyperrefs-pagebackref
\renewcommand*{\backreflastsep}{, }
\renewcommand*{\backreftwosep}{, }
\renewcommand*{\backref}[1]{}
\renewcommand*{\backrefalt}[4]{%
  \ifcase #1 %
    No citations.% use \relax if you do not want the "No citations" message
  \or
    (Cited in page #2).%
  \else
    (Cited in pages #2).%
  \fi%
}

%%
%% This is file `kth-bibl.tex',
%% generated with the docstrip utility.
%%
%% The original source files were:
%%
%% kthesis.dtx  (with options: `biblio')
%% 
%% IMPORTANT NOTICE:
%% 
%% For the copyright see the source file.
%% 
%% Any modified versions of this file must be renamed
%% with new filenames distinct from kth-bibl.tex.
%% 
%% For distribution of the original source see the terms
%% for copying and modification in the file kthesis.dtx.
%% 
%% This generated file may be distributed as long as the
%% original source files, as listed above, are part of the
%% same distribution. (The sources need not necessarily be
%% in the same archive or directory.)
\title{Practical Methods for Privacy Preservation}
\subtitle{A centralised and decentralised approach}
\author{Guillermo Rodr{\'i}guez Cano}
\date{maj 2017}
\thesistype{Doctoral Thesis}
\imprint{Stockholm, Sweden 2017}
\examen{teknologie doktorsexamen i datalogi}
\disputationsdatum{torsdagen den 17 maj 2017 klockan 10.00}
\disputationslokal{Kollegiesalen, Administrationsbyggnaden,
  Kungl Tekniska h{\"o}gskolan, Valhallav{\"a}gen~79, Stockholm}
\isbn{ISBN x-xxxx-xxx-x}
\issn{ISSN xxxx-xxxx}
\isrn{ISRN KTH/xxx/xx-{}-yy/nn-{}-SE}
\trita{TRITA xxx yyyy-nn}
\publisher{Universitetsservice US AB}
\address{
    KTH Royal Institute of Technology\\
    School of Computer Science and Communication\\
    SE-100 44 Stockholm\\
    SWEDEN}
\kthlogo{images/kth_svv_comp_science_comm}


\newcommand{\thesisauthor}{\theauthor}                                        % Nombre y apellidos del autor
\newcommand{\thesiscontactemail}{gurc@kth.se}
\newcommand{\thesissupervisor}{TBD}                              % Nombre y apellidos del tutor/a o tutores
\newcommand{\thesissupervisorcover}{\thesissupervisor}                         % Nombre y apellidos del tutor/a o tutores

\newcommand{\thesistitle}{\thetitle}                  % Título del trabajo
\newcommand{\thesistitleshort}{\thesistitle}                                                    % Título del trabajo (para meta-datos PDF)
\newcommand{\thesislanguage}{EN}
\newcommand{\thesiscopyright}{Copyright (C) 2016, \thesisauthor}                                % Copyright
\newcommand{\thesislicense}{Creative Commons (by-nc-sa) 3.0 Sweden}                             % Nombre de la licencia
\newcommand{\thesislicenseurl}{http://creativecommons.org/licenses/by-nc-sa/3.0/se/legalcode}   % Enlace a los términos de la licencia
\newcommand{\thesiskeywords}{thesis, degree, computer science, Uppsala,
                             model-driven development, product lines, feature model,
                             variability, validation, configuration, hyper-graph, traversal}    % Palabras clave (para incrustar en PDF)
\newcommand{\thesiskeywordsabstractes}{desarrollo dirigido por modelos, l{\'i}neas de productos,
                                       modelo de caracter{\'i}sticas, variabilidad,
                                       validaci{\'o}n, configuraci{\'o}n, hiper-grafo,
                                       recorrido}                                               % Palabras clave en castellano (para resumen)
\newcommand{\thesiskeywordsabstracten}{model-driven development, product lines, feature model,
                                       variability, validation, configuration, hyper-graph,
                                       traversal}                                               % Palabras clave en inglés (para abstract)
\endinput
%%
%% End of file `kth-bibl.tex'.


\usepackage[many]{tcolorbox}
\newtcolorbox{ppBox}[1][]{
    breakable,
    enhanced,
    skin=enhancedmiddle,
    frame hidden,
    interior hidden,
    title=#1,
    arc=0pt,
    outer arc=0pt,
    top=0pt,
    bottom=0pt,
    boxsep=1mm,
    borderline={1.5pt}{0mm}{black},
    fonttitle=\sffamily\footnotesize,
    fontupper=\sffamily\footnotesize,
    fontlower=\sffamily\footnotesize,
    colframe=black,
    colback=white,
    colbacktitle=white,
    coltitle=black,
    fonttitle=\bfseries,
    boxrule=0pt,
    bottomrule=0pt,
    toprule=0pt,
  % leftrule=3pt,
  % rightrule=3pt,
    titlerule=0pt,
    width=8\linewidth/10,
    valign=center,
}

% colored dogear marks
\usepackage[pages=some,contents={},opacity=1.0,scale=1,angle=90]{background}
\usepackage{tikz}
\newcommand*\VerBar[1]{%
	\begin{tikzpicture}
	\fill[#1] (0,0)--(1,0)--(1,1)--(0,1)--cycle;
	\draw[white] (0.5,0.5)node{\rotatebox{-90}{\Huge \textbf \thechapter}};
	\end{tikzpicture}
}
\newcounter{DogearHShift}
\newcommand{\dogear}[1]{%
  \backgroundsetup{position={current page.north east},vshift=0.5cm,%
    hshift=-\theDogearHShift cm,contents={\VerBar{#1}}}%
  \BgThispage%
  \addtocounter{DogearHShift}{1}
}
\newcommand{\dogearRGB}[1]{%
  \definecolor{dogearColor}{RGB}{#1}%
  \dogear{dogearColor}%
}
%usage: \dogear{COLOR}, e.g. \dogear{red}
%   or: \dogearRGB{RED,GREEN,BLUE}, e.g. \dogearRGB{255,0,0}
% ------------------------------------

%%%
% CONFIG: Meta-data properties for PDF and XMP
%
% Notes: Load after kth-bibl (as variables used come from there)
%
\hypersetup{
    pdftitle        = {\thesistitle},
    pdfauthor       = {\thesisauthor},
%    pdfauthortitle={},%
    pdfsubject      = {Doctoral thesis},
%    pdfsubject      = {\thesistype~(\degree),~\department,~\university},
    pdfkeywords     = {\thesiskeywords},
    pdfcreator      = {\thesisauthor~(Supervised~by~\thesissupervisor)},
    pdfproducer     = {\thesisauthor~powered~by~\LaTeX},
%    pdfcontactaddress={},
%    pdfcontactcity={Lentini (SR)},
%    pdfcontactpostcode={},
%    pdfcontactcountry={},
%    pdfcontactphone={},
%    pdfcontactemail={},
%    pdfcontacturl={},
%    pdfcaptionwriter={},
    pdflang={en},
%    pdfcopyright    = {\thesiscopyright},
    pdflicenseurl   = {\thesislicenseurl},
    pdfstartview    = FitB,
    colorlinks=true,
    breaklinks,
    linkcolor={blue},
    citecolor={red},
    urlcolor={blue}
}



% *** 
% -*- mode: TeX -*-
% -*- coding: utf-8 -*-

% *** definitions ***
\newcommand{\etal}{ et\,al. }
\newcommand{\eg}{e.\,g.,\ } % note the trailing comma (recommended by http://grammar.quickanddirtytips.com/ie-eg-oh-my.aspx )
\newcommand{\Eg}{E.\,g.,\ }
\newcommand{\ie}{i.\,e.,\ }
\newcommand{\Ie}{I.\,e.,\ }


%%%
% DOC: Automatized formatting for quotes :) Usually it will be
%      one quote/citation per chaper, and it should be located
%      right after the chapter's title, \chapter{bla bla bla}
%      Usage: \requote{frase}{autor}
%
%      For instance: \requote{To be... or not to be: that is the question}{William Shakespeare}
%
\newcommand{\requote}[2]{
{\fontfamily{\carlitofamily}\selectfont
    \begin{flushright}
        \begin{minipage}[b][][t]{0.45\paperwidth}
            \small{
                \itshape{\textbf{\normalsize ``}#1\textbf{\normalsize ''}}
                \begin{flushright}
                    \textbf{\textsc{#2}}
                \end{flushright}
            }
        \end{minipage}
        \bigskip
    \end{flushright}
}}

% ***
% -*- mode: TeX -*-
% -*- coding: utf-8 -*-

\newcommand{\Amazon}{{\itshape Amazon}\xspace}
\newcommand{\AmazonInc}{{\itshape Amazon.com, Inc.}\xspace}

\newcommand{\Apache}{{\itshape Apache}\xspace}

\newcommand{\Apple}{{\itshape Apple}\xspace}
\newcommand{\AppleInc}{{\itshape Apple Inc.}\xspace}

\newcommand{\Bitcoin}{{\itshape Bitcoin}\xspace}

\newcommand{\Cyworld}{{\itshape Cyworld}\xspace}

\newcommand{\Facebook}{{\itshape Facebook}\xspace}
\newcommand{\FacebookInc}{{\itshape Facebook, Inc.}\xspace}

\newcommand{\Flickr}{{\itshape Flick.r}\xspace}

\newcommand{\Friendster}{{\itshape Friendster}\xspace}

\newcommand{\Google}{{\itshape Google}\xspace}
\newcommand{\GoogleInc}{{\itshape Google Inc.}\xspace}
\newcommand{\GooglePlus}{{\itshape Google+}\xspace}
\newcommand{\GoogleCircles}{{\itshape Google Circles}\xspace}

\newcommand{\Internet}{{\itshape Internet}\xspace}
\newcommand{\InternetArchive}{{\itshape Internet Archive}\xspace}

\newcommand{\LinkedIn}{{\itshape LinkedIn}\xspace}
\newcommand{\LinkedInCorp}{{\itshape LinkedIn Corporation}\xspace}

\newcommand{\LiveJournal}{{\itshape LiveJournal}\xspace}

\newcommand{\Microsoft}{{\itshape Microsoft}\xspace}
\newcommand{\MicrosoftCorp}{{\itshape Microsoft Corporation}\xspace}

\newcommand{\Orkut}{{\itshape Orkut}\xspace}

\newcommand{\PeerSoN}{{\itshape PeerSoN}\xspace}

\newcommand{\Safebook}{{\itshape Safebook}\xspace}

\newcommand{\SuperNova}{{\itshape SuperNova}\xspace}

\newcommand{\Sybil}{{\itshape Sybil}\xspace}

\newcommand{\Twitter}{{\itshape Twitter}\xspace}
\newcommand{\TwitterInc}{{\itshape Twitter, Inc.}\xspace}

\newcommand{\Yahoo}{{\itshape Yahoo Inc.}\xspace}

\newcommand{\YouTube}{{\itshape YouTube}\xspace}

\newcommand{\Wayback}{{\itshape Wayback}\xspace}



\begin{document}

\frontmatter
% \maketitle
% \input{template/doc/latex/kthesis/kth-abs}
% \clearpage
% \selectlanguage{swedish}
% \begin{abstract}
%   Denna fil ger ett avhandlingsskelett.
%   Mer information om \LaTeX-mallen finns i
%   dokumentationen till paketet.
% \end{abstract}
% \selectlanguage{spanish}
% \begin{abstract}
%  Me gusta mucho esto
% \end{abstract}
 \selectlanguage{english}
% \begin{abstract}
%      Pang Ping
% \end{abstract}
\clearpage
\chapter*{Acknowledgements}
I acknowledge the acknowledgement of having written this :)

\section*{Funding}
The research leading to this thesis has been supported by the following projects:
\begin{itemize}
    \item Protection of personal information in social networks (Skydd av personlig information för sociala nätverk)\\
    Funded by the \acl{ssf} grant: SSF FFL09-0086.\\
    \url{http://stratresearch.se/en/research/ongoing-research/framtidens-forskningsledare-4/project/4048/}
    \item Privacy-preserving social and community networks\\
    Funded by the \Acl{vr} grant: VR 2009-3793.\\
    \url{http://vrproj.vr.se/detail.asp?arendeid=69587}
\end{itemize}

We are grateful to the Swedish taxpayers for their funding.

\clearpage
\addtocontents{toc}{~\hfill\textbf{Page}\par}
\tableofcontents
\mainmatter
\chapter{Introduction}
In 1989, while at the European Particle Physics Laboratory, at \ac{cern}, Tim Berners-Lee 
invented the \Ac{www}. No one at that time would have imagined that such invention 
has been centric to the rapid development of the digital age that we are currently 
witnessing.

Such period in human history, mainly characterised by an economy based on information 
processing as opposed to the traditional industry of the industrial revolution, 
has widen the availability of uncountable services to the general population with 
access to the Internet. Not only new business opportunities have stemmed as technological 
developments matured but also services have been overhauled to make them available 
in the digital world.

To give an example, interpersonal communication --- the passing of information between 
entities --- has evolved throughout history as communication advances developed. 
The traditional exchange of paper letters as a means of communication between two 
parties, which still prevails today, has evolved into a much more sophisticated 
and complex form in the information age: electronic mail (e-mail), which is simply 
the parallelism in the digital era to such physical letters.

Such evolution in the means of communication among individuals and other entities 
has also happened at other societal levels. For example, the social structures that 
individuals form when sharing similar interests or activities, namely social networks, 
have also seen a transposition in the digital era. The \Internet and the \ac{www} 
have taken the leading role as tools to convey different types of information from 
one party to another, expanding the boundaries of the concept of social networks, 
heavily based on tangible activity between individuals and entities.

While the concept of social networks is a theoretical term mainly used in the realm 
of social sciences to study and describe the social structures determined by the 
relationships between individuals, groups, and other types of entities such as societies; 
the technological developments of the digital era have popularised its virtual counterpart 
term: \aclp{osn}.

define osns in a brief manner (proper definition is in background). Maybe break 
down in chunks? (Or this is better to move to the background... looks like if there 
is the case).

give examples of osns and evolution

point out to what is the business model (start leaning to privacy issues here). 
Use Oleksandr's examples of TOS from Facebook and Google (also take the one from 
Twitter, LinkedIn etc to show the differences).
Check at the end of: \url{https://en.wikipedia.org/wiki/Ello_(social_network)} for 
a list of services (pick a few... try to find )

talk about the importance of privacy (in general) (mention snowden?)

talk about the particular privacy issues of osns (give examples from media)

discuss possible scenarios for solutions (from no technology, that is, no osns, 
and so no advancements in society to full control of technology, that is, governmental 
espionage: great wall of china and delaying the inevitable progress of society). 
(argue why a middle term solution is best, that is: user controls 
its data, give examples of how business is possible, for example, apple and differential 
privacy in the recent iOS).

Offer some solutions (maybe in categories). Point to decentralization as a key solution
(use \url{https://diasporafoundation.org/about#privacy} idea with the pictures too).

introduce the solutions presented 
in this thesis. Describe them briefly.

There are also some other valid solutions outside the realm of OSNs (introduce DSS 
and describe).

Discuss that these are not the only solutions, nor the best ones: these are some 
of many solutions.

End with a summary of our goal with this thesis (longer version and so description of the title?)


% \acp{osn}, although thought a
%
% %The equivalent of social networks in the digital world, online social netwrosk serve
% %as an alterna
%
% % Social networks
%  is an inherently interdisciplinary academic field which emerged from social psychology, sociology, statistics, and graph theory.
%
%
%
% Social networks, as a bla bla bla, and in particular, online social networks, as
% their equivalent in the digital world (but also as a complement to the tangible
% one) are super awesome and
%
% % Next paragraph is wrong
%
% Such process of information dissemination is one of the basis when forming relations
% among individuals and other entities --- social structures --- resulting in a set
% of social relations commonly known as social networks.
%
%
% people to build social networks or social relations with other people who share similar personal or career interests,[1] activities, backgrounds or real-life connections. The variety of stand-alone and built-in

\section{Motivation}

\section{Outline}

\section{Conventions}
Gender used in papers and thesis (we are neutral but papers and/or thesis may not 
be to ease reading though each paper is consistent in gender on its own anyways)

Although data and information can be considered synonyms in the broader sense 
--- the definition of data is a piece of information --- in the context of this 
thesis we will use the latter --- information --- as the abstract fact about something 
while the former --- data --- as the encoded form of such information that can be 
transferred over a network.


\chapter{Research problem}
Main idea of the problem we are trying to solve (or shortlist of them but try to come up with a general idea first)
- Practical methods to preserve privacy

\section{Challenges addressed}
Practice is nothing without theory and viceversa (but there are trade offs when leaning to either side).

Talk about delimitations as well


\chapter{Background}
The field of \acp{sn} is an extremely vast domain of knowledge and research. It 
has influenced dozens of domains, spanning from the most traditional ones such as 
ethnography or anthropology to more contemporanean ones such as genomics or astrophysics.

% TODO: Do I need to include something here to make the transititon from SNs to OSNs? Otherwise, rephrase the next paragraph and start with such transition?

With the explosion of large scale \acp{osn} new challenges have emerged in connection 
with the sudden availability of new types of data. The inherent increase in the 
amount of data poses a technological challenge because there is a need to cope with 
such size. Moreover, there is a need to investigate and develop efficient methods 
that can answer questions in such large space of data within a reasonable amount 
of time.

At the same time, the collection of such amount of data in \acp{osn} is having a great 
impact on the privacy of the users who use these networks to share information among 
them. Because the data is usually stored by a single centralised entity --- the 
owner of the \ac{osn} --- there is a risk for unintentional, and also intentional, 
wrong-doing of the data without knowledge of the users such as misuse, data leakages 
or even censorship.

% http://dictionary.cambridge.org/dictionary/english/be-two-sides-of-the-same-coin
That is to say, data and privacy seem to be the two sides of the same coin --- \acp{osn}. 



\section{Information systems: on the importance of capabilities}
% This section seems more appropiate for the phd...
% Reference: Social Networks and Information Systems: Ongoing and Future Research Streams
% Short summary: The IS research drawing on social networks can be divided into the following streams: 1) network awareness at both individual and organizational levels, 2) uses of social network analysis related to IS use, and 3) conceptual and technological change in the fast evolving platforms to manage social networks


\section{\Aclp{osn}: centralisation vs decentralisation}

While there is a wide understanding among researchers and practitioners on what 
a \ac{sn} is and, by extension, a \ac{osn}, there is not a canonical definition 
of neither of them in the literature. Authors seem to define their own ``interpretation'' 
as they see fit to their particular problem to solve or topic to discuss without 
much general consensus.

In our work, we have opted for the definition by \citeauthor{boydE07} because we 
believe that is the most representative for the recent works in the field of \acp{osn} 
--- an \ac{osn} site is a ``web-based service that allows individuals to construct 
a public or semi-public profile within a bounded system, articulate a list of other 
users with whom they share a connection, and view and traverse their list of connections 
and those made by others within the system'' --- \cite{boydE07}. 

The two most important parts of the previous definition are the means of communication 
--- the \ac{www} --- and the administration of the users' profiles and their connections 
in a system. It is this system what we need to clarify in fact because the \ac{www} 
is, by construction, an interconnected set of systems thus inherently distributed. 
But the \ac{osn} is a traditionally centralised system as opposed to distributed 
systems such as the \ac{www}.

There are many reasons for which most \acp{osn} designs are centralised --- define here centralised --- 
lie more on the side of data control towards business value \cite{look-for-reference}.
Popular \acp{osn} such as \Facebook, \Twitter, etc take different approaches in the 
way they have designed their users' profiles and the mechanisms for which they share 
and traverse the connections among them and, exchange information. However, most 
of the data, if not all, is stored under the control of the owner of the network\footnote{The 
owner may still use a distributed system to provide the service while still keep 
full control of all the nodes in such distributed setup}.



How most designs are centralised (and why)
What it means to be decentralised
How we can decentralise such centralisation
Trade off (this leads to the main point of privacy that comes now)

\section{Privacy: a never ending battle}

What privacy is

Privacy is a fundamental human right recognized in the UN Declaration of Human Rights, 
the International Convenant on Civil and Political Rights and in many other international 
and regional treaties.

What it implies
Why is it important
How it can be breached or attacked or diminished
What can we do: crypto

\section{Analytics: from raw data to information}
TODO: Develop for PhD kappa

Raw data is a set of unorganized facts about something that when processed, organized 
and structured can be presented in some meaningful manner resulting in information.
Such information is key to get conclusions via a process: analysis.

Analysis of data in OSNs is important from a business perspective but doing so will 
necessarily breach the boundaries of user privacy. Can we do business in a privacy-preserving 
manner? (This is the key question).

\section{Related work}
%NOTE: Undecided, whether this best fit as a subsection or part of the introduction or simply not exist.

Here I could talk about Diaspora, Twister, perhaps Ello, but also the works that try to 
use OSNs in a privacy-preserving manner (that is, using an overlay of encryption 
or perhaps a link to a third party that is under the user's control?).

\chapter{Methodology}
Computer science is the study of the phenomena that surrounds computers by means 
of the theory, experiments and design that leads to the actual construction of computers. 
That is, the computer is not only a collection of interconnected pieces of hardware 
but also a programmed ``living'' device --- the studied ``organism'' --- \cite{NewellS76}.

As such, and derived from the natural sciences as a continuous process, we take 
the scientific method as the main approach to address our research problem by systematic 
observation, measurement, experimentation, induction and formulation of hypotheses, 
testing of our deductions, and possibly the modification of such hypotheses \cite{Oxford14}.


Design science research methodology

(Mixed Methods Research: A Research Paradigm Whose Time Has Come: http://edr.sagepub.com/content/33/7/14.short)

(A Design Science Research Methodology for Information Systems Research: http://www.tandfonline.com/doi/abs/10.2753/MIS0742-1222240302)

Types of information used/analyzed/gathered/studied and source of such data (random, dataset, etc)

Parts of systems designed and/or analyzed

Prototyping or such

Break down these parts (analysis, design, protocol/theory, prototyping) for each 
paper?


\chapter{Contributions}
\renewcommand\thesection{\Alph{section}}
% P2P
\section{\ppPTP}
\begingroup\centering
\begin{ppBox}
    \bibentry{KreitzBGRB12}
\end{ppBox}
\endgroup

\subsection{Summary}
\subsection{Contributions}
%\subsection{Changes for the thesis}

% EI
\section{\ppEI}
\begingroup\centering
\begin{ppBox}
    \bibentry{RodriguezCanoG14}
\end{ppBox}
\endgroup

\subsection{Summary}
\subsection{Contributions}
%\subsection{Changes for the thesis}

% DSS
\section{\ppDSS}
\begingroup\centering
\begin{ppBox}
    \bibentry{GreschbachREB15}
\end{ppBox}
\endgroup

\subsection{Summary}
\subsection{Contributions}
%\subsection{Changes for the thesis}


\chapter{Conclusions}
\renewcommand\thesection{\thechapter.\arabic{section}}
I conclude I am not yet done... will I ever be...? :/

\section{Future work}
Space: the final frontier. These are the voyages of the starship Enterprise. Its 
continuing mission: to explore strange new worlds, to seek out new life and new 
civilizations, to boldly go where no one has gone before. (Star Trek: The Next Generation)


%\chapter*{Bibliography}
%\addcontentsline{toc}{chapter}{Bibliography}
%\dogearRGB{118,118,118}

\bibliographystyle{acm}
\bibliography{data/bibliography}

\end{document}
\endinput

