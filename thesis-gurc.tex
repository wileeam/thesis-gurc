% -*- mode: TeX -*-
% -*- coding: utf-8 -*-

\documentclass[showtrims]{kthesis}
\usepackage[T1]{fontenc}
\usepackage{textcomp}
\usepackage{lmodern}
\usepackage[utf8]{inputenc}
\usepackage[swedish, english, french, spanish]{babel}

\usepackage{mfirstuc}
\usepackage{acro}

% *** capitalisation config for acronyms ***
\acsetup{uc-cmd=\capitalisewords}
% *** acronyms file loading ***
% -*- mode: TeX -*-
% -*- coding: utf-8 -*-

\DeclareAcronym{ssf}{
    short       = SSF,
    long        = Stiftelsen f{\"o}r Strategisk Forskning,
    foreign     = Swedish Foundation for Strategic Research,
    foreign-lang = english
}

\DeclareAcronym{vr}{
    short       = VR,
    long        = Vetenskapsr{\aa}det,
    foreign     = Swedish Research Council,
    foreign-lang = english
}

\DeclareAcronym{un}{
    short       = UN,
    long        = united nations
}

\DeclareAcronym{usa}{
    short       = USA,
    long        = United States of America,
    alt         = US
}

\DeclareAcronym{www}{
    short       = WWW,
    long        = world wide web
}

\DeclareAcronym{prism}{
    short       = PRISM,
    long        = personal record information system methodology
}

\DeclareAcronym{cern}{
    short       = CERN,
    long        = Conseil Europ{\'e}en pour la Recherche Nucl{\'e}aire,
    foreign     = European Organization for Nuclear Research,
    foreign-lang = english
}

\DeclareAcronym{sn}{
    short       = SN,
    long        = social network
}

\DeclareAcronym{osn}{
    short       = OSN,
    long        = online social network,
    long-indefinite = an
}

\DeclareAcronym{dosn}{
    short       = DOSN,
    long        = decentralised online social network
}

\DeclareAcronym{p2p}{
    short       = P2P,
    long        = peer-to-peer
}

\DeclareAcronym{tos}{
    short       = TOS,
    long        = terms of service
}

\DeclareAcronym{pp}{
    short               = PP,
    long                = privacy policy,
    long-plural-form    = privacy policies
}

\DeclareAcronym{udm}{
    short       = UDM,
    long        = user data manifesto
}

\DeclareAcronym{url}{
    short       = URL,
    long        = unified resource locator
}


% -*- mode: TeX -*-
% -*- coding: utf-8 -*-

%%
%% This is file `kth-bibl.tex',
%% generated with the docstrip utility.
%%
%% The original source files were:
%%
%% kthesis.dtx  (with options: `biblio')
%% 
%% IMPORTANT NOTICE:
%% 
%% For the copyright see the source file.
%% 
%% Any modified versions of this file must be renamed
%% with new filenames distinct from kth-bibl.tex.
%% 
%% For distribution of the original source see the terms
%% for copying and modification in the file kthesis.dtx.
%% 
%% This generated file may be distributed as long as the
%% original source files, as listed above, are part of the
%% same distribution. (The sources need not necessarily be
%% in the same archive or directory.)

\title{Toward Privacy-Preserving Decentralised Systems}
\subtitle{}
\author{Guillermo Rodr{\'i}guez Cano}
\date{May 2017}
\thesistype{Licentiate Thesis}
\imprint{Stockholm, Sweden 2017}
\examen{teknologie licentiatexamen i datalogi}
\disputationsdatum{onsdagen den 31:a maj 2017 klockan 10:00}
\disputationslokal{rum 4523, Skolan f{\"o}r datavetenskap och kommunikation,
  Kungliga Tekniska h{\"o}gskolan, Lindstedtsv{\"a}gen 5, Stockholm, Sverige}
\isbn{ISBN 978-91-7729-406-1}
\issn{ISSN 1653-5723}
\isrn{ISRN KTH/CSC/A--17/12-SE}
\trita{TRITA CSC-A-2017:12}
\publisher{Universitetsservice US AB}
\address{
    KTH Royal Institute of Technology\\
    School of Computer Science and Communication\\
    SE-100 44 Stockholm\\
    SWEDEN}
\kthlogo{images/kth_svv_comp_science_comm}


\newcommand{\thesisauthor}{\theauthor}
\newcommand{\thesiscontactemail}{gurc@kth.se}
\newcommand{\thesissupervisor}{Sonja Buchegger and Johan Boye}
\newcommand{\thesissupervisorcover}{\thesissupervisor}
\newcommand{\thesistitle}{\thetitle}
\newcommand{\thesistitleshort}{\thesistitle}
\newcommand{\thesislanguage}{EN}
\newcommand{\thesiscopyright}{Copyright (C) May 2017, \thesisauthor. Some rights reserved.}
\newcommand{\thesislicense}{Creative Commons Attribution-NoDerivs 3.0 Unported (CC BY-ND 3.0)}
\newcommand{\thesislicenseurl}{https://creativecommons.org/licenses/by-nd/3.0/legalcode}
\newcommand{\thesiskeywords}{decentralised information systems, distributed systems, privacy, social networks}
\newcommand{\thesiskeywordsabstracten}{\thesiskeywords}
\endinput


\begin{document}
\frontmatter
% \maketitle
% \input{template/doc/latex/kthesis/kth-abs}
% \clearpage
% \selectlanguage{swedish}
% \begin{abstract}
%   Denna fil ger ett avhandlingsskelett.
%   Mer information om \LaTeX-mallen finns i
%   dokumentationen till paketet.
% \end{abstract}
% \selectlanguage{spanish}
% \begin{abstract}
%  Me gusta mucho esto
% \end{abstract}
 \selectlanguage{english}
% \begin{abstract}
%      Pang Ping
% \end{abstract}
\clearpage
\tableofcontents
\mainmatter
\chapter{Introduction}
In 1989, while at the European Particle Physics Laboratory, at \ac{cern}, Tim Berners-Lee 
invented the \Ac{www}. No one at that time would have imagined that such invention 
has been centric to the rapid development of the digital age that we are currently 
witnessing.

Such period in human history, mainly characterised by an economy based on information 
processing as opposed to the traditional industry of the industrial revolution, 
has widen the availability of uncountable services to the general population with 
access to the Internet. Not only new business opportunities have stemmed as technological 
developments matured but also services have been overhauled to make them available 
in the digital world.

To give an example, interpersonal communication --- the passing of information between 
entities --- has evolved throughout history as communication advances developed. 
The traditional exchange of paper letters as a means of communication between two 
parties, which still prevails today, has evolved into a much more sophisticated 
and complex form in the information age: electronic mail (e-mail), which is simply 
the parallelism in the digital era to such physical letters.

Such evolution in the means of communication among individuals and other entities 
has also happened at other societal levels. For example, the social structures that 
individuals form when sharing similar interests or activities, namely social networks, 
have also seen a transposition in the digital era. The Internet and the \ac{www} 
have taken the leading role as tools to convey different types of information from 
one party to another, expanding the boundaries of the concept of social networks, 
heavily based on tangible activity between individuals and entities.

While the concept of social networks is a theoretical term mainly used in the realm 
of social sciences to study and describe the social structures determined by the 
relationships between individuals, groups, and other types of entities such as societies; 
the technological developments of the digital era have popularised its virtual counterpart 
term: \aclp{osn}.

% \acp{osn}, although thought a
%
% %The equivalent of social networks in the digital world, online social netwrosk serve
% %as an alterna
%
% % Social networks
%  is an inherently interdisciplinary academic field which emerged from social psychology, sociology, statistics, and graph theory.
%
%
%
% Social networks, as a bla bla bla, and in particular, online social networks, as
% their equivalent in the digital world (but also as a complement to the tangible
% one) are super awesome and
%
% % Next paragraph is wrong
%
% Such process of information dissemination is one of the basis when forming relations
% among individuals and other entities --- social structures --- resulting in a set
% of social relations commonly known as social networks.
%
%
% people to build social networks or social relations with other people who share similar personal or career interests,[1] activities, backgrounds or real-life connections. The variety of stand-alone and built-in

\section{Motivation}
\section{Outline}
\section{Conventions}
\section{Funding}

\chapter{Research problem}
\section{Challenges addressed}

\chapter{Background}
The field of \acp{sn} is an extremely vast domain of knowledge and research. It 
has influenced dozens of domains, spanning from the most traditional ones such as 
ethnography or anthropology to more contemporanean ones such as genomics or astrophysics.

With the explosion of large scale \acp{osn} new challenges have emerged in connection 
with the sudden availability of new types of data. Moreover, the inherent increase 
in the amount of data poses a technological challenge because there is a need to 
cope with such size, and there is as well a scientifical challenge as to finding 
methods that can answer questions in such large space of data. 

At the same time, the collection of all this data affects privacy.


\section{Information systems: on the importance of capabilities}
% This section seems more appropiate for the phd...
% Reference: Social Networks and Information Systems: Ongoing and Future Research Streams
% Short summary: The IS research drawing on social networks can be divided into the following streams: 1) network awareness at both individual and organizational levels, 2) uses of social network analysis related to IS use, and 3) conceptual and technological change in the fast evolving platforms to manage social networks


\section{Online social networks: centralised vs decentralised}

What a social network is
defintion

How most designs are centralised (and why)
What it means decentralised
How we can decentralise such centralisation
Trade off (this leads to the main point of privacy that comes now)

\section{Privacy: a never ending battle}

What privacy is

Privacy is a fundamental human right recognized in the UN Declaration of Human Rights, the International Convenant on Civil and Political Rights and in many other international and regional treaties.

What it implies
Why is it important
How it can be breached or attacked or diminished
What can we do

\chapter{Methodology and approach}

Design science research methodology

(Mixed Methods Research: A Research Paradigm Whose Time Has Come: http://edr.sagepub.com/content/33/7/14.short)

(A Design Science Research Methodology for Information Systems Research: http://www.tandfonline.com/doi/abs/10.2753/MIS0742-1222240302)

Types of information used/analyzed/gathered/studied and source of such data (random, dataset, etc)

Parts of systems designed and/or analyzed

Prototyping or such



\chapter{Contributions}
\section{Paper A}
\subsection{Summary}
\subsection{Contributions}
%\subsection{Changes for the thesis}
\section{Paper B}
\subsection{Summary}
\subsection{Contributions}
%\subsection{Changes for the thesis}
\section{Paper C}
\subsection{Summary}
\subsection{Contributions}
%\subsection{Changes for the thesis}


\chapter{Conclusions}

\section{Future work}

\end{document}
\endinput
%%
%% End of file `kth-demo.tex'.
