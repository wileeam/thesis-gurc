% -*- mode: TeX -*-
% -*- coding: utf-8 -*-

% *** definitions ***
\newcommand{\etal}{\xspace~et\xspace~al.\xspace}
\newcommand{\eg}{e.\xspace~g.,\xspace} % note the trailing comma (recommended by http://grammar.quickanddirtytips.com/ie-eg-oh-my.aspx )
\newcommand{\Eg}{E.\xspace~g.,\xspace}
\newcommand{\ie}{i.\xspace~e.,\xspace}
\newcommand{\Ie}{I.\xspace~e.,\xspace}
\newcommand{\vs}{vs.\xspace}

% *** proper circled letters command ****
\newcommand*\circled[1]{
	\tikz[baseline=(char.base)]{\node[shape=circle,draw,inner sep=1.5pt] (char) {#1};}}

% *** customisation for pagebackref option in hyperref package ***
% http://tex.stackexchange.com/questions/38149/removing-double-entries-from-hyperrefs-pagebackref
\renewcommand*{\backref}[1]{}
\renewcommand*{\backrefalt}[4]{%
  \ifcase #1 %
    No citations.% use \relax if you do not want the "No citations" message
  \or
    (Cited in pp. #2).%
  \else
    (Cited in pp. #2).%
  \fi%
}

%%%
% DOC: Automatized formatting for quotes :) Usually it will be
%      one quote/citation per chaper, and it should be located
%      right after the chapter's title, \chapter{bla bla bla}
%      Usage: \requote{frase}{autor}
%
%      For instance: \requote{To be... or not to be: that is the question}{William Shakespeare}
%
\newcommand{\requote}[2]{
{\fontfamily{\carlitofamily}\selectfont
    \begin{flushright}
        \begin{minipage}[b][][t]{0.45\paperwidth}
            \small{
                \itshape{\textbf{\normalsize ``}#1\textbf{\normalsize ''}}
                \begin{flushright}
                    \textbf{\textsc{#2}}
                \end{flushright}
            }
        \end{minipage}
        \bigskip
    \end{flushright}
}}

% http://tex.stackexchange.com/questions/254318/chapter-style-with-tcolorbox/254328
% http://tex.stackexchange.com/questions/301185/modify-fancy-chapter-heading-to-show-chapter-name
% \definecolor{titlebgdark}{RGB}{0,163,243}
% \definecolor{titlebglight}{RGB}{191,233,251}
% \definecolor{titlebgdark}{RGB}{25,84,186}
\definecolor{titlebgdark}{RGB}{98,146,46}
% \definecolor{titlebglight}{RGB}{36,160,216}
\definecolor{titlebglight}{RGB}{176,201,43}
\titleformat{\chapter}[display]
  {\normalfont\huge\bfseries}
  {}
  {20pt}
  {%
    \begin{tcolorbox}[
      enhanced,
      colback=titlebgdark,
      boxrule=0.25cm,
      colframe=titlebglight,
      arc=0pt,
      outer arc=0pt,
      leftrule=0pt,
      rightrule=0pt,
      fontupper=\color{white}\sffamily\bfseries\huge,
      enlarge left by=-1in-\hoffset-\oddsidemargin, 
      enlarge right by=-\paperwidth+1in+\hoffset+\oddsidemargin+\textwidth,
      % width=\paperwidth,
      width=\paperwidth+1in,
      height=0.6cm+0.6cm+3\baselineskip,
      valign=center,
      left=1in+\hoffset+\oddsidemargin, 
      % right=\paperwidth-1in-\hoffset-\oddsidemargin-\textwidth,
      right=\paperwidth-1in-\hoffset-\oddsidemargin-\textwidth+1in,
      top=0.6cm, 
      bottom=0.6cm,
      overlay={
        \node[
          fill=titlebgdark,
          draw=titlebglight,
          line width=0.15cm,
          inner sep=0pt,
          text width=1.7cm,
          minimum height=1.7cm,
          align=center,
          font=\color{white}\sffamily\bfseries\fontsize{30}{36}\selectfont
        ] 
        (chapname)
        % at ([xshift=-1in]frame.north east)
        at ([xshift=-1in-1in]frame.north east)
        {\thechapter};
        \node[font=\small,anchor=south,inner sep=2pt] at (chapname.north)
        % {\MakeUppercase\chaptertitlename};
        {};  
      } 
    ]
    #1
    \end{tcolorbox}%
  }
\titleformat{name=\chapter,numberless}[display]
  {\normalfont\huge\bfseries}
  {}
  {20pt}
  {%
    \begin{tcolorbox}[
      enhanced,
      colback=titlebgdark,
      boxrule=0.25cm,
      colframe=titlebglight,
      arc=0pt,
      outer arc=0pt,
      remember as=title,
      leftrule=0pt,
      rightrule=0pt,
      fontupper=\color{white}\sffamily\bfseries\huge,
      enlarge left by=-1in-\hoffset-\oddsidemargin, 
      enlarge right by=-\paperwidth+1in+\hoffset+\oddsidemargin+\textwidth+1in,
      % width=\paperwidth,
      width=\paperwidth+1in,
      height=0.6cm+0.6cm+3\baselineskip,
      valign=center,
      left=1in+\hoffset+\oddsidemargin, 
      % right=\paperwidth-1in-\hoffset-\oddsidemargin-\textwidth,
      right=\paperwidth-1in-\hoffset-\oddsidemargin-\textwidth+1in,
      top=0.6cm, 
      bottom=0.6cm, 
    ]
    #1
    \end{tcolorbox}%
  }
\titlespacing*{\chapter}
  {0pt}{0pt}{10pt}
\makeatother

% Make format of TOC heading same as the one used in the chapter
% http://tex.stackexchange.com/a/19484/120692
\makeatletter
\renewcommand\tableofcontents{%
        \chapter{\contentsname
        \@mkboth{%
           \MakeUppercase\contentsname}{\MakeUppercase\contentsname}}%
        \@starttoc{toc}%
        }
\makeatother

% http://tex.stackexchange.com/questions/54115/how-to-let-part-stay-solo-page-and-no-page-number
% \makeatletter
% \renewcommand\part{%
%   \if@openright
%     \cleardoublepage
%   \else
%     \clearpage
%   \fi
%   \thispagestyle{empty}%   % Original »plain« replaced by »emptyx
%   \if@twocolumn
%     \onecolumn
%     \@tempswatrue
%   \else
%     \@tempswafalse
%   \fi
%   \null\vfil
%   \secdef\@part\@spart}
% \makeatother

% http://tex.stackexchange.com/questions/159551/customizing-part-style-with-tikz
% \definecolor{mybluei}{RGB}{0,173,239}
% \definecolor{mybluei}{RGB}{25,84,186}
% \definecolor{myblueii}{RGB}{63,200,244}
% \definecolor{myblueiii}{RGB}{199,234,253}
% \definecolor{myblueiii}{RGB}{36,160,216}
\definecolor{bgdark}{RGB}{98,146,46}
\definecolor{bglight}{RGB}{176,201,43}
\renewcommand\partnumfont{% font specification for the number
    % \fontsize{380}{130}\color{myblueii}\selectfont%
    \fontsize{380}{130}\color{bgdark}\sffamily\selectfont%
}

\renewcommand\partnamefont{% font specification for the name "PART"
    % \normalfont\color{white}\scshape\small\bfseries
    \color{white}\sffamily\bfseries\fontsize{30}{36}\selectfont
}

\titleformat{\part}
  {\normalfont\huge\filleft}
  % {\sffamily\bfseries\normalfont\huge\filleft}
  {}
  {20pt}
  {\begin{tikzpicture}[remember picture,overlay]
  \fill[bglight] 
    (current page.north west) rectangle ([yshift=-13cm,xshift=1in]current page.north east);   
  \node[
      fill=bgdark,
      text width=2\paperwidth,
      rounded corners=6cm,
      text depth=18cm,
      anchor=center,
      inner sep=0pt] at (current page.north east) (parttop)
    {\thepart};%
    {};
  \node[
      anchor=south east,
      inner sep=0pt,
      outer sep=0pt] (partnum) at ([xshift=-20pt]parttop.south) 
    {\partnumfont\thepart};
    % {\partnumfont};
  \node[
      anchor=south,
      inner sep=0pt] (partname) at ([yshift=2pt]partnum.south)   
  % {\partnamefont PART};
  {\partnamefont};
  \node[
      anchor=north east,
      align=right,
      inner xsep=10pt] at ([yshift=-1.5cm]partname.east|-partnum.south) 
  % {\parbox{.7\textwidth}{\raggedleft#1}};
  {\parbox{.7\textwidth}{\raggedleft\sffamily\bfseries\Huge\selectfont #1}};
  \end{tikzpicture}%
  }


\makeatletter
\newcommand*{\greek}[1]{%
   \expandafter\@greek\csname c@#1\endcsname
}
\newcommand*{\@greek}[1]{%
   $\ifcase#1\or\alpha\or\beta\or\gamma\or\delta\or\varepsilon
     \or\zeta\or\eta\or\theta\or\iota\or\kappa\or\lambda
     \or\mu\or\nu\or\xi\or o\or\pi\or\varrho\or\sigma
     \or\tau\or\upsilon\or\phi\or\chi\or\psi\or\omega
     \else\@ctrerr\fi$%
}
\makeatother

\newcommand\firstToLow[1]{%
 {%
   \renewcommand{\mfirstucMakeUppercase}{\MakeLowercase}%
   \makefirstuc{#1}%
 }%
}

\makeatletter
\newcommand*{\lcnameref}[1]{%
 \begingroup 
   \let\label\@gobble
   \NR@setref{#1}\lc@thirdoffive{#1}%
  \endgroup
}
\newcommand{\lc@thirdoffive}[5]{\firstToLow{#3}}
\makeatother

% reset footnote counters per chapter (and chapter*)
\makeatletter
\pretocmd{\@schapter}{\setcounter{footnote}{0}}{}{}
\pretocmd{\@chapter}{\setcounter{footnote}{0}}{}{}
\makeatother

% correct case for namerefs, see http://tex.stackexchange.com/questions/1655/correct-case-in-namerefs
\newcommand{\lnameref}[1]{%
\bgroup
\let\nmu\MakeLowercase
\nameref{#1}\egroup}
\newcommand{\fnameref}[1]{%
\bgroup
\def\nmu{\let\nmu\MakeLowercase}%
\nameref{#1}\egroup}

\newcommand{\nmu}{}

%introduce \powerset - hint by http://matheplanet.com/matheplanet/nuke/html/viewtopic.php?topic=136492&post_id=997377
\DeclareFontFamily{U}{MnSymbolC}{}
\DeclareSymbolFont{MnSyC}{U}{MnSymbolC}{m}{n}
\DeclareFontShape{U}{MnSymbolC}{m}{n}{
    <-6>  MnSymbolC5
   <6-7>  MnSymbolC6
   <7-8>  MnSymbolC7
   <8-9>  MnSymbolC8
   <9-10> MnSymbolC9
  <10-12> MnSymbolC10
  <12->   MnSymbolC12%
}{}
\DeclareMathSymbol{\powerset}{\mathord}{MnSyC}{180}


\newtcolorbox{ppBox}[1][]{
    breakable,
    enhanced,
    skin=enhancedmiddle,
    frame hidden,
    interior hidden,
    title=#1,
    arc=0pt,
    outer arc=0pt,
    top=0pt,
    bottom=0pt,
    boxsep=1mm,
    borderline={1.5pt}{0mm}{black},
    fonttitle=\sffamily\footnotesize,
    fontupper=\sffamily\footnotesize,
    fontlower=\sffamily\footnotesize,
    colframe=black,
    colback=white,
    colbacktitle=white,
    coltitle=black,
    fonttitle=\bfseries,
    boxrule=0pt,
    bottomrule=0pt,
    toprule=0pt,
  % leftrule=3pt,
  % rightrule=3pt,
    titlerule=0pt,
    width=8\linewidth/10,
    valign=center,
}

